\documentclass[11pt,a4paper,twoside]{book}
\usepackage[utf8]{inputenc}
\usepackage[spanish]{babel}
\usepackage{amsmath}
\usepackage{graphicx}
\usepackage{amsfonts}
\usepackage{amssymb}
\usepackage[left=2cm,right=2cm,top=2cm,bottom=2cm]{geometry}
\usepackage{float}
\usepackage{pdfpages}
\usepackage{longtable}
\author{Víctor de Tejada Molera}
\begin{document}
\chapter*{Resumen}
    La psicoacústica es una rama del conocimiento que estudia cómo percibimos el sonido las personas. Este campo es muy amplio y tiene numerosas aplicaciones como puede ser la acústica arquitectónica, los videojuegos o la industria audiovisual, por citar algunos ejemplos.

    La forma más habitual de trabajo en este campo es mediante la realización de test subjetivos que ayudan a validar o descartar hipótesis sobre la percepción auditva. Estos test se componen de un determinado número de personas que se ven expuestas a estímulos sonoros y deben responder a una serie de cuestiones sobre su percepción de dichos estímulos. Para su correcta realización, antes han de determinarse algunas cuestiones como qué se pretende evaluar, el número de participantes, el análisis estadístico a realizar, etc. Estas características resultan claves para conseguir unos datos útiles a la hora de extraer conclusiones.

    En este proyecto, se pretende estudiar cómo varía la percepción de los participantes con la escucha de las respuestas al impulso biaurales de una sala. Estas respuestas se han obtenido variando la posición de recepción, mientras se mantiene estática la posición de la fuente sonora. De esta forma, se ha logrado simular cómo una persona escucharía las respuestas al impulso si se encontrara sentado en distintas posiciones dentro del patio de butacas de un auditorio.

    Para lograr este objetivo, se establece una primera toma de datos con el objetivo de realizar un test preliminar con el que llevar a cabo una primera comprobación de hipótesis. 

    A continuación, se efectúa una segunda toma de datos con la que, utilizando unos micrófonos biaurales, se obtienen las señales con las que 34 personas realizan el test definitivo.

    De forma paralela, se ha ido desarrollando una aplicación en el lenguaje de programación Python que dispone de una interfaz gráfica, así como el resto de herramientas necesarias, para que los participantes respondan los test.

    Con los datos obtenidos se realizan dos tipos de análisis de datos: por un lado se lleva a cabo un análisis más convencional utilizando la norma UNE-EN ISO 10399. Esta norma determina la probabilidad de la hipótesis de partida sea errónea en función del número de respuestas ``correctas'' y el número total de respuestas. Por otro lado, se completa un análisis mediante \textit{modelos thurstonianos} que permiten efectuar comparaciones cuantitativas entre agrupaciones de datos, a priori, independientes.

    Finalmente, con estos dos análisis se extraen las conclusiones proyecto, así como posibles mejoras y aplicaciones para futuros estudios.
    
    \chapter*{Abstract}
    Psychoacoustics is an area of knowledge that studies how people perceive sound. This field is broad and has numerous applications such as architectural acoustics, video games or the audiovisual industry, to name a few examples.

    The most common way of working this field is by performing subjective tests. These are composed of a certain number of people who are exposed to stimuli and must answer a series of questions about their perception about these stimuli. In order to achieve this purpose, certain questions must be solved, such as what is to be evaluated, the number of participants, the statistical analysis to be used, etc. These characteristics are key to obtain useful data for drawing conclusions.

    In this project, the objective is to study how the perception of the participants varies with the listening of the responses to the binaural impulse of a room. These responses have been obtained by varying the reception position, while keeping the position of the sound source static. By doing this, it has been possible to simulate how a person would listen to the impulse responses if he/she were seated in different positions within the stalls of an auditorium.

    To achieve this objective, a first collection of data was carried out in order to perform a preliminary test which will help us to check an initial hypothesis. 

    Then, a second data collection is produced, using binaural microphones, to pick up the signals with which the definitive test is made.

    At the same time, an application has been developed using the Python programming language, as well as the rest of necessary tools, to ensure the correct performance of the participants during the tests.

    Two types of data analysis are performed: on the one hand, a more conventional analysis is made using the UNE-EN ISO 10399 standard. This standard determines the probability of making a wrong statement according to the number of ``right'' answers and the total number of answers. On the other hand, another analysis is made using Thurstonian models that allow quantitative comparisons between independent groups of data.

    Finally, with these two analyses, conclusions are obtained, as well as possible improvements and applications for future studies.
    
    
\end{document}