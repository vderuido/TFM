\documentclass[11pt,a4paper,twoside]{book}
\usepackage[utf8]{inputenc}
\usepackage[spanish]{babel}
\usepackage{amsmath}
\usepackage{graphicx}
\usepackage{amsfonts}
\usepackage{amssymb}
\usepackage{cite}
\usepackage[left=2cm,right=2cm,top=2cm,bottom=2cm]{geometry}
\usepackage{float}
\author{Víctor de Tejada Molera}
\begin{document}
    \chapter{Realización de los test subjetivos de audio}
        Como ya se estableció en el capítulo de ``Introducción'', el objetivo del proyecto es el de analizar las diferencias perceptuales que se obtienen al escuchar las respuestas al impulso obtenidas al modificar la posición de escucha o recepción dentro de un determinado espacio.
        
        Para simplificar la toma de datos del estudio, se parte de la premisa de que los oyentes perciben diferencias sonoras en función de la distancia a la que se encuentre la fuente sonora del receptor, pero que dichas percepciones son similares para posiciones simétricas dentro del espacio de estudio.
        
        Para validar esta hipótesis, que permitiría que las muestras recogidas fueran de sólo la mitad del auditorio, se plantea un test previo que se describe en el apartado ``Test preliminar''
        
        Una vez comprobada la hipótesis y realizada la adquisición de todos los datos necesarios, se diseña un segundo test más completo y con una interfaz de usuario más refinada a partir de las experiencias adquiridas en el test anterior. No obstante, por los motivos ya comentados en el apartado 2.4., se estimó conveniente repetir la toma de datos y realizar dicho test de nuevo. En el apartado ``Características de los test finales'' se explican las particularidades de ambas tomas de datos y sus respectivas diferencias.
        \section{Test preliminar}
            En este apartado se explican las características del test previo que se realiza para determinar que la hipótesis inicial es, a priori, correcta. Esta hipótesis afirma que las dimensiones y características de la sala permiten que la distancia sea un factor perceptible por los oyentes.
            
            \subsection{Características del test previo}
                Para la realización de este test, se han seguido las recomendaciones propuestas en \cite{Tejada2020}. Para ello, se plantean las siguientes preguntas:
                \begin{itemize}
                    \item \textbf{¿En qué categoría se engloba el estudio?}: Para responder a esta pregunta, hay que observar las diferentes categorías que se presentan en \cite{Tejada2020}. Estas categorías son: Sensibilidad, Localización, Molestia, Inteligibilidad y Reverberación. Para el caso de este test, el que más se corresponde es el de Sensibilidad, puesto que el objetivo es ver si las personas participantes tienen la capacidad para distinguir si los sonidos son iguales o no.
                    \item \textbf{¿Qué tipo de test es el más adecuado?}: Para los objetivos de este test, se ha optado por un test 2-AFC también conocido como \textit{Same/Different}. Este tipo de test obliga al participante a elegir entre dos opciones; en nuestro caso, si los audios que se presentan son iguales o diferentes. Esto tiene la ventaja de que es fácil de codificar y analizar estadísticamente aunque no da mucha información concreta, pero es suficiente para el objetivo del proyecto.
                    \item \textbf{¿Participantes expertos o inexpertos?,¿cuántos necesito?}: Como este test no es el definitivo, sino que se trata de una comprobación previa, no se han considerado requisitos tan estrictos como los que se hubieran seguido de otra forma. Por este motivo, se ha optado por utilizar participantes con y sin experiencia. El número total de participantes ha sido de 10 que es lo mínimo que se ha utilizado en algunos test de temática similar \cite{2019MNowak}.
                    \item \textbf{¿Qué tipo de señal utilizo?}: En nuestro caso, se utilizan las respuestas al impulso convolucionadas obtenidas previamente del salón de actos siguiendo el procedimiento seguido en el capítulo de "Toma de datos". En concreto para este experimento se han utilizado las respuestas de las zonas más cercanas y lejanas a la fuente en las posiciones centrales y de los extremos laterales tanto izquierdo como derecho. En la figura \ref{fig:posicionesTestPrueba} se muestran las posiciones que formaban parte del test.
                    
                    \begin{figure}[H]
                    \includegraphics[scale=0.1]{../imagenes/auditorio_butacas_prueba.png}
			        \centering
			        \caption{Posiciones de las grabaciones utilizadas para el test de prueba.}
			        \label{fig:posicionesTestPrueba}
                \end{figure}
                    
                    \item \textbf{¿Cuál es tipo de análisis de datos?}: Para el análisis de los datos, se ha optado por realizar un análisis del coeficiente $\alpha$ de acuerdo con las especificaciones de la norma ISO-10399 \cite{ISO10399}. En dicha norma, se proporciona una tabla que muestra el número de respuestas correctas necesarias para concluir que dos eventos son diferentes con una determinada incertidumbre para un determinado número de respuestas totales.
                \end{itemize}
 
            \subsection{Procedimiento del test previo}
                Una vez determinadas todas las características del test previo, se procede a completar el test de forma individual a cada uno de los diez participantes. La sala donde se realizaron las escuchas fue una de las salas de laboratorio del Centro de Investigación en Tecnologías Software y Sistemas Multimedia para la Sostenibilidad (CITSEM). Esta sala se encuentra aislada de las zonas de tránsito y sus ventanas no dan a zonas con tráfico. De esta forma, se pueden controlar fácilmente las fuentes de ruido ajenas al experimento. Es importante remarcar que, debido a la pandemia producidad por la COVID-19, las ventanas tenían que estar abiertas en todo momento con el fin de conseguir unas condiciones de ventilación apropiadas que evitaran posibles contagios. Este es el motivo por el que se optara por una sala cuyas ventanas no dieran a fuentes de ruido externas.
                
                El equipo utilizado consistió en un ordenador portátil con sistema operativo Debian con 12GB de memoria RAM y unos auriculares Tascam-TH02. La escucha de los diferentes audios y la inclusión de las respuestas se realiza mediante la interfaz gráfica que ya se explicó en el apartado ``Desarrollo de aplicación en Python''. En la figura \ref{fig:interfazInicial} se puede observar la interfaz utilizada en este test. Previa a las escuchas por parte de los participantes, se comprobó que las señales se reprodujeran con un nivel adecuado. Además, se les explicaba a los participantes las carácterísticas del test que iban a realizar, así como el número de preguntas y en qué propiedades debían fijarse para facilitarles su realización. También se les pedía que leyeran y firmaran un ``consentimiento informado'' cuya estructura puede consultarse en el anexo D. Una vez finalizada la realización del test y, debido al carácter preliminar del mismo, se les preguntó sobre posibles mejoras que se podían hacer para facilitar su realización, como cambios en la interfaz gráfica, la inclusión de audios de ejemplo, etc.
                
                 \begin{figure}[H]
                    \includegraphics[scale=0.6]{../imagenes/uiIni.png}
			        \centering
			        \caption{Interfaz gráfica para la realización del test previo.}
			        \label{fig:interfazInicial}
                \end{figure}
                
        \section{Experimentos finales}
            En este apartado se analizan las características y procedimientos utilizado en los experimentos finales.     
            
            \subsection{Características de los test finales}
                Como ya se comentó en el análisis del estado del arte, no existe un consenso sobre la mejor forma de realizar un test subjetivo de audio. Algunos estudios como \cite{Tejada2020}, \cite{delaPrida2019} y \cite{delaPrida2021} hacen sugerencias sobre qué características pueden funcionar mejor en función de lo que se pretenda analizar en dicho test. Para nuestro caso particular, se ha vuelto a optar por seguir las recomendaciones propuestas en \cite{Tejada2020} respondiendo a las preguntas que en él se plantean. Como ya se comentó en el apartado de ``Toma de datos'', se tuvieron que repetir los test de audio debido a un fallo en el procesado de las grabaciones. A pesar de esto, las características de ambas pruebas son prácticamente idénticas. Por este motivo, se engloban todas ellas en el mismo apartado. Dentro de cada subapartado se comentan, si existen, las diferencias que pueda haber entre ambas.
            
                \begin{itemize}
                
                    \item \textbf{¿En qué categoría se engloba el estudo?}: Al igual que en el test previo, se pretende determinar la sensibilidad que tienen las personas para distinguir las diferencias perceptuales al escuchar sonidos cuando se encuentran en distintas posiciones respecto de la fuente acústica. Por este motivo, se puede concluir que este test se encuentra englobado dentro de los que se consideran como de Sensibilidad.
                    \item \textbf{¿Qué tipo de test es el más adecuado?}: Para el estudio que se pretende hacer, existen numerosos tipos de test que pueden arrojar datos interesantes. Entre ellos, los dos más habituales son los de referencia oculta como ABX o los Duo-Trio. No obstante, se ha optado por repetir el tipo de test utilizado en el estudio previo; es decir 2-AFC o \textit{Same/Different}. Los motivos que han llevado a esta elección han sido la facilidad de implementación y codificación, su simpleza que lo hace apropiado para todo tipo de participantes y que permite arrojar resultados bastante fiables mediante un análisis relativamente sencillo usando modelos Thurstonianos, además de que existen normas internacionales que permiten obtener resultados extra que pueden ser interesantes de comparar con los de dichos modelos.
                    \item \textbf{¿Participantes expertos o inexpertos?, ¿cuántos necesito?}: A la hora de plantear el experimento, desde un primer momento se planteó como una situación que afectaba a la población general. Por este motivo, se consideró que no era relevante que los participantes tuvieran una experiencia particular en la realización de este tipo de test, aunque obviamente, es necesario que no padezcan de ningun problema auditivo que afecte a su nivel de audición, ya sea por la edad, enfermedad, etc. A pesar de todo esto, se consideró que todas personas participantes debían pasar por una breve sesión de entrenamiento previo para que supieran de antemano en qué elementos de los sonidos tenían que fijarse. Estas indicaciones vienen reflejadas en las normas \cite{UIT1116, UIT1534, UIT1284, EBU3286, UIT1285, UIT1286} que muestran la utilidad de este tipo de entrenamientos para obtener mejores resultados.
                
                    En cuanto al número de participantes totales, se siguieron las recomendaciones de las mismas normas, así como las conclusiones extraídas de \cite{Tejada2020} y de los otros experimentos consultados de temática similar \cite{2005IWitew, 2019DJSchlit, 2016SKlockgether, 2019LKritly, 2019GPulvirenti, 2019MNowak, 2011VEmiya}. De esta forma, se concluyó que el número de participantes debía de ser de un mínimo de 30 personas.
                    
                    
                    \item \textbf{¿Qué tipo de señal utilizo?}: Las señales utilizadas para la realización del primer test fueron los archivos de audio generados tras la convolución de una voz femenina, con cada una de las respuestas impulsivas grabadas en el salón de actos de la ETSIST de la UPM. El proceso de obtención de dichas respuestas se encuentra explicado en el apartado de ``Toma de datos''. Se trata de un total de 100 archivos de audio de 7 segundos de duración. Esta duración se encuentra dentro de los límites marcados por \cite{UIT1116, UIT1534, UIT1284, EBU3286, UIT1285, UIT1286} para evitar que los oyentes se ``acostumbren'' al  sonido. Así mismo, se encuentra dentro del abánico de duraciones que se ha observado en distintos proyectos de similares características. \newline
                
                    Para el segundo test, se optó por utilizar diréctamente las respuestas al impulso obtenidas tras el procesado que realiza el software \textit{Dirac}, como ya se explica también en el apartado de ``Toma de datos''. En este caso, se utilizan un total de 83 archivos distintos de 2 segundos de duración. Al igual que en la prueba anterior, la duración se encuentra dentro de los límites marcados por \cite{UIT1116, UIT1534, UIT1284, EBU3286, UIT1285, UIT1286}, así como otros estudios para evitar la fatiga auditiva y que los participantes se acostumbren a los estímulos.
                
                    Para la realización de ambos experimentos, el volumen al cual se reproducen las señales no es relevante, siempre y cuando no se modifique a lo largo de toda la sesión. Por este motivo, el participante puede escoger el que le sea más cómodo, con la única restricción de que el volumen debe ser el mismo para la escucha de los dos audios que conforman cada pregunta.
                    \item \textbf{¿Cuál es el tipo de análisis de datos?}: para los test finales se han optado por dos análisis diferentes. Por un lado, se mantiene el análisis del coeficiente $\alpha$ siguiendo las especificicaciones de la norma ISO-10399\cite{ISO10399} y las tablas que ofrece en sus anexos. Por otro lado, se realiza un análisis mediante modelos thurstonianos, ya reseñados en el capítulo de ``Estado del arte''. Ambos análisis, el de la norma ISO-10399 y los modelos thurstonianos se realizan para dos diferentes disposiciones de los datos: según la distancia del punto medio de la posición de los dos audios respecto de la fuente, y según la distancia a la que se encuentre la posición de ambos audios.
                     
                \end{itemize}
                
                Otro de los aspectos claves a determinar es la duración del test y el número de preguntas (o rondas) que tendrán que realizar los partcipantes. Las normativas internacionales y diferentes fuentes bibliográficas \cite{UIT1116, UIT1534, UIT1284, EBU3286, UIT1285, UIT1286, ZwickerFactsModels, BlauertSpatialHearing, GelfandStanley}, limitan a 30 minutos el tiempo máximo que los participantes pueden invertir en los test sin la utilización de descansos y evitar así la fatiga auditiva. Esto limita el número de rondas que los participantes son capaces de responder en dicho tiempo y, por este motivo, es necesario ajustar la cantidad de preguntas para que se ajuste lo mejor posible a una duración de 20-25 minutos y así tener margen hasta la media hora límite.
                
                A raíz del tiempo que llevó a los voluntarios del test previo completar sus preguntas, se decidió que los test finales constaran de 25 rondas o preguntas. De esta forma se satisfacieron las limitaciones de tiempo y nuestra necesidad de obtener el mayor número de datos posible.
                
            \subsection{Procedimiento del test final}
        
                \subsubsection*{Localización}
                    En primer lugar, se organizaron los turnos para la realización del test. El espacio escogido para la realización de las escuchas consistió en un laboratorio del CITSEM localizado en el Campus Sur de la UPM. Se escogió esta localización porque, debido a la situación pandémica producida por el COVID-19, era necesario que las ventanas estuvieran abiertas en todo momento. Se planteó la utilización de aparatos de depuración del aire, pero su ruido era comparable o más molesto todavía que la realización del mismo con las ventanas abiertas. No obstante, se escogió una localización donde las ventanas daban a una zona poco transitada, tanto por peatones como por vehículos, de forma que se conseguía un ambiente suficientemente controlado y silencioso. Si antes de comenzar el test se consideraba que el ruido externo era mayor del habitual, se cerraban las ventanas durante su realización y se volvían a abrir al finalizar el experimento.
            
                \subsubsection*{Material utilizado}
                    Para la realización de ambos test se utilizó el mismo material que en el test preliminar. En concreto:
                    \begin{itemize}
                        \item Ordenador portátil con S.O. Debian 11 con procesador \textit{Intel} i5 y 12GB de RAM.
                        \item Software \textit{Gnome Builder} para la ejecución de la aplicación.
                        \item Auriculares Tascam-TH02.
                        \item Declaraciones de consentimiento.
                        
                    \end{itemize}
                
                \subsubsection*{Realización del test}
                    En primer lugar, todas las personas participantes debían inscribirse en las franjas habilitadas en un calendario con el fin de poder organizar las entradas y salidas evitando que se produjeran acumulaciones de personas en espacios cerrados. Una vez inscritas y cuando llegaba su turno, se les explicaba brevemente en qué consistía el experimento del que iban a formar parte y se les daba a leer una hoja de información al participante en la que se explica en mayor detalle las particularidades del test, las partes de las que consta, así como información relevante en temas de salud y protección de datos. Una vez leído, debían rellenar una declaración responsable aceptando los aspectos marcados en el documento anterior. Los modelos de ambos documentos pueden consultarse en su totalidad en el Anexo D.
                
                    A continuación daba comienzo el test con el tutorial. En él se le presentan al participante tres de las posibilidades que podría encontrarse (que los audios fueran de la misma posición, que fueran de filas distintas y que fueran de butacas distintas de la misma fila). Como el objetivo de este tutorial es que sepan identificar qué aspectos determinan que dos audios sean iguales o diferentes, para los tres casos se optó por utilizar casos críticos, estos son los mismos que se utilizaban en el test previo, donde las posiciones estaban lo más separadas posibles (ver figura \ref{fig:posicionesTestPrueba}). La reproducción de los audios se podía repetir el número de veces que se considerara necesario. En la figura \ref{fig:interfazTutorial} puede observarse la interfaz de dicha parte.
                
                    \begin{figure}[H]
                        \includegraphics[scale=0.6]{../imagenes/interfaz_tutorial.png}
			            \centering
			            \caption{Apariencia del entrenamiento previo}
			            \label{fig:interfazTutorial}
                    \end{figure}
                
                    Una vez el participante consideraba que estaba preparado, pulsaba el botón de ``Empezar test'' para dar inicio a la segunda parte.
                
                    En esta segunda parte al usuario se le presentan dos audios, escogidos de forma aleatoria, de todos las posicions posibles de la sala y debe determinar si lo que percibe son dos audios que suenan igual o no. Por otro lado, el participante debe especificar si se encuentra seguro de su respuesta. El usuario puede repetir la reproducción de los audios las veces que quiera, pero debe esperar a que el audio termine de reproducirse antes de poder escoger si repetirlo o escuchar el otro audio. No existe la opción de ``No sabe/no contesta'', por lo que en caso de no poder decidirse, el usuario debe escoger una de las dos opciones de forma aleatoria y marcar la opción de ``no seguro''. Una vez ha decidido su respuesta, debe pinchar en el botón de ``Enviar respuesta'' para que se le presenten los dos siguientes audios. Esto permite que, en caso de duda, los participantes puedan modificar su respuesta hasta que pulsen dicho botón. A partir de ahí, no se permite volver atrás o modificar las respuestas. El experimento termina tras las 25 rondas de preguntas. Una vez terminada esta parte, el trabajo de las personas voluntarias ha terminado y puede marcharse para dar paso al siguiente participante. En todos los casos, se dispuso de un tiempo entre cada test para seguir los protocolos establecidos por la Covid-19 (Ventilación del espacio, limpieza de todos los elementos, etc.). En la figura \ref{fig:interfazTestFin} puede observarse la interfaz utilizada para la realización de esta segunda parte.
                
                    \begin{figure}[H]
                        \includegraphics[scale=0.6]{../imagenes/interFin.png}
			            \centering
			            \caption{Apariencia de la interfaz del test.}
			            \label{fig:interfazTestFin}
                    \end{figure}
                
                    Todos los resultados se almacenan durante la ejecución del experimento en un archivo con terminación ``.csv'', que es el que se utiliza posteriormente para el análisis de los datos. Estos datos se disponen de forma anónima en el Anexo F.
                
                    Para la realización del test participaron un total de 31 voluntarios para la primera versión de este test y 34 en la segunda. Las edades del primer test estaban comprendidas entre los 18 y los 60 años, mientras que la segunda las edades se encontraban en el rango de los 18 a los 33 años. En ningún momento la duración de los test superó los 30 minutos. En la figura \ref{fig:participantes} se puede observar una imagen de varios de los partcipantes durante el transcurso del test.
                
                    \begin{figure}[H]
                        \includegraphics[scale=0.45]{../imagenes/participantes.png}
			            \centering
			            \caption{Varios de los participantes durante el transcurso de los test.}
			            \label{fig:participantes}
                    \end{figure}
            
            
\bibliography{biblio.bib}
\bibliographystyle{ieeetr}
\end{document}