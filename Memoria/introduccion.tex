\documentclass[11pt,a4paper,twoside]{book}
\usepackage[utf8]{inputenc}
\usepackage[spanish]{babel}
\usepackage{amsmath}
\usepackage{graphicx}
\usepackage{amsfonts}
\usepackage{amssymb}
\usepackage[left=2cm,right=2cm,top=2cm,bottom=2cm]{geometry}
\usepackage{float}
\usepackage{pdfpages}
\usepackage{longtable}
\author{Víctor de Tejada Molera}
\begin{document}
\chapter{Introducción}
    La respuesta al impulso de una sala es como se denomina al comportamiento acústico que se obtiene en una determinada posición de recepción cuando es espacio es excitado con una señal del tipo ``Delta'' desde una posición de emisión. Este tipo de respuestas tienen una gran importancia ya que aportan información sobre la reverberación del espacio, la inteligibilidad, la capacidad para realizar localizaciones acústicas dentro del recinto, etc. Por este motivo, se pueden considerar como una de las señales más útiles para la caracterización acústica de un determinado espacio. 
    
    A pesar de todas estas bondades, las respuestas al impulso tienen una gran limitación: sólo aportan información sobre cómo se comporta el espacio para el par de puntos que se fijan como punto de emisión y de recepción. Por este motivo, si se quiere caracterizar de forma adecuada un recinto, se deben tomar un elevado número de respuestas de forma que se consiga obtener un número de puntos de recepción suficientes.
    
    ¿Pero cuántas posiciones son realmente necesarias? Si se analiza un espacio como un aula, un auditorio o una sala de conferencias, las posiciones de emisión lógicas vienen marcadas por las posiciones donde estarían colocados los ponentes o los músicos, pero con las posiciones de recepción la situación se complica puesto que es habitual que los espacios cuenten con cientos o miles de asientos para el público. Para solucionar esta pregunta conviene estudiar cómo varía la percepción de las personas y comprobar si existen posiciones para los que los participantes son incapaces de distinguir si están en una posición u otra.
    
    Para resolver estas cuestiones entra en juego la Psicoacústica. La psicoacústica es el campo de la ciencia que estudia cómo las personas perciben el sonido y la forma más habitual de trabajo es mediante la realización de test subjetivos de audio. Estos test se componen de un determinado número de personas que se ven expuestos a estímulos (acústicos, pero a veces también visuales), y deben responder a una serie de cuestiones sobre su percepción sobre dichos estímulos.
    
    La psicoacústica, como se puede suponer, depende en gran medida de procesos estocásticos para la elaboración de sus conclusiones. Por este motivo, es importante disponer de un gran número de participantes para que los resultados tengan relevancia a nivel estadístico. Si el número de participantes fuera pequeño, las desviaciones obtenidas tendrían valores muy elevados y los resultados podrían estar fuertemente influidos por el azar, haciendo muy complicado volver a obtener valores similares y que otros estudios fueran capaces de reproducir el mismo experimento.
    
\end{document}