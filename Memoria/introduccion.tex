\documentclass[11pt,a4paper,twoside]{book}
\usepackage[utf8]{inputenc}
\usepackage[spanish]{babel}
\usepackage{amsmath}
\usepackage{graphicx}
\usepackage{amsfonts}
\usepackage{amssymb}
\usepackage[left=2cm,right=2cm,top=2cm,bottom=2cm]{geometry}
\usepackage{float}
\usepackage{pdfpages}
\usepackage{longtable}
\author{Víctor de Tejada Molera}
\begin{document}
\chapter{Introducción}
    La respuesta al impulso de una sala se denomina como el comportamiento acústico que se obtiene en una determinada posición de recepción, cuando ese espacio es excitado con una señal del tipo ``Delta'', desde una posición de emisión. Este tipo de respuestas tienen una gran importancia ya que aportan información sobre la reverberación del espacio, la inteligibilidad, la capacidad para realizar localizaciones acústicas dentro del recinto, etc. Por este motivo, se pueden considerar como una de las señales más útiles para la caracterización acústica de un determinado espacio. 
    
    A pesar de todas estas bondades, las respuestas al impulso tienen una gran limitación: sólo aportan información sobre cómo se comporta el espacio para el par de puntos que se fijan como punto de emisión y de recepción. Por tanto, si se quiere caracterizar de forma adecuada un recinto, se deben tomar un elevado número de respuestas de forma que se consiga obtener un número de puntos de recepción suficientes.
    
    ¿Pero cuántas posiciones son realmente necesarias? Si se analiza un espacio como un aula, un auditorio o una sala de conferencias, las posiciones de emisión lógicas vienen marcadas por las posiciones dónde estarían colocados los ponentes o los músicos. Con las posiciones de recepción la situación se complica, puesto que es habitual que los espacios cuenten con cientos o miles de asientos para el público. Para solucionar esta pregunta, conviene estudiar cómo varía la percepción de las personas y comprobar si existen posiciones para los que los participantes son incapaces de distinguir si están en una posición u otra.
    
    Para resolver estas cuestiones entra en juego la \textit{Psicoacústica}. La psicoacústica es el campo de la ciencia que estudia cómo las personas perciben el sonido, y la forma más habitual de trabajo es mediante la realización de test subjetivos de audio. Estos test se componen de un determinado número de personas que se ven expuestas a estímulos (acústicos, pero a veces también visuales), y deben responder a una serie de cuestiones sobre su percepción sobre dichos estímulos.
    
    La psicoacústica, como se puede suponer, depende en gran medida de procesos estadísticos para la elaboración de sus conclusiones. Por este motivo, es importante disponer de un gran número de participantes para que los resultados tengan relevancia estadística. Si el número de participantes fuera pequeño, las desviaciones obtenidas tendrían valores muy elevados y los resultados podrían estar fuertemente influidos por el azar, haciendo muy complicado volver a obtener valores similares y que otros estudios fueran capaces de reproducir el mismo experimento.
    
    Para conseguir unos resultados que sean válidos, se debe, además, analizar qué tipo de participantes se busca que respondan al test. La utilización de personas con experiencia o conocimientos relacionados con la materia de estudio, puede hacer que el personal necesario se reduzca notablemente, o que el número de preguntas pueda aumentarse si se supone que responden a las preguntas más rápidamente. No obstante, pueden darse casos en los que resulte más útil la utilización de participantes sin experiencia para recoger un comportamiento más general.
    
    También resulta importante determinar un buen sistema para el análisis de los datos, puesto que la utilización de una formas u otras pueden resultar más útiles para la evaluación de ciertos indicadores perceptuales.
    
    \section{Objetivos}
        Con todo esto, el objetivo de este proyecto es el de comprobar las diferencias perceptuales que se obtienen entre distintas respuestas impulsivas tomadas en distintas posiciones de una sala.
        
        Para lograr este objetivo, se identifican una serie de objetivos secundarios. Entre otros se pueden citar:
        
        \begin{itemize}
            \item Escoger una sala adecuada donde realizar las medidas que tenga unas características determinadas como tamaño, propiedades acústicas invariantes y no dependientes del exterior, etc.
            \item Estudio del número de respuestas al impulso necesarias para el proyecto, tanto en distintas posiciones como en posiciones repetidas para validar la invarianza de las medidas y del recinto. Estimación del tiempo y esfuerzo necesarios para realizar las medidas y planificación de la tareas. Estrategias para optimizar el proceso.
            \item Obtención de un número relevante de respuestas al impulso en dicha sala de acuerdo con las estimaciones y estrategias establecidas anteriormente.
            \item Postprocesado de las señales obtenidas para que puedan formar una base de datos sobres las que realizar test psicoacústicos.
            \item Diseño de los test más adecuados para alcanzar el objetivo principal. Esto incluye la elección de los métodos estadísticos y todas las características propias de cada tipo de test.
      
        \end{itemize}
        
    \section{Organización del documento}
        Este documento se ha organizado de la siguiente forma:
        
        En primer lugar, en el capítulo de ``Antecedentes'' se realiza un análisis del estado del arte de los test psicoacústicos y así clasificarlos en función de diferentes elementos como el número de participantes, la experiencia, duración de las señales o el tipo de análisis utilizado, entre otros. También se realiza una breve introducción a algunos conceptos teóricos clave para la comprensión de las metodologías utilizadas en el análisis estadístico de los datos del proyecto.
        
        A continuación, se encuentra el capítulo de ``Toma de datos''. Aquí se explica, de forma detallada, todo el proceso seguido para la adquisición de las respuestas al impulso. Se parte de una descripción del espacio y se continúa con la metodología seguida para la selección de las posiciones de recepción y emisión. También se comenta el proceso seguido para el procesado de las respuestas para su correcta reproducción en los test. Por último, se da información sobre la metodología seguida para el cálculo de las distancias que será útil para el análisis de los resultados.
        
        Después, se encuentra ``Desarrollo de la aplicación del test subjetivo''. En este apartado se explican las tecnologías y algoritmos utilizados para la creación de la aplicación que recoge las respuestas de los participantes en los diferentes test, así como las modificaciones que ha ido sufriendo a lo largo del proyecto.
        
        El siguiente capítulo es el de ``Realización de los test subjetivos de audio''. En él se comentan las características de los test realizados, tanto del test previo, como de los test finales. De la misma forma, se hace un recorrido por el procidimiento seguido para sus realizaciones.
        
        Los siguientes apartados se centran en mostrar los resultados de los experimentos y el análisis estadístico, respectivamente. En él se incluyen las tablas relativas a cada uno de los test y a los diferentes tipos de análisis y agrupaciones de los datos.
        
        Para finalizar, se incluye un capítulo de ``Conclusiones'' donde se analiza el cumplimiento de los objetivos marcados al inicio del documento. También se ofrecen reflexiones para estudios posteriores y posibles nuevas líneas de investigación.
        
        Al término del documento, se han añadido varios anexos en los que se aportan un análisis de los costes del proyecto y algunas tablas de interés demasiado amplias para incluirlas en el cuerpo de la memoria, por citar algunos ejemplos.
    
\end{document}