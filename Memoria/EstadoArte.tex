\documentclass[11pt,a4paper]{book}
\usepackage[utf8]{inputenc}
\usepackage[spanish]{babel}
\usepackage{amsmath}
\usepackage{amsfonts}
\usepackage{amssymb}
\usepackage[left=2cm,right=2cm,top=2cm,bottom=2cm]{geometry}
\author{Víctor de Tejada Molera}
\begin{document}
\chapter{Estado del arte}
    Para el análisis del estado del arte, se recurre a bases de datos científicas como el sistema \textit{Ingenio UPM} y buscadores de artículos científicos como \textit{Google Scholar}. Con ellos, se han introducido términos de búsqueda como ``\textit{Psychoacoustics}'', ``\textit{Subjective test acoustics}'' o ``\textit{Auditorium subjective test}''. Los resultados arrojados por dichas plataformas, se revisan y se lee el resumen o \textit{abstract} de los que se consideran que pueden ser útiles para el proyecto. Si el resumen confirma que la temática concuerda con la del estudio, se procede a la lectura del resto del artículo y se extraen los elementos que se consideran más importantes. Estos son principalmente: el tipo de análisis estadístico que se ha utilizado, el número de personas que han realizado el test, el tipo de test (tipo de preguntas, formato, duración de los audios utilizados, etc.), entre otros.
    
    Por otro lado, se ha realizado paralelamente búsquedas de libros de cosnulta especializados en temas como de detección de señales, análisis estadístico de datos y modelos Thurstonianos. Para ello, se realizan nuevas consultas a los medios ya presentados y personas con experiencia contrastasda como el Dr. Daniel de la Prida, que es uno de los autores de varios de los artículos consultados previamente.\newline
    
    Utilizando estos medios, se han consultado alrededor de 30 documentos entre libros de referencia, normativas internacionales, artículos de congreso, trabajos de fin de grado, etc. En ellos, como se menciona en REFERENCIAVICTOR, las características de los test subjetivos aplicados son muy diferentes entre sí e incluyen una gran variedad en cuanto al número de participantes o procedimientos se refiere.
    
    \section{Características de la documentación consultada}
    
    Al analizar los distintos documentos, se ha observado una inconsistencia notable en el número de personas que realizan los estudios. Por un lado, se tienen estudios que utilizan un número muy pequeño de personas, menos de 10, como es el caso de REFERENCIASSENSIBILIDAD. Por otro lado, algunos casos presentan números mucho mayores llegando a utilizar más de 100 personas, como ocurre en REFERENCIAMETHODOLOGY. 
    
    Por otro lado, se ha observado una gran variedad de formatos para hacer los test. Los más habituales se presentan a continuación.
    
    \begin{itemize}
        \item \textbf{Test de diferencias}REFERENCIASDIFERENCIAS: En este tipo de test al oyente se le presentan 2 o más audios y tiene que determinar si son iguales o no. Tiene la ventaja de que son sencillos de implementar y de analizar.
        \item \textbf{Test Duo-Trio}REFERENCIASDUOTRIO: Para este formato, se presentan tres audios. Uno de ellos es igual a  uno de los otros dos que se denomina ``Referencia''. El objetivo del oyente es determinar cuál de los audios que se le presenta es dicha igual a dicha referencia.
        \item \textbf{Test de escalas numéricas}REFERENCIASESCALAS: El objetivo de este test es que el oyente cuantifique alguna determinada característica de las señales de audio mediante una escala numérica (generalmente mediante un \textit{fader} o similar). Algunas variantes permiten que se realice la cuantificación de varios elementos en paralelo o que se pida que se ordenen varios audios simultáneamente en función de la característica que se pretenda evaluar (Test de MUSHRA).
        \item \textbf{ABX}: En este tipo de test se presentan tres señales. El oyente tiene que determinar cuál de las dos primeras que se le presentan es igual a la que es presentada en último lugar. El experimento se repite varias veces modificando la última señal para que no sea siempre el mismo audio. 
    \end{itemize}
    Existen otros tipos de test como los triangulares o los A/NotA que se reflejan y explican en REFERENCIADELAPRIDA21. No obstante, no se han encontrado test en los que se aplique dicha metodología, a parte de ese artículo, por lo que no se han considerado para las siguientes partes del proyecto.
    
    Cada uno de de los tipos de test mencionandos arriba tienen sus propias particularidades. Sus ventajas y sus inconvenientes tanto para su realización como para el análisis de sus resultados. Esto último se comenta de forma más extensiva en el apartado de ``Análisis estadístico''. 
    
    A nivel de interacción por parte de la persona participante, existe división de opiniones sobre la posibilidad del usuario de repetir la reproducción de los estímulos o incluso la capacidad de escoger cuándo se reproducen diréctamente. No obstante, normas como REFERENCIANORMAS recomiendan que los participantes tengan, siempre que sea posible, la posibilidad de intereactuar con los estímulos de forma directa y repetir su reproducción si así lo desean.
    
    En cuanto a la duración de los estímulos, se han observado que los valores suelen rondar los 8-10 segundos, obteniéndose casos con un máximo de 15 segundos como en REFERENCIAWITEW. Esto concuerda con las recomendaciones de REFERENCIANORMAS al respecto donde se indica que la duración de dichos estímulos debe ser lo suficientemente reducida para evitar que los participantes se acostumbren a los estímulos. Unido a esto, la duración total de la sesión, a pesar de ser el apartado del que menos información se hace pública, se recomienda que la duración no exceda de los 30 minutos, siendo necesario la inclusión de descansos de la misma duración, si se tiene que alargar, como es el caso de REFERENCIAKRITLY donde se tiene una duración de casi 90 minutos.\newline
    
    A la vista de todo lo expuesto, se corrobora la visión de REFERENCIADETEJADA y REFERENCIADELAPRIDA de la gran vriedad de sistemas que se siguen en la actualidad para realizar los test perceptuales. Por ello, para nuestro test es necesario identificar sus particularidades para poder determinar las características que mejor se ajusten a nuestro proyecto.
    
\end{document}