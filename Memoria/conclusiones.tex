\documentclass[11pt,a4paper,twoside]{book}
\usepackage[utf8]{inputenc}
\usepackage[spanish]{babel}
\usepackage{amsmath}
\usepackage{graphicx}
\usepackage{amsfonts}
\usepackage{amssymb}
\usepackage[left=2cm,right=2cm,top=2cm,bottom=2cm]{geometry}
\usepackage{float}
\usepackage{pdfpages}
\usepackage{longtable}
\author{Víctor de Tejada Molera}
\begin{document}
\chapter{Conclusiones}
    En este proyecto se ha estudiado la percepción de las respuestas al impulso de una sala en diferentes posiciones de escucha. Para ello, se ha realizado un análisis del estado del arte de los test subjetivos relacionados con la percepción auditva. De esta forma, se han fijado las características y requerimientos necesarios para la correcta toma de datos y la realización de nuestro test perceptual. De forma paralela, se ha diseñado una aplicación escrita en Python para la interacción por parte de los participantes y el análisis de los datos (esta última parte junto al desarrollo de un script en lenguaje de programación R). Por último, se ha realizado el análisis de los resultados categorizándolos en función de dos tipos de distribuciones.
    
    \section{Satisfacción de los objetivos secundarios}
    
        Con todo este trabajo, se ha tratado de cumplir una serie de objetivos secundarios:
        
        \subsection*{Selección de una sala adecuada donde realizar las medidas}    
            El primero de ellos consistía en la selección de una sala adecuada donde realizar las medidas. Esta sala debía tener un tamaño lo suficientemente grande para que pudieran existir diferencias perceptuales importantes. Así mismo, la sala debía disponer de unas propiedades acústicas invariantes y que no dependieran del exterior. Esto se ha conseguido utilizando como espacio para las medidas, el auditorio de la ETSIST de la UPM. Este espacio cuenta con un aforo superior a las 300 butacas y su acústica está pensada para música coral con un aislamiento acústico suficiente para minimizar los efectos del ruido exterior. Con esto, los mayores desafíos se han visto reflejados en el control de las fuentes de ruido interno como zumbidos eléctricos o el efecto de los sistemas de climatización. Todas estas medidas han limitado el posible efecto de variables externas en las escuchas de los participantes.
        
        \subsection*{Estudio del número de respuestas necesarias}
            Otro de los objetivos era el de estudiar el número de respuestas al impulso necesarias para el proyecto tanto para posiciones distintas como en posiciones repetidas.A la vista de los resultados y el análisis realizado, se puede concluir que la percepción auditiva de las respuestas al impulso de un determinado espacio dependen, en gran medida, de la distancia a la fuente sonora. De esta forma, se puede ver como las diferencias son más perceptibles a medida que las respuestas se alejan de la fuente hasta que, en las posiciones más alejadas, dichas diferencias dejan de percibirse con una pendiente bastante abrupta. También se puede concluir que la distancia absoluta entre las posiciones de escucha tiene, a su vez, un rol muy importante a la hora de encontrar diferencias perceptuales. En este caso, se ha observado que las diferencias se encuentran más fácilmente a medida que aumenta dicha distancia. A priori, parece que esa percepción de diferencias se estabiliza llegado un determinado punto, pero no se tienen datos suficientes para afirmarlo. Debido a esto, se puede afirmar que la mejor forma de realizar los tomas de datos no es uniforme, sino que en las posiciones más cercanas y alejadas, la densidad de los datos puede ser menor, ya que la capacidad de los participantes para distinguir las señales es bastante menor que en el centro. En esta parte central, la habilidad para distinguir los estímulos es mayor, por lo que es necesario tener un mayor volumen de datos. 
    
            Las mayores dificultades que se han encontrado para la consecución de este objetivo es la imposibilidad de establecer unos criterios específicos para el número de respuestas. Aquí entran en juego variables como las dimensiones de la sala, la reverberación o la disposición del patio de butacas. Por este motivo, se antoja necesaria la realización de un análisis específico según el tipo de espacio sobre el que vayan a tomarse las medidas. También hay que tener en cuenta que la toma de datos es una de las partes de los proyectos más propensa a errores, lo que puede llevar a repeticiones y demoras en la realización del proyecto.
        
        \subsection*{Postprocesado de las señales para test perceptuales}
            Para el objetivo del postprocesado de las señales para su utilización en test psicoacústicos, se ha visto que no hay un sistema único para llevarse a cabo. El procedimiento varía en función del tipo de señales que se vayan a utilizar. Por ejemplo, para el caso de respuestas impulsivas, la utilización del software ``Dirac'' se ha mostrado como la opción más cómoda ya que, además de permitir la grabación de las señales, puede confirgurarse para realizar el procesado de forma automática. Si se optara por otro tipo de señales, como las respuestas convolucionadas con audios anecoicos, hay que acudir a otras herramientas que, preferiblemente, permitan la automatización del proceso para grupos de señales.
        
        \subsection*{Diseño de los test más adecuados para lograr el objetivo principal}
            El último subojetivo era el de diseñar los test más adecuados para lograr el objetivo principal. Esto se ha conseguido, en gran medida, gracias a la utilización de un test sencillo de utilizar para los participantes, lo que nos ha permitido contar con personas sin experiencia previa. El \textit{handicap} de la experiencia se ha suplido, en parte, con la realización de un entrenamiento previo por parte de todos los participantes, independientemente de sus conocimientos previos. Gracias a esto, se ha formado a las personas para que supieran responder a las cuestiones que se les preguntaba y obtener así el mayor volumen de datos posible. 
    
            Atendiendo a los procedimientos de análisis, se ha visto la utilidad de los modelos thurstonianos frente a otros tipos de análisis más tradicionales como la norma UNE-EN ISO 10399 \cite{ISO10399}. Estos modelos permiten comparar el nivel de diferencia entre diferentes grupos de datos (incluso siendo completamente independientes). Esto permite realizar análisis estadísticamente más complejos e interesantes que las posibilidades que brindan otros sistemas. 
    
            No obstante, al final, la utilidad del tipo de análisis depende de lo que se pretenda estudiar. De esta forma, si se pretende estudiar sólamente si unas respuestas son estadísticamente representativas, la norma UNE-EN ISO 10399 es un sistema rápido y sencillo para dar respuesta a esta cuestión. Por otro lado, si se pretende estudiar la variación de las diferencias percibidas a lo largo de un espacio, los modelos thurstonianos cuentan con herramientas que pueden ser más útiles. 
    
            Dentro de este objetivo, los mayores desafios se han encontrado a la hora de escoger el tipo de test adecuado. Existen diferentes opciones que permiten obtener un tipo de resultados u otros. Esto puede repercutir en un incremento de la dificultad del test o añadir un nivel adicional de detalle en los resultados. Es importante realizar un análisis previo sobre qué se pretende estudiar para seleccionar el modelo que mejor satisface sus necesidades. Esto incluye también al tipo de análisis de datos que, como se ha expresado antes, tiene una gran repercusión, no sólo en la complejidad y el tiempo de cálculo, sino en la calidad de los resultados obtenidos.
    
    \section{Satisfacción del objetivo principal}
        El objetivo principal del proyecto era el de comprobar las diferencias perceptuales entre distintas respuestas al impulso tomadas en distintas posiciones de una sala. A raiz de lo comentado en el apartado anterior, se puede concluir que las diferencias percibidas dependen en gran medida de la distancia entre ambas posiciones. De esta forma, las diferencias se detectan más fácilmente a medida que la distancia aumenta.
        
        Por otro lado, el efecto de la distancia también es relevante si se analiza la distancia de las posiciones de escucha respecto de la fuente sonora. Aquí se observa que la sensibilidad de los participantes para detectar las diferencias aumenta a medida que se alejan de la fuente hasta que, llegado un punto en la parte más lejana de la sala, la sensibilidad disminuye de forma rápida.
        
        Con todo esto, se puede decir que el objetivo marcado en el proyecto se ha cumplido de forma satisfactoria. Del mismo modo, se ha cumplido la hipótesis inicial de que la percepción de las respuestas al impulso varían en función de la distancia.
    
    \section{Mejoras y nuevas líneas de investigación}
    
        Si se empezara ahora el proyecto, habría algunos elementos que se hubieran realizado de forma distinta.
        
        Por un lado, se hubiera modificado la distribución de las posiciones en la toma de datos. En vez de seleccionar los datos de una de cada tres butacas con un desplazamiento de una butaca entre filas, se hubiera hecho una distribución dependiente de la distancia a la fuente sonora. De esta forma, se hubieran tomado menos datos en las posiciones más cercanas y el grueso de las medidas se encontrarían en las zonas medias y lejanas. También, de haber disponido del tiempo y material necesario, se hubiera repetido los test y análisis utilizando otro tipo de señales para comprobar si los resultados varían en función de los audios que se aplican.
    
        Otras posibles futuras líneas de investigación incluyen el análisis de otros aspectos perceptuales más concretos como la coloración de las señales, la percepción de reverberación o comprobar de qué forma las variaciones del nivel de percepción son modificadas en función de diferentes tipos de salas (de tamaños diferentes, variando el acondicionamiento acústico, etc.)
        
    \section{Posibles aplicaciones del proyecto}
        Los resultados de este proyecto pueden resultar de utilidad dentro del campo de las mediciones acústicas. De esta manera, si se van a realizar mediciones relacionadas con la percepción auditiva, podrían ahorrarse medidas en las posiciones más cercanas a la fuente sonora y concentrarlas en puntos más sensibles como la zona central.
        
        También existen aplicaciones para la recreación virtual de espacios acústicos. Aquí podría simplificarse la carga de trabajo al poder centrar los esfuerzos en las áreas donde la sensibilidad de los oyentes es mayor.
        
        Para ambos casos, no obstante, sería necesario realizar un análisis previo de las características de los espacios para limitar correctamente las diferentes zonas de percepción auditiva. 
    

\end{document}