\documentclass[11pt,a4paper,twoside]{book}
\usepackage[utf8]{inputenc}
\usepackage[spanish]{babel}
\usepackage{amsmath}
\usepackage{graphicx}
\usepackage{amsfonts}
\usepackage{amssymb}
\usepackage[left=2cm,right=2cm,top=2cm,bottom=2cm]{geometry}
\usepackage{float}
\usepackage{pdfpages}
\usepackage{longtable}
\author{Víctor de Tejada Molera}
\begin{document}
\chapter{Conclusiones}
    A la vista de los resultados y el análisis realizado, se puede concluir que la percepción auditiva de las respuestas al impulso de un determinado espacio dependen, en gran medida, de la distancia a la fuente sonora. De esta forma, se puede ver como las diferencias son más perceptibles a medida que las respuestas se alejan de la fuente hasta que, en las posiciones más alejadas, dichas diferencias dejan de percibirse con una pendiente bastante abrupta.
        
    También se puede concluir que la distancia absoluta entre las posiciones de escucha tiene, a su vez, un rol muy importante a la hora de encontrar diferencias perceptuales. En este caso, se ha observado que las diferencias se encuentran más fácilmente a medida que aumenta dicha distancia. A priori, parece que esa percepción de diferencias se estabiliza llegado un determinado punto, pero no se tienen datos suficientes para afirmarlo. Esto podría ser un tema interesante para un nuevo proyecto.
    
    Otras futuras líneas de investigación incluyen el análisis de otros aspectos perceptuales más concretos como la coloración de las señales, la percepción de reverberación o comprobar de qué forma las variaciones del nivel de percepción son modificadas en función de diferentes tipos de salas (de tamaños diferentes, con mayor o menor absorción, etc.). También puede resultar de interés realizar el mismo análisis, pero utilizando otro tipo de señales, como pueden ser el resultado de la convolución de las respuestas al impulso con una señal anecoica.
        
    Atendiendo a los procedimientos de análisis, se ha visto la utilidad de los modelos thurstonianos frente a otros tipos de análisis más tradicionales como la norma UNE-EN ISO 10399 \cite{ISO10399}. Estos modelos permiten comparar el nivel de diferencia entre diferentes grupos de datos (incluso siendo completamente independientes). Esto permite realizar análisis estadísticamente más complejos e interesantes que las posibilidades que brindan otros sistemas. 
    
    No obstante, al final, la utilidad del tipo de análisis depende de lo que se pretenda estudiar. De esta forma, si se pretende estudiar sólamente si unas respuestas son estadísticamente representativas, la norma UNE-EN ISO 10399 es un sistema rápido y sencillo para dar respuesta a esta cuestión. Por otro lado, si se pretende estudiar la variación de las diferencias percibidas a lo largo de un espacio, los modelos thurstonianos cuentan con herramientas que pueden ser más útiles. 
    
    

\end{document}