\documentclass[11pt,a4paper,twoside]{book}
\usepackage[utf8]{inputenc}
\usepackage[spanish]{babel}
\usepackage{amsmath}
\usepackage{graphicx}
\usepackage{amsfonts}
\usepackage{amssymb}
\usepackage[left=2cm,right=2cm,top=2cm,bottom=2cm]{geometry}
\author{Víctor de Tejada Molera}
\begin{document}
\chapter{Análisis estadístico}    
    \section{Conceptos teóricos}    
        \subsection*{UNE-EN ISO 10399}
            La norma UNE-EN ISO 10399: ``Análisis sensorial. Metodología. Ensayo Duo-Trio.'' REFERENCIANORMA es un documento que permite analizar la probabilidad de que eventos perceptuales sean percibidos como iguales o diferentes según el número de respuestas definidas como correctas y/o erróneas al realizar experimentos basados en test duo-trio.
        
            Para ello, en primer lugar se definen diferentes términos que aparecen de forma constante a lo largo de la norma. Para nuestro caso particular, los términos más relevantes son:
        
            \begin{itemize}
                \item alpha-risk o $\alpha$-risk: es la probabilidad para poder afirmar que existe una diferencia perceptual cuando en realidad no existe.
                \item beta-risk o $\beta$-risk: es la probabilidad para poder afirmar que no existe una diferencia perceptual cuando en realidad sí existe.
                \item diferencia perceptual: situación en la que dos o maś muestras pueden ser distinguidas por sus propiedades sensitivas (a través del oído, tacto, gusto, vista, etc.)
                \item similaridad perceptual: situación en la que las diferencias entre muestras son tan pequeñas que no pueden distinguirse entre sí de forma sensitiva.
            \end{itemize}
            Para el cálculo de las probabilidades $\alpha$-risk y $\beta$-risk, la norma proporciona dos sendas tablas que se encuentran en el Anexo A de dicha norma (las tablas A.1 y A.2 respectivamente).
        \subsection*{Modelos Thursthonianos}
    
\end{document}