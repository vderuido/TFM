\documentclass[11pt,a4paper,twoside]{book}
\usepackage[utf8]{inputenc}
\usepackage[spanish]{babel}
\usepackage{amsmath}
\usepackage{graphicx}
\usepackage{amsfonts}
\usepackage{amssymb}
\usepackage[left=2cm,right=2cm,top=2cm,bottom=2cm]{geometry}
\author{Víctor de Tejada Molera}
\begin{document}
\chapter{Teoría estadística}
    En este apartado se pretende explicar con ligera profundidad algunos conceptos estadísticos que son utilizados a lo largo de este proyecto. En ningún caso pretende ser una explicación densa con demostraciones matemáticas y formulaciones complejas; más bien un intento de facilitar el entendimiento de los conceptos sobre teoría estadística que se aplican y su motivo en lugar de alguna de las otras técnicas que suelen aplicarse. Si se desea expandir la información sobre alguno de estos campos, se recomienda la lectura de los diferentes documentos que se referencian a lo largo de este capítulo y que pueden encontrarse en la bibliografía al final del documento.
    
    \section{Conceptos básicos}
        
    
    \section{UNE-EN ISO 10399}
    
    \section{Modelos Thursthonianos}
    
\end{document}