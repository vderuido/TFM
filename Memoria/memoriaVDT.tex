\documentclass[11pt,a4paper]{book}
\usepackage[utf8]{inputenc}
\usepackage[spanish]{babel}
\renewcommand{\spanishlisttablename}{Índice de tablas}
\renewcommand\spanishtablename{Tabla}
\renewcommand{\spanishappendixname}{Anexo}
\usepackage{amsmath}
\usepackage{amsfonts}
\usepackage{graphicx}
\usepackage{amssymb}
\usepackage[left=2cm,right=2cm,top=2cm,bottom=2cm]{geometry}
\usepackage{cite}
\usepackage{babelbib}
\usepackage{float}
\usepackage{pdfpages}
\usepackage{longtable}
\usepackage{eurosym}
\author{Víctor de Tejada Molera}

\begin{document}

\tableofcontents
\listoffigures
\listoftables
\chapter{Introducción}
    La respuesta al impulso de una sala se denomina como el comportamiento acústico que se obtiene en una determinada posición de recepción, cuando ese espacio es excitado con una señal del tipo ``Delta'', desde una posición de emisión. Este tipo de respuestas tienen una gran importancia ya que aportan información sobre la reverberación del espacio, la inteligibilidad, la capacidad para realizar localizaciones acústicas dentro del recinto, etc. Por este motivo, se pueden considerar como una de las señales más útiles para la caracterización acústica de un determinado espacio. 
    
    A pesar de todas estas bondades, las respuestas al impulso tienen una gran limitación: sólo aportan información sobre cómo se comporta el espacio para el par de puntos que se fijan como punto de emisión y de recepción. Por tanto, si se quiere caracterizar de forma adecuada un recinto, se deben tomar un elevado número de respuestas de forma que se consiga obtener un número de puntos de recepción suficientes.
    
    ¿Pero cuántas posiciones son realmente necesarias? Si se analiza un espacio como un aula, un auditorio o una sala de conferencias, las posiciones de emisión lógicas vienen marcadas por las posiciones dónde estarían colocados los ponentes o los músicos. Con las posiciones de recepción la situación se complica, puesto que es habitual que los espacios cuenten con cientos o miles de asientos para el público. Para solucionar esta pregunta, conviene estudiar cómo varía la percepción de las personas y comprobar si existen posiciones para los que los participantes son incapaces de distinguir si están en una posición u otra.
    
    Para resolver estas cuestiones entra en juego la \textit{Psicoacústica}. La psicoacústica es el campo de la ciencia que estudia cómo las personas perciben el sonido, y la forma más habitual de trabajo es mediante la realización de test subjetivos de audio. Estos test se componen de un determinado número de personas que se ven expuestas a estímulos (acústicos, pero a veces también visuales), y deben responder a una serie de cuestiones sobre su percepción sobre dichos estímulos.
    
    La psicoacústica, como se puede suponer, depende en gran medida de procesos estadísticos para la elaboración de sus conclusiones. Por este motivo, es importante disponer de un gran número de participantes para que los resultados tengan relevancia estadística. Si el número de participantes fuera pequeño, las desviaciones obtenidas tendrían valores muy elevados y los resultados podrían estar fuertemente influidos por el azar, haciendo muy complicado volver a obtener valores similares y que otros estudios fueran capaces de reproducir el mismo experimento.
    
    Para conseguir unos resultados que sean válidos, se debe, además, analizar qué tipo de participantes se busca que respondan al test. La utilización de personas con experiencia o conocimientos relacionados con la materia de estudio, puede hacer que el personal necesario se reduzca notablemente, o que el número de preguntas pueda aumentarse si se supone que responden a las preguntas más rápidamente. No obstante, pueden darse casos en los que resulte más útil la utilización de participantes sin experiencia para recoger un comportamiento más general.
    
    También resulta importante determinar un buen sistema para el análisis de los datos, puesto que la utilización de una formas u otras pueden resultar más útiles para la evaluación de ciertos indicadores perceptuales.
    
    \section{Objetivos}
        Con todo esto, el objetivo de este proyecto es el de comprobar las diferencias perceptuales que se obtienen entre distintas respuestas impulsivas tomadas en distintas posiciones de una sala.
        
        Para lograr este objetivo, se identifican una serie de objetivos secundarios. Entre otros se pueden citar:
        
        \begin{itemize}
            \item Escoger una sala adecuada donde realizar las medidas que tenga unas características determinadas como tamaño, propiedades acústicas invariantes y no dependientes del exterior, etc.
            \item Estudio del número de respuestas al impulso necesarias para el proyecto, tanto en distintas posiciones como en posiciones repetidas para validar la invarianza de las medidas y del recinto. Estimación del tiempo y esfuerzo necesarios para realizar las medidas y planificación de la tareas. Estrategias para optimizar el proceso.
            \item Obtención de un número relevante de respuestas al impulso en dicha sala de acuerdo con las estimaciones y estrategias establecidas anteriormente.
            \item Postprocesado de las señales obtenidas para que puedan formar una base de datos sobres las que realizar test psicoacústicos.
            \item Diseño de los test más adecuados para alcanzar el objetivo principal. Esto incluye la elección de los métodos estadísticos y todas las características propias de cada tipo de test.
      
        \end{itemize}
        
    \section{Organización del documento}
        Este documento se ha organizado de la siguiente forma:
        
        En primer lugar, en el capítulo de ``Antecedentes'' se realiza un análisis del estado del arte de los test psicoacústicos y así clasificarlos en función de diferentes elementos como el número de participantes, la experiencia, duración de las señales o el tipo de análisis utilizado, entre otros. También se realiza una breve introducción a algunos conceptos teóricos clave para la comprensión de las metodologías utilizadas en el análisis estadístico de los datos del proyecto.
        
        A continuación, se encuentra el capítulo de ``Toma de datos''. Aquí se explica, de forma detallada, todo el proceso seguido para la adquisición de las respuestas al impulso. Se parte de una descripción del espacio y se continúa con la metodología seguida para la selección de las posiciones de recepción y emisión. También se comenta el proceso seguido para el procesado de las respuestas para su correcta reproducción en los test. Por último, se da información sobre la metodología seguida para el cálculo de las distancias que será útil para el análisis de los resultados.
        
        Después, se encuentra ``Desarrollo de la aplicación del test subjetivo''. En este apartado se explican las tecnologías y algoritmos utilizados para la creación de la aplicación que recoge las respuestas de los participantes en los diferentes test, así como las modificaciones que ha ido sufriendo a lo largo del proyecto.
        
        El siguiente capítulo es el de ``Realización de los test subjetivos de audio''. En él se comentan las características de los test realizados, tanto del test previo, como de los test finales. De la misma forma, se hace un recorrido por el procidimiento seguido para sus realizaciones.
        
        Los siguientes apartados se centran en mostrar los resultados de los experimentos y el análisis estadístico, respectivamente. En él se incluyen las tablas relativas a cada uno de los test y a los diferentes tipos de análisis y agrupaciones de los datos.
        
        Para finalizar, se incluye un capítulo de ``Conclusiones'' donde se analiza el cumplimiento de los objetivos marcados al inicio del documento. También se ofrecen reflexiones para estudios posteriores y posibles nuevas líneas de investigación.
        
        Al término del documento, se han añadido varios anexos en los que se aportan un análisis de los costes del proyecto y algunas tablas de interés demasiado amplias para incluirlas en el cuerpo de la memoria, por citar algunos ejemplos.

\chapter{Antecedentes}
    En este capítulo se pretende realizar un análisis del estado del arte referente a los experimentos psicoacústicos; concretamente aquellos que se centran en la evaluación de diferencias perceptuales en entornos cerrados como auditorios, cines, salas de conferencias, etc.
    
    Para la búsqueda de esta información, se recurre a bases de datos científicas como el sistema \textit{Ingenio UPM} y buscadores de artículos científicos como \textit{Google Scholar}. Con ellos, se han introducido términos de búsqueda como ``\textit{Psychoacoustics}'', ``\textit{Subjective test acoustics}'' o ``\textit{Auditorium subjective test}''. Los resultados arrojados por dichas plataformas, se revisan y se lee el resumen o \textit{abstract} de los que se consideran que pueden ser útiles para el proyecto. Si el resumen confirma que la temática concuerda con la del estudio, se procede a la lectura del resto del artículo y se extraen los elementos más importantes. Estos son principalmente: el tipo de análisis estadístico que se ha utilizado, el número de personas que han realizado el test, el tipo de test (tipo de preguntas, formato, duración de los audios utilizados, etc.), entre otros.
    
    Por otro lado, se ha realizado, paralelamente, una búsqueda de fuentes bibliográficas especializados en temas como de detección de señales, análisis estadístico de datos y, más específicamente, sobre modelos Thurstonianos. Para ello, se realizan nuevas consultas a los medios ya presentados a personas con experiencia contrastasda en la realización y el estudio de experimentos psicoacústicos.\newline
    
    Utilizando esta metodología, se han consultado alrededor de 50 documentos entre libros de referencia, normativas internacionales, artículos en revistas y congresos, trabajos de fin de grado, etc. De todo este material, han acabado siendo de utilidad 30 fuentes bibliográficas aproximadamente.
    
    \section{Análisis de la documentación consultada}
    
    Al analizar los distintos documentos, se ha observado una gran inconsistencia en las características de los diferentes estudios que se han analizado. Este problema ya se comenta en algunas fuentes como \cite{Tejada2020}. En este apartado se analizan las características de los proyectos y normativas según varios de sus aspectos más representativos. 
    
        \subsection{Aspectos subjetivos a analizar}
		    Los aspectos subjetivos que se pueden analizar mediante test perceptuales son muy variados. Por un lado algunos experimentos como en \cite{1995GASoulodre, 2016SKlockgether,  2016BPostma, 2002PZahorik} se centran en estudiar las características relacionadas con la reverberación de las salas (persistencia del sonido, diferencias entre sonidos anecoicos y reverberantes, etc.). Otros estudios se centran en el estudio de la inteligibilidad de diferentes señales acústicas \cite{ 1999JBradley, 2010FMartellotta,  2010MVigeant}, mientras otras se centran en temas de localización de sonora \cite{1997SCarlile, 2019MShiell, 2019MYamada, 2019DMorikawa}. También se encuentran aquellos experimentos en los que se pretende estudiar características como las diferencias entre señales, la sonoridad o diferencias de nivel \cite{2019GPulvirenti, 2016SKlockgether, 2019MNowak, 2011VEmiya, 2019LKritly, 2005IWitew,  2019DJSchlit}. Por último, una de las áreas que está teniendo más importancia en los últimos años se corresponden con el estudio de la molestia acústica \cite{2019VHongisto, 2019JLee, 2019VRajala}.
		    
            Algunos documentos como \cite{Tejada2020} tratan de ordenar estas fuentes en diferentes categorías y facilitar el análisis de grandes cantidades de bibliografía y extraer sus elementos comunes.
            
        \subsection{Tipos de test}
    		Por otro lado, se ha observado una gran variedad de formatos para hacer los test. Los más habituales se presentan a continuación.
    
    		\begin{itemize}
        		\item \textbf{Test de diferencias}: En este tipo de test al oyente se le presentan 2 o más audios y tiene que determinar si son iguales o no. Tiene la ventaja de que son sencillos de implementar y de analizar. En algunos sitios, se le llama ``duo'' o ``binomial'' por lo que es importante revisarlos a conciencia para no confundirlos con los ``Duo-Trio'' o un ``ABX''. Algunos de los artículos en los que han aplicado este tipo de test son \cite{1995GASoulodre, 1999JBradley,2002PZahorik, 2005IWitew, 2010MVigeant, 2019LKritly}. Un ejemplo concreto de este tipo de test y que tendrá gran importancia a lo largo del proyecto es el denominado ``2-AFC'' (\textit{Two-Alternative Forced Choice}). En este test, el participante está ``forzado'' a escoger entre dos opciones, no pudiendo seleccionarse una opción intermedia, como respuesta a una pregunta relacionada con su exposición a un determinado estímulo \cite{PsychophysicsB}.
        		\item \textbf{Test Duo-Trio}: Para este formato, se presentan tres audios. Uno de ellos es igual a  uno de los otros dos que se denomina ``Referencia''. El objetivo del oyente es determinar cuál de los audios que se le presenta es igual a dicha referencia. Algunos de los estudios que han utilizado este tipo de test son \cite{delaPrida2019, delaPrida2021}.
        		\item \textbf{ABX}: En este tipo de test se presentan tres señales. El oyente tiene que determinar cuál de las dos primeras que se le presentan es igual a la que es presentada en último lugar. El experimento se repite varias veces modificando la última señal para que no sea siempre el mismo audio. Algunos ejemplos donde se aplican este tipo de test se encuentran en \cite{delaPrida2021, Braun2004}.
        		\item \textbf{Test de escalas numéricas}: El objetivo de este test es que el oyente cuantifique alguna determinada característica de las señales de audio mediante una escala numérica. Algunas variantes permiten que se realice la cuantificación de varios elementos en paralelo o que se pida que se ordenen varios audios simultáneamente en función de la característica que se pretenda evaluar (Test de MUSHRA \cite{UIT1534}). Este tipo de test son muy útiles y aparecen como recomendación por parte de normativas internacionales \cite{UIT1116, UIT1284, UIT1534}.También aparecen en algunos documentos como \cite{2011VEmiya,2016BPostma, 2019ZShao,2019GPulvirenti,2019VRajala, 2019JLee, 2019DMorikawa}. 
        		
    		\end{itemize}
    		Existen otros tipos de test como los triangulares o los A/NotA \cite{delaPrida2021, Brockhoff2009}. No obstante, no se han encontrado experimentos en los que se apliquen dichas metodologías, a parte de ese artículo, por lo que no se han considerado para las siguientes partes del proyecto.
    
    		Cada uno de los tipos de test mencionandos arriba tienen sus propias particularidades, sus ventajas y sus inconvenientes tanto para su realización como para el análisis de sus resultados que se comentarán posteriormente en el capítulo de ``Realización de los test subjetivos de audio''.
    
	    \subsection{Características de los participantes}
    		Si se analizan las características de las personas que participan en los test perceptuales, se observa una gran variedad tanto en el número como en su experiencia. 
    		
    		Por un lado, se tienen estudios que utilizan un número muy pequeño de personas, diez o menos, como es el caso de \cite{2019MNowak, 2002PZahorik, 2016SKlockgether,2019ZShao}. Por otro lado, algunos casos presentan números mucho mayores llegando a utilizar casi cien personas, como ocurre en \cite{1954JEgan}. También se observa que el número de participantes varía en función del aspecto subjetivo que pretenda analizarse \cite{Tejada2020}; de esta forma, los estudios que estudian la molestia acústica son los que utilizan un mayor número de participantes (en torno a los 30-40 según \cite{Tejada2020}), mientras que el resto utilizan unas 20 personas aproximadamente, que se corresponde con las recomendaciones de las normativas \cite{UIT1116, UIT1284, UIT1534}.

            Otro aspecto importante muy relacionado con el número de participantes consiste en la utilización de participantes con o sin experiencia en este tipo de experimentos psicoacústicos. Normativas como \cite{UIT1116, UIT1284, UIT1534} recomiendan la utilización de participantes con experiencia cuando los aspectos a analizar son bastante técnicos o requieren de personas con oídos especializados en la detección de ciertos eventos (como, por ejemplo, educación musical avanzada). No obstante, aceptan la presencia de participantes sin experiencia si se aumenta el número total de personas que participan en el experimento o si se les aplica algún tipo de entrenamiento previo, incluso se recomienda realizarlo para participantes con experiencia.

            A pesar de estas recomendaciones, la realidad es bastante diversa y se encuentran estudios como \cite{1995GASoulodre,1999JBradley, 2005IWitew, 2010FMartellotta, 2010MVigeant, 1997SCarlile, 2011VEmiya, 2016BPostma, 2019ZShao, 2019VRajala, 2019MShiell, 2019JGroose, Braun2004, delaPrida2019, delaPrida2021} en los que sí se aplican entrenamientos previos, mientras que en otros estudios como \cite{Brockhoff2009,2019LKritly,2019DMorikawa, 2019DJSchlit} no se realizan. También se han encontrado varios estudios en los que no se reflejaba si se habían realizado estos entrenamientos o no \cite{2002PZahorik, 2016SKlockgether, 2019GPulvirenti,2019MNowak, 2019MYamada, 2019JLee, Christensen2009}.
    
	    \subsection{Interacción del participante con los estímulos auditivos}
    		A nivel de interacción por parte de la persona participante, existe división de opiniones sobre la posibilidad del usuario de poder controlar la reproducción de los estímulos \cite{1995GASoulodre, 2010FMartellotta, 2010MVigeant, 2011VEmiya,2019DMorikawa} frente a \cite{1999JBradley, 2002PZahorik, 2005IWitew, 2016SKlockgether, 2019VRajala, 2019VHongisto} que no permiten a los participantes esa opción. No obstante, normas como \cite{UIT1116,UIT1534, UIT1284,EBU3286, UIT1285, UIT1286} recomiendan que los participantes tengan, siempre que sea posible, la capacidad de interactuar con los estímulos de forma directa y repetir su reproducción si así lo desean. En ningún momento se hace mención sobre el tiempo máximo que un usuario debe pasar con cada estímulo.
	
        \subsection{Duración del experimento}
    		En cuanto a la duración de los estímulos, se han observado que los valores suelen rondar de 1 a 5 segundos \cite{2010FMartellotta, 2016SKlockgether, 2011VEmiya, 2019LKritly, 2019GPulvirenti, 2002PZahorik, 2019MShiell, 2019JGroose, 2019MNowak, 2019MYamada, 2019DMorikawa, 2019DJSchlit}. No obstante, también se han encontrado estudios donde la duración se encuentra entre los 8-10 segundos \cite{2010FMartellotta, 2019JLee, 2010MVigeant}, obteniéndose casos con un máximo de 15 segundos como en \cite{1995GASoulodre, 2005IWitew}. Esto concuerda con las recomendaciones de \cite{UIT1116,UIT1534, UIT1284,EBU3286, UIT1285, UIT1286} al respecto donde se indica que la duración de dichos estímulos debe ser lo suficientemente reducida para evitar que los participantes se acostumbren a las señales \cite{GelfandStanley}. Unido a esto, la duración total de la sesión, a pesar de ser el apartado del que menos información se aporta, se recomienda que no exceda de los 30 minutos , siendo necesaria la inclusión de descansos de la misma duración, si se tiene que alargar \cite{UIT1116, UIT1284, UIT1534, ZwickerFactsModels, GelfandStanley, BlauertSpatialHearing}. Esto ocurre en casos como \cite{2019LKritly} donde se tiene una duración de casi 90 minutos. En ningún momento se especifica si el entrenamiento previo computa dentro de la duración del experimento, aunque en nuestra experiencia particular, se suele incluir.
    
	    \subsection{Análisis de los datos}
    		En cuanto al análisis de los datos, históricamente se han utilizado procedimientos como los análisis de la varianza (ANOVA) \cite{2011VEmiya, 2016BPostma, 2019LKritly, 2019GPulvirenti, 2019MShiell, 2019VHongisto}, o normas como \cite{ISO10399}. No obstante, en estudios más recientes como \cite{delaPrida2019, delaPrida2021} aparecen los modelos thurstonianos como una alternativa sencilla de aplicar y con cualidades que pueden resultar útiles para ciertos tipos de estudio. Algunos de estos ejemplos se explican con más profundidad en el apartado ``1.2.2 Modelos Thurstonianos''.\newline
    
    A la vista de todo lo expuesto, queda patente la gran variedad de sistemas que se siguen en la actualidad para realizar los test perceptuales. Por ello, para nuestro test es necesario identificar sus particularidades para poder determinar las características que mejor se ajusten al mismo. Esto se trabaja en profundidad en los apartados ``4.1.1 Características del test previo'' y ``4.1.2 Características de los test finales'' siguiendo la metodología del capítulo ``Toma de datos''.
    
    \section{Conceptos teóricos para el análisis estadístico}    
        \subsection{UNE-EN ISO 10399}
            La norma UNE-EN ISO 10399: ``Análisis sensorial. Metodología. Ensayo Duo-Trio.''\cite{ISO10399} es un documento que permite analizar la probabilidad de que eventos perceptuales sean percibidos como iguales o diferentes según el número de respuestas definidas como correctas o erróneas al realizar experimentos basados en test duo-trio. Esta norma no se centra en señales acústicas, sino que puede utilizarse para experimentos subjetivos relacionados con olores, sabores imágnees, etc.
            
        
            Al comienzo del documento se definen diferentes términos que aparecen de forma constante a lo largo de la norma. Para nuestro caso particular, los términos más relevantes son:
        
            \begin{itemize}
                \item alpha-risk o $\alpha$-risk: es la probabilidad, o riesgo, de afirmar que existe una diferencia perceptual cuando en realidad no existe.
                \item beta-risk o $\beta$-risk: es la probabilidad, o riegos, de afirmar que no existe una diferencia perceptual cuando en realidad sí existe.
                \item diferencia perceptual: situación en la que dos o maś muestras pueden ser distinguidas por sus propiedades sensitivas (a través del oído, tacto, gusto, vista, etc.)
                \item similaridad perceptual: situación en la que las diferencias entre muestras son tan pequeñas que no pueden distinguirse entre sí de forma sensitiva.
            \end{itemize}
            Para el cálculo de las probabilidades $\alpha$-risk y $\beta$-risk, la norma proporciona sendas tablas. En este proyecto sólo se precisa el cálculo de $\alpha$ que se encuentran en el Anexo B  (la tabla B.1). También se proporciona la ecuación \ref{eq:alpha} donde se puede calcular el mínimo de respuestas ``correctas'' para que se obtenga un determinado valor de $\alpha$-risk. Con estas herramientas se puede comprobar si los resultados de un test arrojan valores para $\alpha$ y $\beta$ se encuentran dentro de unos determinados valores (20\%, 10\%, 5\%, 1\% y 0.1\%)
            
            \begin{equation}
                x=(n/2)+z*\sqrt{n/4}
                \label{eq:alpha}
            \end{equation}
            
            Donde:
            \begin{itemize}
                \item x: número de respuestas correctas mínimas necesarias para que se obtenga un determinado $\alpha$-risk.
                \item n: número de respuestas totales.
                \item z: variable que toma un valor en función de $\alpha$-risk:
                \begin{itemize}
                    \item 0.84 para $\alpha=0.20$.
                    \item 1.28 para $\alpha=0.10$.
                    \item 1.64 para $\alpha=0.05$.
                    \item 2.33 para $\alpha=0.01$.
                    \item 3.09 para $\alpha=0.001$.
                    
                \end{itemize}
            \end{itemize}
            
            Un ejemplo de aplicación para conocer el coeficiente $\alpha$-risk sería el siguiente: Suponemos que tenemos un caso de estudio entre dos señales acústicas distintas entre sí.  Se ha realizado un test perceptual y se han obtenido 100 respuestas, 87 de las cuales opinan que las dos señales son diferentes.
            
            En primer lugar, se acude a la tabla del Anexo B para comprobar si están los datos para ese número de respuestas. En este caso, se comprueba que, desgraciadamente, la tabla sólo arroja datos hata las 88 respuestas, por lo que se procede a resolver la ecuación \ref{eq:alpha}. Sustituyendo las variables por los datos que se tienen, se obtienen 5 valores diferentes de x para cada uno de los posibles valores de $\alpha$:
            \begin{itemize}
                \item $x=55$ para $\alpha=0.20$.
                \item $x=57$ para $\alpha=0.10$.
                \item $x=59$ para $\alpha=0.05$.
                \item $x=62$ para $\alpha=0.01$.
                \item $x=66$ para $\alpha=0.001$.
            \end{itemize}
            Es importante remarcar que los resultados siempre se redondean hacia arriba, puesto que es imposible tener fracciones de respuestas.
            
            Una vez obtenidos los valores, el valor de $\alpha$ se corresponde con el del caso más cercano que sea inferior a nuestro número de respuestas diferentes. En nuestro caso, $\alpha<0.001$ ya que 66 es el número más cercano a 87 (y además se cumple que es menor).
            
            Como ya se explicó anteriormente, el coeficiente $\alpha$-risk muestra la probabilidad de que dos eventos sean diferentes cuando, en realidad, son iguales. Por este motivo, cuanto más pequeño es el valor del coeficiente, estadísticamente más representativa es la diferencia perceptual. Por este motivo, en estudios como el nuestro, se busca obtener valores pequeños.
            
            Puede ocurrir que se obtengan situaciones donde el número de respuestas en las que se han indicado que son diferentes sea menor que para el caso de $\alpha=0.20$ (en el ejemplo anterior, si se hubiera obtenido menos de 55 respuestas marcadas como diferentes). Si este fuera el caso, no se podría concluir que estadísticamente la percepción de dichos estímulos es diferente. En todo caso, habría que concluir que las posibles diferencias son debidas, mayoritariamente, al azar. 
            
        \subsection{Modelos Thurstonianos}
            Los \textit{modelos Thurstonianos}\footnote{Nombrados así por Louis Leon Thurstone.}\cite{PsychophysicsB, SignalB, delaPrida2019} son modelos estadísticos en los que se utilizan variables de distribuciones normales y que se utilizan en gran medida en estudios de discriminación sensorial. 
            
            En el caso concreto de la psicoacústica, se utilizan generalmente para obtener un valor ``$d'$'' que da información ordenada y cuantitativa sobre una determinada percepción subjetiva. Además de este valor, se calcula a su vez la desviación estandar ``$\sigma$''. Con estos dos valores, se puede aproximar las respuestas de un test subjetivo como una sucesión de distribuciones gaussianas en las que en función del valor ``$d'$''y ``$\sigma$'' están más o menos superpuestas. Esta superposición da información sobre la probabilidad de que ambos estímulos puedan ser distinguibles o no entre sí y cuánto. Esto se observa más fácilmente en la figura \ref{fig:modelost} obtenida en \cite{PsychophysicsB}.
            
            \begin{figure}[H]
                \includegraphics[scale=0.7]{../imagenes/modelosthurst.png}
			    \centering
			    \caption{Ejemplo de cálculo de $d'$ utilizando modelos thurstonianos. Fte: \cite{PsychophysicsB} }
			    \label{fig:modelost}
            \end{figure}
            
            Utilizando la figura \ref{fig:modelost} como ejemplo, se puede suponer que S1 y S2 son las distribuciones obtenidas de las respuestas de un test en las que S1 son las respuestas en las que dos estímulos han sido identificados como iguales, mientras que S2 se corresponde con las respuestas cuando los dos estímulos se identifican como diferentes. Cuanto más se solapen ambas distribuciones, más difícil será para los participantes distinguirlas. Gráficamente resulta fácil comprobar que las variables que determinan este solapamiento es principalemente $d'$, sin dejar de lado la importancia que tiene la desviación estándar $\sigma$.
            
            Además de lo ya expuesto, este tipo de análisis es especialmente interesante porque nos permite ordenar los valores obtenidos para $d'$ de forma que las diferencias entre ellos nos da información cuantitativa sobre cómo de diferentes o de similares son cada una de las distribuciones. Con la facilidad añadida que tiene este sistema para representar gráficamente mediante las técnicas habituales como diagramas de barras, entre otros. También tiene la ventaja de que el valor del coeficiente $d'$ es independiente del tipo de test que se realice, incluso es posible compararlos entre sí \cite{delaPrida2021}.
            

\chapter{Toma de datos}
	Para la realización de este proyecto, es necesaria la obtención de un gran número de respuestas al impulso en diferentes posiciones de una sala. En este capítulo, se repasan y analizan los criterios seguidos para la toma de los datos que posteriormente son procesados y utilizados en el test subjetivo de audio. Del mismo modo, se comenta el espacio escogido para la toma de los datos y el equipo utilizado.

	\section{Localización}

		\subsection{Descripción del espacio}

 			En nuestro caso particular, se ha optado por el auditorio de la ETSIST ubicada en el Campus Sur de la UPM. Este salón de actos está diseñado como sala de conferencias y como espacio para actuaciones de música coral. 
 
 			El espacio tiene unas dimensiones de 25 metros de largo, por 13.64 metros de ancho. El auditorio tiene una altura aproximada de unos 5 metros. En la figura \ref{fig:auditorio} puede observarse la vista en planta de la sala junto a sus diferentes zonas.
 			
 			\begin{figure}
				\includegraphics[scale=0.6]{../imagenes/auditorioColor.png}
				\centering
				\caption{Vista en planta del auditorio de la ETSIST. La zona verde se corresponde con el escenario. La zona roja con el patio de butacas. La zona amarilla con los pasillos y zonas de paso. La zona azul son las bambalinas laterales. Las lineas discontínuas marcan la superficie de la concha acústica y los paneles de madera del techo.}
				\label{fig:auditorio}
			\end{figure}
 
 			El escenario se encuentra ligéramente levantado sobre el suelo del salón de actos mediante una estructura de madera. A los lados del escenario se encuentran unos paneles giratorios, también de madera, que dan acceso a dos bambalinas laterales. La parte trasera del escenario cuenta con un gran panel de madera situado en el centro de la pared. El techo del escenario cuenta con una concha acústica de madera y ocupa practicamente toda su superficie. El resto del techo se compone de paneles de madera ligeramente superpuestos y con ligera inclinación para evitar superficies paralelas.
 
 			Las paredes del auditorio se encuentran recubiertas en su mayoría por paneles de madera, excepto por una ligera zona acristalada hacia la zona superior. Las puertas situadas en los laterales son metálicas revestidas de madera. 
 
 			La pared posterior, así como el resto de las paredes están compuestas principalmente de \textit{hydropanel} y láminas de madera. En la parte trasera también se encuentra una puerta metálica y una ventana acristalada que comunican con la sala de control.
 
 			El suelo de la zona más cercana al escenario está compuesto por láminas de madera, mientras que en la zona de butacas, se compone de linóleo.
 			
 			Finalmente, el patio de butacas está constituido por 485 asientos individuales tapizados de material textil y rellenos de material esponjoso que no se ha podido consultar.
 
 			La mayor parte de los elementos y materiales citados en los párrafos anteriores pueden observarse en la figura \ref{fig:materiales}.
 
 			\begin{figure}
 				\includegraphics[scale=.25]{../imagenes/fuente.jpg}%
 				\includegraphics[scale=.25]{../imagenes/fuente2.jpg} 
 				\centering
 				\caption{El auditorio de la ETSIST. Izquierda: patio de butacas visto desde el escenario. Derecha: escenario}
 				\label{fig:materiales}

 			\end{figure}
 


		\subsection{Distribución del patio de butacas}
			Este espacio está compuesto de dos zonas de butacas distintas:

 			Por un lado, dieciseis filas de butacas paralelas en escalera de veintiocho butacas cada una. Estas butacas se encuentran numeradas desde el centro hacia los extremos; los números pares se encuentran hacia la derecha (mirando desde el escenario, ver figura \ref{fig:materiales}) y los números impares se encuentran hacia la zona de la izquierda.
 
 			Por otro lado, existen otras dos filas de butacas situadas por delante del primer grupo. Estas tres filas se encuentran divididas, a su vez, en tres grupos: uno centrado y otro en cada lado (izquierda y derecha). La numeración sigue el mismo orden que las filas mostradas anteriormente. La numeración comienza en el centro del area central y los números pares se incrementan hacia la derecha. Esta numeración continúa en las zonas adyacentes como si se trataran de la misma fila. Se produce una situación análoga con las impares en la zona izquierda del auditorio.
 
 			En la figura \ref{fig:auditorio} se puede observar un esquema en planta del auditorio en el que se aprecian tanto las filas convencionales, como las áreas adicionales.
			
			
    \section{Primera toma de datos}
	    \subsection{Numeración y selección de las posiciones de toma de datos}
		    Como se plantea disponer de un gran número de respuestas al impulso, es importante determinar previamente un sistema para numerar inequívocamente las distintas posiciones en las que se van a tomar los datos. Por suerte, tanto las filas como las butacas ya se encuentran numeradas de antemano. Por ello, se decide mantener esta numeración para almacenar y referencias los datos de cada una de las posiciones. De esta forma, se tiene que para la zona de butacas paralelas la numeración es de la forma: ``FxBy''. Siendo ``x'' el número de la fila en la que se encuentra e ``y'' el número de la butaca.

		    Para la zona delantera, la numeración es ligéramente distinta y es de la forma ``AxBy''. Donde, de nuevo, ``x'' e ``y'' representan el número de fila y butaca respectivamente.

		    De esta forma, dos ejemplos pueden ser ``F3B27'' y ``A2B8''; siendo las posiciones de la butaca 27 de la fila 3 de butacas paralelas y la butaca 8 de la fila 3 de la zona delantera, respectivamente.

		    Con esta forma de numeración de butacas, se puede proceder a la selección de las diferentes posiciones en las que se obtienen las respuestas impulsivas de la sala. Para el caso de este proyecto, se ha supuesto que el salón de actos es simétrico en la parte izquierda y derecha. Por este motivo, se han determinado posiciones en la parte par de la sala. Después, se ha distribuido el área seleccionando una butaca y dejando butacas libres antes de seleccionar la siguiente posición para cada fila. Así mismo, para la fila siguiente no se selecciona la butaca que está en frente de la seleccionada, sino la inmediatamente a continuación a su derecha. De esta forma, se consigue un muestreo más uniforme de la sala. En total se tomaron 100 grabaciones en esta primera toma de datos y en la figura \ref{fig:butacasMarcadas} puede observarse un esquema con dicha distribución. 
		
		    \begin{figure}
	            \includegraphics[scale=0.1]{../imagenes/auditorio_butacas_marcadas.png}
			    \centering
			    \caption{Distribución de las posiciones de grabación de las respuestas impulsivas. El cuadrado azul marca la posición de la fuente de sonido y los círculos rojos las posiciones de los micrófonos.}
			    \label{fig:butacasMarcadas}
	        \end{figure}

		    Como la toma de los datos se ha realizado en el transcurso de dos días, se ha optado por repetir las medidas en algunas posiciones. De esta forma se pretende obtener más información y asgurarnos de que la respuesta en dicha posición no varía en gran medida de un día para otro (por variaciones de temperatura, humedad, etc.). También se seleccionaron algunas posiciones de la parte impar del escenario. El motivo era para comprobar en la versión preliminar del test si nuestra hipótesis inicial de considerar la distancia como un elemento clave dentro de la percepción subjetiva era respaldada por un pequeño número de participantes. Estas posiciones se encuentran en las posiciones estremas de la sala; es decir, las zonas más cercanas y lejanas al centro del escenario en la primera y última fila. 

	    \subsection{Equipo utilizado para la toma de los datos}
	        Para la grabación de las respuestas impulsivas se ha utilizado el siguiente equipamiento:
	        \begin{itemize}
	            \item Ordenador portátil ASUS 2 con Windows 10 y software \textit{Dirac} instalado.
	            \item Tarjeta de sonido MOTU UltraLite-mk3 Hybrid.
	            \item Amplificador de potencia Crown XLS 2002.
	            \item Fuente dodecaédrica AVM DO-12.
	            \item Micrófono ominidireccional AKG-CK92.
	            \item Micrófono bidireccional AKG-CK94.
	            \item Alargaderas y cableado necesario para el conexionado de los equipos.
	            \item trípodes y sujecciones para los micrófonos.
	        
	        \end{itemize}
	    
	        En la figura \ref{fig:equipos} se pueden observar imágenes de varios de dichos equipos, en la figura \ref{fig:control} se muestra cómo se organiza la zona de control para la realización de las medidas ubicada dentro del mismo salón de actos y en la figura \ref{fig:bloques1} se puede observar el diagrama de bloques de los equipos conectados para la toma de datos.
	    
	        \begin{figure}[H]
	            \includegraphics[scale=0.6]{../imagenes/equipos.jpg}
			    \centering
			    \caption{De izquierda a derecha y arriba a abajo: Fuente dodecaédrica, disposición de los micrófonos de medida, tarjeta de sonido MOTU y amplificador de potencia Crown.}
			    \label{fig:equipos}
	        \end{figure}
	    
	        \begin{figure}[H]
	            \includegraphics[scale=0.4]{../imagenes/control.jpg}
			    \centering
			    \caption{Disposición de la zona de control para la realización de las grabaciones.}
			    \label{fig:control}
	        \end{figure}
	        
	        \begin{figure}[H]
	            \includegraphics[scale=0.5]{../imagenes/diagrama_bloques1.png}
			    \centering
			    \caption{Diagrama de bloques utilizado para la primera toma de datos}
			    \label{fig:bloques1}
	        \end{figure}
	    
        \subsection{Metodología para la grabación de las respuestas al impulso}
            Una vez escogidas las posiciones de medida y conectados los equipos, se procede a realizar una medida de las condiciones atmosféricas. Más concretamente, se toma una medida de la temperatura y la humedad relativa presente en el recinto. En el día 2 de junio de 2021 (primer día de mediciones) se registró una temperatura de 27ºC y una humedad relativa del 34.5\% tanto al inicio como al fin de las medidas. El 9 de junio de 2021 (segundo día) se obtuvieron las mismas temperaturas y humedades relativas que el primer día.
        
            A continuación, se coloca la fuente dodecaédrica en el centro del escenario, que es el punto de referencia que se va utilizar para realizar las medidas; también se colocan ambos micrófonos en el trípode, a la altura que se correspondería con los oídos de una persona de ``altura media'', y se situa en la posición más lejana respecto de la fuente que se va a medir. Una vez situados, se enciende el amplificador, la tarjeta de sonido y el ordenador. Tras comprobar que el ordenador detecta correctamente la tarjeta de sonido, se inicia el programa \textit{Dirac}. En él, se accede a los parámetros de configuración de las medidas fijando las entradas y salidas de la tarjeta de sonido correspondientes. También se determina la señal con la que se van a obtener las respuestas al impulso. En nuestro caso, se opta por un barrido sinusoidal lineal desde los 20 Hz a los 20 kHz de algo más de cinco segundos de duración. Se opta por esta duración porque el tiempo de reverberación aproximado del salón de actos es de algo más de 1.2 segundos. Por este motivo, el barrido debe tener una duración de al menos el doble de duración y esta opción tiene margen suficiente.
        
            Una vez configurado el software, se realizan varias pruebas de la medida ajustando la ganancia del amplificador de forma que se obtiene una relación señal a ruido suficiente. Es en este momento en el que se aprecia un ruido eléctrico debido al sistema de iluminación del auditorio y se apagan para minimizar dicho ruido. También se optó por no encender el sistema de climatización del auditorio. Para el cálculo mediante la estimación del tiempo de reverberación T20, se requiere una relación señal a ruido de 35 dB en todas las bandas de frecuencia. Este ajuste se realiza teniendo especial precaución de que la fuente no empiece a distorsionar. Una vez ajustado el amplificador, se repite la medición con los micrófonos situados en la posición más cercana a la fuente para comprobar que los niveles recibidos no saturan.
        
            Con estas comprobaciones realizadas, se procede a la obtención de las respuestas impulsivas. Se comienza desde las filas más alejadas a la fuente y se va avanzando una fila cada vez que se concluye la anterior siguiendo los esquemas presentandos en el apartado de ``Numeración y selección de las posiciones de toma de datos''. Para cada posición, una vez concluidos los dos barridos que realiza el software, se comprueba que la relación señal a ruido en todas las bandas sea superior a 35dB, como se explicó anteriormente. Si se produce algún evento extraño durante la reproducción o se sospecha que la respuesta es errónea, se repite la grabación. Este procedimiento se repite en todas las posiciones a lo largo de las dos sesiones de medidas.
        
            Es importante remarcar que se aseguró de que los cables utilizados y la ganancia aplicada fuera la misma en ambas sesiones. También se repitieron medidas en algunas filas con el fin de comprobar posteriormente que las respuestas son indistinguibles por parte de los oyentes.
        
            Al  final del proceso de toma de datos, se obtuvieron un total de 100 archivos de audio en formato ``.wav'' con la respuesta al impulso obtenida en un total de 87 posiciones distintas.
            
            \section{Procesado de los datos - Auralizaciones}
                Una vez obtenidos los datos, se planteó la cuestión sobre qué tipo de audio sería más conveniente: si las respuestas al impulso o bien dichas respuestas convolucionadas con otros audios anecoicos (también conocido como audios \textit{auralizados}). En un primer momento se optó por la segunda opción por varios motivos.
        
                El primero es que las personas están más acostumbradas a escuchar, y por tanto comparar, sonidos de carácter más melódico que se alejan de las respuestas al impulso, las cuales son parecidas al producido por un disparo.
        
                Otro motivo es que este sistema permite escoger un tipo de sonido que se ajuste en buena medida al tipo de evento para el que la sala está pensada, con la ventaja añadida de que no se necesita que dicho evento se produzca en el momento de las grabaciones.
        
                El audio utilizado está obtenido de la página web de OpenAir de la universidad de York que incluye numerosos audios anecóicos de diferentes categorías y también respuestas impulsivas de diferente tipos de espacios grabados por todo el mundo. Particularmente, se escoge la grabación de una voz femenina haciendo una escala musical. Esto se corresponde con el uso principal del Salón de actos que consiste en conferencias y música de coro.
        
                Para la auralización se utiliza el software de Cycling74 \textit{MaxMSP}\footnote{https://cycling74.com/products/max} junto con el plugin de la Universidad de Huddersfield, ``\textit{HISSTOOLS}''\footnote{http://eprints.hud.ac.uk/id/eprint/14897/}. Este programa permite realizar la convolución de la respuesta al impulso grabada con el audio anecoico de forma que se obtiene el audio auralizado que se desea. Para ello, se crea un pequeño script o ``\textit{patch}'' que realiza dicho procedimiento para todas las respuestas impulsivas previamente grabadas y se almacenan las convoluciones en una nueva carpeta de forma automática. Dicho \textit{patch} puede observarse en la siguiente figura \ref{fig:convolver_max}.
                
                Por comodidad, se mantiene la misma numeración y terminología para los audios convolucionados que se había seguido para las grabaciones; siendo la única diferencia la carpeta donde se almacenan.
        
                \begin{figure}[H]
	                \includegraphics[scale=0.4]{../imagenes/convolver_max.png}
			        \centering
			        \caption{Patch en \textit{MaxMSP} para la convolución de las respuestas impulsivas con el audio anecoico.}
			        \label{fig:convolver_max}
	            \end{figure}
        
                
            
    \section{Segunda toma de datos}
        Durante fases posteriores del proyecto, una vez realizado el test perceptual, se repasó nuevamente la metodología seguida y se encontró una posible fuente de error en el procesado de las señales adquiridas en esta primera toma de datos. El error se debía a un posible fallo en la configurción del programa \textit{Dirac} que podía traducirse en que las señales que se obtenían tras el procesado no se correspondieran exáctamente a la respuesta biaural de las respuestas al impulso entre los puntos de emisión y recepción en el auditorio.  Ante las dudas razonables sobre si ese procesado podía afectar a los resultados del proyecto, se decidió realizar una nueva toma de datos aprovechando además la disponibilidad en ese momento de unos micrófonos biaurales. Con estas nuevas grabaciones se generó el test subjetivo definitivo
        
        
       \subsection{Numeración y selección de las butacas en la segunda toma de datos}
            Para la segunda toma de datos, se siguió la misma numeración que la vez anterior. Es decir, para las butacas normales, se sigue la estructura ``FxBy'' siendo ``x'' e ``y'' el número de fila y de butaca respectivamente tal y como ya se mostró en el apartado de la primera toma de datos.
                
            Del mismo modo, para las tres filas más próximas al escenario, la numeración utilizada es de la forma ``AxBy'' en la que, de nuevo, ``x'' e ``y'' tienen el mismo significado.\newline
                
            La selección de las posiciones de grabación también sigue el mismo criterio aplicado la otra vez: se selecciona una de cada tres butacas de la mitad derecha del escenario. En la fila siguiente se desplaza la primera butaca una posición a la derecha. Puede volverse a consultar la figura \ref{fig:butacasMarcadas} para observar todas las posiciones de grabación.
                
            Al contrario que la otra vez, esta vez no se han seleccionado las posiciones extremas del lado izquierdo ya que, tras el experimento previo, se vio que los participantes podían distinguirlos perfectamente. Del mismo modo, como la toma de datos se realizó en un único día, no se repitieron medidas en ninguna posición, aunque tampoco sería necesario porque ya se observó tras el primer test que los oyentes eran incapaces de distinguir las señales entre sí. Esto puede comprobarse en el apartado de ``Análisis del test previo'' dentro del capítulo ``Análisis estadístico''.
                
                
        \subsection{Equipo utilizado para la segunda toma de datos}
            Para esta nueva toma de datos, la principal diferencia en torno al equipo utilizado se encuentra en los micrófonos. En este caso, se usaron unos micrófonos biaurales ``Roland CS-10EM'' que tienen la particularidad de poseer la forma de auriculares \textit{in-ear} y se conectan mediante un conector mini-jack estéreo. Estos micrófonos necesitan que se les suministe una alimentación de entre 2 y 5 Voltios, algo que no es posible en la tarjeta de sonido MOTU que se utilizó en la anterior toma de datos. Por este motivo, se realiza la adquisición de las señales a través de una grabadora de audio externa que permite aplicar dicha alimentación. Para nuestro caso, se ha optado por una ``Yamaha pr7''. En la figura \ref{fig:microsBi} pueden observarse tanto los micrófonos como la grabadora.
                
            \begin{figure}[H]
                \includegraphics[scale=0.3]{../imagenes/MicroBi.jpg}
                \centering
                \caption{Micrófonos biaurales Roland y grabadora Yamaha utilizada para la grabación de las respuestas.}
                \label{fig:microsBi}
            \end{figure}
                
            La tarjeta de sonido se sigue utilizando para controlar la fuente de sonido dodecaédrica al igual que la vez anterior.
                
            A modo de resumen, se adjunta una nueva lista con todo el equipamiento utilizado para la grabación de las respuestas:
                
            \begin{itemize}
                \item Ordenador portátil ASUS 2 con Windows 10 y software \textit{Dirac} instalado.
	            \item Tarjeta de sonido MOTU UltraLite-mk3 Hybrid.
	            \item Amplificador de potencia Crown XLS 2002.
	            \item Fuente dodecaédrica AVM DO-12.
	            \item Alargaderas y cableado necesario para el conexionado de los equipos.
	            \item Micrófonos biaurales Roland CS-10EM.
	            \item Grabadora portátil Yamaha pr7.
            \end{itemize}
                
            En la figura \ref{fig:equipos}, ya presentada anteriormente, pueden observarse imágenes de algunos de los equipos ya mencionados y en la figura \ref{fig:bloques2} X el nuevo diagráma de bloques para esta toma de datos.
            
            \begin{figure}
	            \includegraphics[scale=0.47]{../imagenes/diagrama_bloques2.png}
			    \centering
			    \caption{Diagrama de bloques utilizado para la segunda toma de datos}
			    \label{fig:bloques2}
	        \end{figure}
            
            
                
        \subsection{Metodología para la grabación de las respuestas al impulso}
            La metodología seguida para la grabación de esta tanda de señales es muy similar a la ya comentada anteriormente. Las medidas se hicieron el 15 de octubre de 2021.
                
            En primer lugar, se procede a hacer una escucha de las fuentes de ruido que se aprecian dentro del espacio. De esta forma, se localizó un ruido eléctrico debido a la iluminación del salón de actos. Este ruido desaparecía al apagar las luces, por lo que se optó por dejarlas apagadas durante las grabaciones, ya que desde los ventanales entraba luz suficiente para poder realizar el trabajo. Del mismo modo, se localizó un ruido esporádico procedente de uno de los altavoces montados en el escenario. A primera vista parecía debido a algún problema de contacto, pero no había forma de asegurarse. Por suerte, era de un nivel aceptable y su duración era muy escasa de forma que no se percibía mientras se reproducían los barridos.
                
            Una vez analizadas las diferentes fuentes de ruido, se procede a calibrar el sistema de forma que la relación señal a ruido en las posiciones más lejanas respecto de la fuente sean aceptables (mayor o igual a 35dB) y que ni la fuente ni los micrófonos saturaran en ningún momento.
                
            Una vez hecha esta comprobación, se comenzó con las grabaciones. El orden seguido fue desde las filas más alejadas a la fuente a las más cercanas. La persona encargada de ser sujeto de las medidas se colocaba los auriculares y se sentaba en la butaca que correspondía como se aprecia en la figura \ref{fig:Nico}. Una vez colocado, se comenzaba la reproducción de cuatro rondas de barridos frecuenciales de 20Hz a 20kHz. El sujeto debía iniciar manualmente la grabación una vez terminaba de reproducirse el primer barrido y finalizarla una vez terminaba el tercer barrido. Una vez realizada la grabación, se comprobaba que se había grabado correctamente y se pasaba a la siguiente posición. 
                
            \begin{figure}[H]
                \includegraphics[scale=0.3]{../imagenes/Nico.jpg}
                \centering
                \caption{Ejemplo de la disposición del sujeto para la grabación de las señales. Los micrófonos se encuentran en sus oídos.}
                \label{fig:Nico}
            \end{figure}
                
            En total se obtuvieron unas 83 grabaciones que, a continuación, se procesaban en el mismo ordenador utilizando el software \textit{Dirac} con el fin de obtener las respuestas al impulso. Para ello, se utiliza el mismo menú utilizado en la primera toma de datos, pero esta vez, en lugar de realizar una grabación, se importa la señal ya grabada y se le pide al programa que calcule la respuesta al impulso utilizando como señal de referencia la misma utilizada en la reproducción de la fuente. De forma que el programa genera al final un archivo como el que se observa en la figura \ref{fig:IRDirac}.
            
            \begin{figure}[H]
                \includegraphics[scale=0.3]{../imagenes/IRDirac.png}
                \centering
                \caption{Respuesta al impulso obtenida con el software \textit{Dirac}.}
                \label{fig:IRDirac}
            \end{figure}
            
        \subsection{Procesado de los datos}
            Para esta segunda toma de datos, se decidió no realizar el proceso de auralización de las señales con los audios anecoicos. Esto se debe a que el proceso de auralización puede incluir una serie de variables aleatorias adicionales que pueden condicionar los resultados. Además, para la detección de diferencias perceptuales, que es el objetivo del proyecto, las respuestas al impulso tienen suficiente información acústica para realizar el proyecto. Este sistema cuenta con la ventaja adicional de que la duración de las señales es menor, por lo que se reduce la posibilidad de que los participantes experimenten fatiga auditiva durante la realización del test.
            
            A nivel de procesado para esta parte, sólo se ha recortado la duración del fichero de audio para ajustarlo a la duración de la señal y no hacer que los oyentes escucharan largos tramos de silencio. En ningún momento se modifican las ganancias de las señales para evitar que la información acústica pueda quedar modificada respecto de los valores de grabación. El ajuste de volumen para que sea cómodo se realizará en el test mediante los ajustes de volumen del sistema operativo.
            
    \section{Medida de distancias clave}
        Para un correcto análisis de los resultados, es necesario realizar una serie de medidas en el auditorio. Esto nos permite, más adelante, poder organizar los resultados en función de diferentes referencias y ver si existe alguna relación entre ellas. En nuestro caso, hemos considerado como importantes las distancias de la fuente dodecaédrica a la primera butaca, la distancia entre filas, entre butacas, el ancho del pasillo, entre otras. Las distancias que se midieron pueden observarse en la figura \ref{fig:distancias} y el resto pueden obtenerse a partir de las que ahí vienen reflejadas.
        
        \begin{figure}
                \includegraphics[scale=0.4]{../imagenes/AuditorioMedidasButacas.pdf}
                \centering
                \caption{Medidas de distancias tomadas dentro del auditorio.}
                \label{fig:distancias}
            \end{figure}
            
\chapter{Desarrollo de la aplicación del test subjetivo.}
        En este capítulo se explican las herramientas y lenguajes utilizados para la creación de la aplicación de escritorio con la que se realizan posteriormente los test subjetivos de audio necesarios para el proyecto. Así mismo, se explica el algoritmo utilizado y las diferentes versiones utilizadas para la creación de la interfaz de usuario.
        
        \section{Lenguajes y programas utilizados}
            Para el desarrollo de la aplicación, se ha optado por utilizar el lenguaje de programación Python3. Este lenguaje tiene la ventaja de que incluye numerosas librerías que permiten trabajar con interfaces gráficas con las que los participantes de los test puedan interactuar de forma sencilla. Para nuestro caso particular, se ha optado por utilizar la librería del entorno gráfico GTK (la librería se llama \textit{PyGObject}\footnote{https://pygobject.readthedocs.io/en/latest/}) que es una de las más utilizadas en entornos Linux, aunque también puede utilizarse en entornos de Windows o Mac.
            
            Para el diseño de la interfaz, se podría haber codificado diréctamente en lenguaje python. No obstante, se ha utilizado el programa \textit{Glade}\footnote{https://glade.gnome.org/} que permite utilizar un entorno gráfico para la creación de todos los elementos de la interfaz. Con dicho programa se obtiene un fichero XML con extensión ``.glade'' que es el que el \textit{script} de Python lee y con el que se genera la interfaz que el usuario utiliza. En la figura \ref{fig:gladeInic} se puede observar la pantalla de inicio de dicho programa. 
            
            \begin{figure}[H]
                \begin{center}
                    \includegraphics[scale=.2]{../imagenes/gladeInicio.png}
                    \caption{Ventana de inicio del programa \textit{Glade}.}
                    \label{fig:gladeInic}
                \end{center}
            \end{figure}
            
           \newpage

           Para la codificación de los scripts de Python se ha utilizado el programa \textit{Gnome Builder}\footnote{https://apps.gnome.org/app/org.gnome.Builder/}; un entorno de aplicaciones que incluye todas las herramientas para depurar, compilar y ejecutar los scripts dentro del mismo espacio. Al igual que con \textit{Glade}, el programa es gratuito de código abierto. En la figura \ref{fig:builderIni} se encuentra una captura de la interfaz del programa.
           
            \begin{figure}[H]
                \begin{center}
                    \includegraphics[scale=.2]{../imagenes/builderIni.png}
                    \caption{Captura del programa \textit{Gnome Builder}.}
                    \label{fig:builderIni}
                \end{center}
            \end{figure}
            
        \section{Características de la interfaz de usuario}
            Como se ha comentado en el apartado anterior, la interfaz de usuario se ha diseñado utilizando el programa \textit{Glade}. Desde un principio, se ha querido diseñar una interfaz que sea lo más simple posible, no sólo por facilidad para realizarla, sino para que las personas participantes puedan centrarse exclusivamente en los objetivos del test y hacer que su uso no suponga ninguna dificultad.  
            
            En una primera instancia se decidió generar una interfaz en la que los elementos principales fueran dos botones con los que el usuario pudiera escoger cuál de los audios quería reproducir. Las respuestas se recogerían en dos casillas de verificación, o \textit{toggles}. También se incluyó un botón en la parte inferior para enviar cada una de las respuestas. En la parte superior se incluye un texto que indica el número de pregunta por la que va el test. Todos estos elementos se pueden observar en la figura \ref{fig:uiIni}.
            
            \begin{figure}[H]
                \begin{center}
                    \includegraphics[scale=.6]{../imagenes/uiIni.png}
                    \caption{Versión inicial de la interfaz de usuario.}
                    \label{fig:uiIni}
                \end{center}
            \end{figure}
            
            Esta configuración es la que se utiliza en el test previo. Al finalizar dicho test, se les pidió a los participantes que propusieran diferentes mejoras para perfeccionar la interfaz de usuario. Entre ellas, las más habituales consistieron en la sustitución de los \textit{toggles} por elementos más grandes al estilo de \textit{switches} o interruptores y la actualización de la posición de dichos elementos a su estado inicial entre cada una de las preguntas.
            Atendiendo a estas propuestas, se modifica la interfaz con el resultado que se muestra en la figura \ref{fig:uiFin}.
            
            \begin{figure}
                \begin{center}
                    \includegraphics[scale=.6]{../imagenes/interFin.png}
                    \caption{Versión final de la interfaz de usuario.}
                    \label{fig:uiFin}
                \end{center}
            \end{figure}
            
        \section{Codificación del script en Python}            
            El procedimiento seguido para la codificación de la aplicación ha sido gradual. En primer lugar se han diseñado diferentes métodos y funciones que se apliquen diréctamente desde la consola. De esta forma, se comprueba que el algoritmo funciona correctamente y que se extraen los resultados en el formato deseado.
            
            Una vez, se ha comprobado que este \textit{script} funciona, se modifica el mismo haciendo que las funciones sean llamadas por los diferentes elementos de la interfaz de usuario ya creada y que se referencian dentro del código de Python.
            
            Durante todo el proceso, se realizaron consultas a la documentación de la API de Python para GTK\footnote{https://python-gtk-3-tutorial.readthedocs.io/en/latest/} y de las diferentes librerías necesarias para el desarrollo de la aplicación. Varios ejemplos de estas librerías son:
            \begin{itemize}
                \item \textbf{numpy}\footnote{https://numpy.org/}: para el trabajo con matrices al estilo de programas como \textit{Matlab}.
                \item \textbf{pandas}\footnote{https://pandas.pydata.org/}: para la edición y creación de estructuras de datos.
                \item \textbf{os}\footnote{https://docs.python.org/3/library/os.html}: para la navegación dentro de sistemas de ficheros del sistema operativo.
                \item \textbf{gi}\footnote{https://pygobject.readthedocs.io/en/latest/guide/api/api.html}: para el manejo de los elementos gráficos y la gestión de sus señales.
            \end{itemize}
            
            Por último durante el periodo de tiempo entre los dos tests subjetivos de audio se añadió la funcionalidad que permitía reproducir los dos audios pulsando las teclas ``1'' y ``2'' del teclado respectivamente sin necesidad de utilizar el ratón. Esta funcionalidad se incorporó como sugerencia por parte de varias de las personas participantes que comunicaron las dificultades de escuchar los audios con los ojos cerrados (decisión personal) al tener que pinchar los botones con el ratón del ordenador.
            
            \subsection{Estructura del código}
                El código de Python se encuentra dividido en diferentes partes:
                En primer lugar, la zona donde se importan las diferentes librerías para el correcto funcionamiento del script. Aquí se incluyen las que permiten reproducir los diferentes audios, manejar información, navegar por el sistema de archivos del ordenador, la interfaz gráfica, etc.
                
                A continuación, se encuentra una zona donde se definen diferentes funciones que serán usadas repetidamente a lo largo del \textit{script}. 
                
                Después, se definen algunas variables globales necesarias y da comienzo la clase \textit{Main} de la aplicación. En ella, se inicializan las referencias a los diferentes elementos de la interfaz gráfica con las que los usuarios interactúan y se establecen qué funciones se llaman en función de las señales que manejan. Estas funciones se definen en último lugar a continuación de dichas inicializaciones. En el anexo C se puede leer el código en su totalidad.
                
\chapter{Realización de los test subjetivos de audio}
        Como ya se estableció en el capítulo de ``Introducción'', el objetivo del proyecto es el de analizar las diferencias perceptuales que se obtienen al escuchar las respuestas al impulso obtenidas al modificar la posición de escucha o recepción dentro de un determinado espacio.
        
        Para simplificar la toma de datos del estudio, se parte de la premisa de que los oyentes perciben diferencias sonoras en función de la distancia a la que se encuentre la fuente sonora del receptor, pero que dichas percepciones son similares para posiciones simétricas dentro del espacio de estudio.
        
        Para validar esta hipótesis, que permitiría que las muestras recogidas fueran de sólo la mitad del auditorio, se plantea un test previo que se describe en el apartado ``Test preliminar''
        
        Una vez comprobada la hipótesis y realizada la adquisición de todos los datos necesarios, se diseña un segundo test más completo y con una interfaz de usuario más refinada a partir de las experiencias adquiridas en el test anterior. No obstante, por los motivos ya comentados en el apartado 2.4., se estimó conveniente repetir la toma de datos y realizar dicho test de nuevo. En el apartado ``Características de los test finales'' se explican las particularidades de ambas tomas de datos y sus respectivas diferencias.
        \section{Test preliminar}
            En este apartado se explican las características del test previo que se realiza para determinar que la hipótesis inicial es, a priori, correcta. Esta hipótesis afirma que las dimensiones y características de la sala permiten que la distancia sea un factor perceptible por los oyentes.
            
            \subsection{Características del test previo}
                Para la realización de este test, se han seguido las recomendaciones propuestas en \cite{Tejada2020}. Para ello, se plantean las siguientes preguntas:
                \begin{itemize}
                    \item \textbf{¿En qué categoría se engloba el estudio?}: Para responder a esta pregunta, hay que observar las diferentes categorías que se presentan en \cite{Tejada2020}. Estas categorías son: Sensibilidad, Localización, Molestia, Inteligibilidad y Reverberación. Para el caso de este test, el que más se corresponde es el de Sensibilidad, puesto que el objetivo es ver si las personas participantes tienen la capacidad para distinguir si los sonidos son iguales o no.
                    \item \textbf{¿Qué tipo de test es el más adecuado?}: Para los objetivos de este test, se ha optado por un test 2-AFC también conocido como \textit{Same/Different}. Este tipo de test obliga al participante a elegir entre dos opciones; en nuestro caso, si los audios que se presentan son iguales o diferentes. Esto tiene la ventaja de que es fácil de codificar y analizar estadísticamente aunque no da mucha información concreta, pero es suficiente para el objetivo del proyecto.
                    \item \textbf{¿Participantes expertos o inexpertos?,¿cuántos necesito?}: Como este test no es el definitivo, sino que se trata de una comprobación previa, no se han considerado requisitos tan estrictos como los que se hubieran seguido de otra forma. Por este motivo, se ha optado por utilizar participantes con y sin experiencia. El número total de participantes ha sido de 10 que es lo mínimo que se ha utilizado en algunos test de temática similar \cite{2019MNowak}.
                    \item \textbf{¿Qué tipo de señal utilizo?}: En nuestro caso, se utilizan las respuestas al impulso convolucionadas obtenidas previamente del salón de actos siguiendo el procedimiento seguido en el capítulo de "Toma de datos". En concreto para este experimento se han utilizado las respuestas de las zonas más cercanas y lejanas a la fuente en las posiciones centrales y de los extremos laterales tanto izquierdo como derecho. En la figura \ref{fig:posicionesTestPrueba} se muestran las posiciones que formaban parte del test.
                    
                    \begin{figure}[H]
                    \includegraphics[scale=0.1]{../imagenes/auditorio_butacas_prueba.png}
			        \centering
			        \caption{Posiciones de las grabaciones utilizadas para el test de prueba.}
			        \label{fig:posicionesTestPrueba}
                \end{figure}
                    
                    \item \textbf{¿Cuál es tipo de análisis de datos?}: Para el análisis de los datos, se ha optado por realizar un análisis del coeficiente $\alpha$ de acuerdo con las especificaciones de la norma ISO-10399 \cite{ISO10399}. En dicha norma, se proporciona una tabla que muestra el número de respuestas correctas necesarias para concluir que dos eventos son diferentes con una determinada incertidumbre para un determinado número de respuestas totales.
                \end{itemize}
 
            \subsection{Procedimiento del test previo}
                Una vez determinadas todas las características del test previo, se procede a completar el test de forma individual a cada uno de los diez participantes. La sala donde se realizaron las escuchas fue una de las salas de laboratorio del Centro de Investigación en Tecnologías Software y Sistemas Multimedia para la Sostenibilidad (CITSEM). Esta sala se encuentra aislada de las zonas de tránsito y sus ventanas no dan a zonas con tráfico. De esta forma, se pueden controlar fácilmente las fuentes de ruido ajenas al experimento. Es importante remarcar que, debido a la pandemia producidad por la COVID-19, las ventanas tenían que estar abiertas en todo momento con el fin de conseguir unas condiciones de ventilación apropiadas que evitaran posibles contagios. Este es el motivo por el que se optara por una sala cuyas ventanas no dieran a fuentes de ruido externas.
                
                El equipo utilizado consistió en un ordenador portátil con sistema operativo Debian con 12GB de memoria RAM y unos auriculares Tascam-TH02. La escucha de los diferentes audios y la inclusión de las respuestas se realiza mediante la interfaz gráfica que ya se explicó en el apartado ``Desarrollo de aplicación en Python''. En la figura \ref{fig:interfazInicial} se puede observar la interfaz utilizada en este test. Previa a las escuchas por parte de los participantes, se comprobó que las señales se reprodujeran con un nivel adecuado. Además, se les explicaba a los participantes las carácterísticas del test que iban a realizar, así como el número de preguntas y en qué propiedades debían fijarse para facilitarles su realización. También se les pedía que leyeran y firmaran un ``consentimiento informado'' cuya estructura puede consultarse en el anexo D. Una vez finalizada la realización del test y, debido al carácter preliminar del mismo, se les preguntó sobre posibles mejoras que se podían hacer para facilitar su realización, como cambios en la interfaz gráfica, la inclusión de audios de ejemplo, etc.
                
                 \begin{figure}[H]
                    \includegraphics[scale=0.6]{../imagenes/uiIni.png}
			        \centering
			        \caption{Interfaz gráfica para la realización del test previo.}
			        \label{fig:interfazInicial}
                \end{figure}
                
        \section{Experimentos finales}
            En este apartado se analizan las características y procedimientos utilizado en los experimentos finales.     
            
            \subsection{Características de los test finales}
                Como ya se comentó en el análisis del estado del arte, no existe un consenso sobre la mejor forma de realizar un test subjetivo de audio. Algunos estudios como \cite{Tejada2020}, \cite{delaPrida2019} y \cite{delaPrida2021} hacen sugerencias sobre qué características pueden funcionar mejor en función de lo que se pretenda analizar en dicho test. Para nuestro caso particular, se ha vuelto a optar por seguir las recomendaciones propuestas en \cite{Tejada2020} respondiendo a las preguntas que en él se plantean. Como ya se comentó en el apartado de ``Toma de datos'', se tuvieron que repetir los test de audio debido a un fallo en el procesado de las grabaciones. A pesar de esto, las características de ambas pruebas son prácticamente idénticas. Por este motivo, se engloban todas ellas en el mismo apartado. Dentro de cada subapartado se comentan, si existen, las diferencias que pueda haber entre ambas.
            
                \begin{itemize}
                
                    \item \textbf{¿En qué categoría se engloba el estudo?}: Al igual que en el test previo, se pretende determinar la sensibilidad que tienen las personas para distinguir las diferencias perceptuales al escuchar sonidos cuando se encuentran en distintas posiciones respecto de la fuente acústica. Por este motivo, se puede concluir que este test se encuentra englobado dentro de los que se consideran como de Sensibilidad.
                    \item \textbf{¿Qué tipo de test es el más adecuado?}: Para el estudio que se pretende hacer, existen numerosos tipos de test que pueden arrojar datos interesantes. Entre ellos, los dos más habituales son los de referencia oculta como ABX o los Duo-Trio. No obstante, se ha optado por repetir el tipo de test utilizado en el estudio previo; es decir 2-AFC o \textit{Same/Different}. Los motivos que han llevado a esta elección han sido la facilidad de implementación y codificación, su simpleza que lo hace apropiado para todo tipo de participantes y que permite arrojar resultados bastante fiables mediante un análisis relativamente sencillo usando modelos Thurstonianos, además de que existen normas internacionales que permiten obtener resultados extra que pueden ser interesantes de comparar con los de dichos modelos.
                    \item \textbf{¿Participantes expertos o inexpertos?, ¿cuántos necesito?}: A la hora de plantear el experimento, desde un primer momento se planteó como una situación que afectaba a la población general. Por este motivo, se consideró que no era relevante que los participantes tuvieran una experiencia particular en la realización de este tipo de test, aunque obviamente, es necesario que no padezcan de ningun problema auditivo que afecte a su nivel de audición, ya sea por la edad, enfermedad, etc. A pesar de todo esto, se consideró que todas personas participantes debían pasar por una breve sesión de entrenamiento previo para que supieran de antemano en qué elementos de los sonidos tenían que fijarse. Estas indicaciones vienen reflejadas en las normas \cite{UIT1116, UIT1534, UIT1284, EBU3286, UIT1285, UIT1286} que muestran la utilidad de este tipo de entrenamientos para obtener mejores resultados.
                
                    En cuanto al número de participantes totales, se siguieron las recomendaciones de las mismas normas, así como las conclusiones extraídas de \cite{Tejada2020} y de los otros experimentos consultados de temática similar \cite{2005IWitew, 2019DJSchlit, 2016SKlockgether, 2019LKritly, 2019GPulvirenti, 2019MNowak, 2011VEmiya}. De esta forma, se concluyó que el número de participantes debía de ser de un mínimo de 30 personas.
                    
                    
                    \item \textbf{¿Qué tipo de señal utilizo?}: Las señales utilizadas para la realización del primer test fueron los archivos de audio generados tras la convolución de una voz femenina, con cada una de las respuestas impulsivas grabadas en el salón de actos de la ETSIST de la UPM. El proceso de obtención de dichas respuestas se encuentra explicado en el apartado de ``Toma de datos''. Se trata de un total de 100 archivos de audio de 7 segundos de duración. Esta duración se encuentra dentro de los límites marcados por \cite{UIT1116, UIT1534, UIT1284, EBU3286, UIT1285, UIT1286} para evitar que los oyentes se ``acostumbren'' al  sonido. Así mismo, se encuentra dentro del abánico de duraciones que se ha observado en distintos proyectos de similares características. \newline
                
                    Para el segundo test, se optó por utilizar diréctamente las respuestas al impulso obtenidas tras el procesado que realiza el software \textit{Dirac}, como ya se explica también en el apartado de ``Toma de datos''. En este caso, se utilizan un total de 83 archivos distintos de 2 segundos de duración. Al igual que en la prueba anterior, la duración se encuentra dentro de los límites marcados por \cite{UIT1116, UIT1534, UIT1284, EBU3286, UIT1285, UIT1286}, así como otros estudios para evitar la fatiga auditiva y que los participantes se acostumbren a los estímulos.
                
                    Para la realización de ambos experimentos, el volumen al cual se reproducen las señales no es relevante, siempre y cuando no se modifique a lo largo de toda la sesión. Por este motivo, el participante puede escoger el que le sea más cómodo, con la única restricción de que el volumen debe ser el mismo para la escucha de los dos audios que conforman cada pregunta.
                    \item \textbf{¿Cuál es el tipo de análisis de datos?}: para los test finales se han optado por dos análisis diferentes. Por un lado, se mantiene el análisis del coeficiente $\alpha$ siguiendo las especificicaciones de la norma ISO-10399\cite{ISO10399} y las tablas que ofrece en sus anexos. Por otro lado, se realiza un análisis mediante modelos thurstonianos, ya reseñados en el capítulo de ``Estado del arte''. Ambos análisis, el de la norma ISO-10399 y los modelos thurstonianos se realizan para dos diferentes disposiciones de los datos: según la distancia del punto medio de la posición de los dos audios respecto de la fuente, y según la distancia a la que se encuentre la posición de ambos audios.
                     
                \end{itemize}
                
                Otro de los aspectos claves a determinar es la duración del test y el número de preguntas (o rondas) que tendrán que realizar los partcipantes. Las normativas internacionales y diferentes fuentes bibliográficas \cite{UIT1116, UIT1534, UIT1284, EBU3286, UIT1285, UIT1286, ZwickerFactsModels, BlauertSpatialHearing, GelfandStanley}, limitan a 30 minutos el tiempo máximo que los participantes pueden invertir en los test sin la utilización de descansos y evitar así la fatiga auditiva. Esto limita el número de rondas que los participantes son capaces de responder en dicho tiempo y, por este motivo, es necesario ajustar la cantidad de preguntas para que se ajuste lo mejor posible a una duración de 20-25 minutos y así tener margen hasta la media hora límite.
                
                A raíz del tiempo que llevó a los voluntarios del test previo completar sus preguntas, se decidió que los test finales constaran de 25 rondas o preguntas. De esta forma se satisfacieron las limitaciones de tiempo y nuestra necesidad de obtener el mayor número de datos posible.
                
            \subsection{Procedimiento del test final}
        
                \subsubsection*{Localización}
                    En primer lugar, se organizaron los turnos para la realización del test. El espacio escogido para la realización de las escuchas consistió en un laboratorio del CITSEM localizado en el Campus Sur de la UPM. Se escogió esta localización porque, debido a la situación pandémica producida por el COVID-19, era necesario que las ventanas estuvieran abiertas en todo momento. Se planteó la utilización de aparatos de depuración del aire, pero su ruido era comparable o más molesto todavía que la realización del mismo con las ventanas abiertas. No obstante, se escogió una localización donde las ventanas daban a una zona poco transitada, tanto por peatones como por vehículos, de forma que se conseguía un ambiente suficientemente controlado y silencioso. Si antes de comenzar el test se consideraba que el ruido externo era mayor del habitual, se cerraban las ventanas durante su realización y se volvían a abrir al finalizar el experimento.
            
                \subsubsection*{Material utilizado}
                    Para la realización de ambos test se utilizó el mismo material que en el test preliminar. En concreto:
                    \begin{itemize}
                        \item Ordenador portátil con S.O. Debian 11 con procesador \textit{Intel} i5 y 12GB de RAM.
                        \item Software \textit{Gnome Builder} para la ejecución de la aplicación.
                        \item Auriculares Tascam-TH02.
                        \item Declaraciones de consentimiento.
                        
                    \end{itemize}
                
                \subsubsection*{Realización del test}
                    En primer lugar, todas las personas participantes debían inscribirse en las franjas habilitadas en un calendario con el fin de poder organizar las entradas y salidas evitando que se produjeran acumulaciones de personas en espacios cerrados. Una vez inscritas y cuando llegaba su turno, se les explicaba brevemente en qué consistía el experimento del que iban a formar parte y se les daba a leer una hoja de información al participante en la que se explica en mayor detalle las particularidades del test, las partes de las que consta, así como información relevante en temas de salud y protección de datos. Una vez leído, debían rellenar una declaración responsable aceptando los aspectos marcados en el documento anterior. Los modelos de ambos documentos pueden consultarse en su totalidad en el Anexo D.
                
                    A continuación daba comienzo el test con el tutorial. En él se le presentan al participante tres de las posibilidades que podría encontrarse (que los audios fueran de la misma posición, que fueran de filas distintas y que fueran de butacas distintas de la misma fila). Como el objetivo de este tutorial es que sepan identificar qué aspectos determinan que dos audios sean iguales o diferentes, para los tres casos se optó por utilizar casos críticos, estos son los mismos que se utilizaban en el test previo, donde las posiciones estaban lo más separadas posibles (ver figura \ref{fig:posicionesTestPrueba}). La reproducción de los audios se podía repetir el número de veces que se considerara necesario. En la figura \ref{fig:interfazTutorial} puede observarse la interfaz de dicha parte.
                
                    \begin{figure}[H]
                        \includegraphics[scale=0.6]{../imagenes/interfaz_tutorial.png}
			            \centering
			            \caption{Apariencia del entrenamiento previo}
			            \label{fig:interfazTutorial}
                    \end{figure}
                
                    Una vez el participante consideraba que estaba preparado, pulsaba el botón de ``Empezar test'' para dar inicio a la segunda parte.
                
                    En esta segunda parte al usuario se le presentan dos audios, escogidos de forma aleatoria, de todos las posicions posibles de la sala y debe determinar si lo que percibe son dos audios que suenan igual o no. Por otro lado, el participante debe especificar si se encuentra seguro de su respuesta. El usuario puede repetir la reproducción de los audios las veces que quiera, pero debe esperar a que el audio termine de reproducirse antes de poder escoger si repetirlo o escuchar el otro audio. No existe la opción de ``No sabe/no contesta'', por lo que en caso de no poder decidirse, el usuario debe escoger una de las dos opciones de forma aleatoria y marcar la opción de ``no seguro''. Una vez ha decidido su respuesta, debe pinchar en el botón de ``Enviar respuesta'' para que se le presenten los dos siguientes audios. Esto permite que, en caso de duda, los participantes puedan modificar su respuesta hasta que pulsen dicho botón. A partir de ahí, no se permite volver atrás o modificar las respuestas. El experimento termina tras las 25 rondas de preguntas. Una vez terminada esta parte, el trabajo de las personas voluntarias ha terminado y puede marcharse para dar paso al siguiente participante. En todos los casos, se dispuso de un tiempo entre cada test para seguir los protocolos establecidos por la Covid-19 (Ventilación del espacio, limpieza de todos los elementos, etc.). En la figura \ref{fig:interfazTestFin} puede observarse la interfaz utilizada para la realización de esta segunda parte.
                
                    \begin{figure}[H]
                        \includegraphics[scale=0.6]{../imagenes/interFin.png}
			            \centering
			            \caption{Apariencia de la interfaz del test.}
			            \label{fig:interfazTestFin}
                    \end{figure}
                
                    Todos los resultados se almacenan durante la ejecución del experimento en un archivo con terminación ``.csv'', que es el que se utiliza posteriormente para el análisis de los datos. Estos datos se disponen de forma anónima en el Anexo F.
                
                    Para la realización del test participaron un total de 31 voluntarios para la primera versión de este test y 34 en la segunda. Las edades del primer test estaban comprendidas entre los 18 y los 60 años, mientras que la segunda las edades se encontraban en el rango de los 18 a los 33 años. En ningún momento la duración de los test superó los 30 minutos. En la figura \ref{fig:participantes} se puede observar una imagen de varios de los partcipantes durante el transcurso del test.
                
                    \begin{figure}[H]
                        \includegraphics[scale=0.45]{../imagenes/participantes.png}
			            \centering
			            \caption{Varios de los participantes durante el transcurso de los test.}
			            \label{fig:participantes}
                    \end{figure}
                    
\chapter{Resultados de los experimentos}
    En este apartado se muestran los resultados obtenidos de los diferentes test perceptuales que se han realizado a lo largo del proyecto. 
    \section{Resultados del test previo}
        Como ya se comentó en apartados anteriores, en este test previo el objetivo consiste en comprobar que los participantes podían realmente distinguir los estímulos cuando se correspondían a las de las posiciones más alejadas entre sí. También se quiere comprobar que la sala es lo suficientemente simétrica como para simplificar la toma de datos a una de las mitades del auditorio. De esta forma, se han obtenido 70 preguntas que pueden agruparse en cuatro casos posibles: 
        \begin{itemize}
            \item \textbf{Butacas separadas horizontalmente}: Los audios se corresponden a posiciones en la misma fila, pero distinta butaca.
            \item \textbf{Butacas separadas verticalmente}: Las señales se corresponden con posiciones en distinta fila, pero mismo número de butaca.
            \item \textbf{Butacas separadas tanto vertical como horizontalmente}: Los audios se corresponden a posiciones diferentes tanto de fila como de butaca.
            \item \textbf{butacas en la misma posición}: La fila y el número de butaca para las dos señales es la misma.
        \end{itemize}
        Este test, como ya se comentó anteriormente, fue realizado por un total de 10 personas, con 7 preguntas por persona. Los resultados totales pueden observarse en la tabla \ref{tablaTestPrevio}. Para cada caso, se ha desglosado el número total de respuestas para cada posible distribución de posiciones, el número de respuestas marcadas como diferentes e iguales (independientemente del nivel de seguridad), y el número de respuestas marcadas con un ``Nivel de seguridad: poca'' en función de la respuesta. En el anexo F se encuentra una tabla con cada uno de los resultados individuales con el nombre de cada posición exacta.
        
        \begin{table}[H]
			\begin{center}
			\begin{scriptsize}
			\begin{tabular}{| c | c | c | c | c | c |}
			    \hline
				\textbf{Separación butacas}&\textbf{Respuestas}&\textbf{Iguales}&\textbf{Diferentes}&\textbf{Duda iguales}&\textbf{Duda diferentes}\\ \hline
                Horizontal&27&9&18&2&2\\ \hline
                Vertical&10&1&9&0&1\\ \hline
                Horizontal y vertical&23&5&18&1&2\\ \hline
                Misma posición&10&9&1&1&0\\ \hline
			\end{tabular}
			\caption{Resultados del test previo.}
			\label{tablaTestPrevio}
			\end{scriptsize}
			\end{center}	
		\end{table}	
	\section{Resultados del test final}
	    Para este test, se han agrupado los resultados de dos formas de manera que resulta más fácil entenderlos para el análisis que se realiza más adelante en el apartado de ``Análisis estadístico''. La primera forma de organización es en función de la distancia relativa entre butacas.Para este tipo de agrupación, los resultados son los mostrados en la tabla \ref{tablaTestButacas} que muestran los resultados de las 34 personas que participaron en el test produciendo un total de 850 respuestas. Esta separación se indica en la primera columna con intervalos de la forma $[x-y)$ donde se expresa el rango de distancias en las que se incluyen las respuestas. Por ejemplo, el intervalo $[2-3)$ incluye todas las respuestas en las que la distancia relativa entre butacas se encuentra entre los 2 metros (incluidos) y los 3 metros (sin incluir). 
	    
	    \begin{table}[H]
			\begin{center}
			\begin{scriptsize}
			\begin{tabular}{| c | c | c | c | c | c |}
			    \hline
				\textbf{Distancia [m]}&\textbf{Respuestas}&\textbf{Iguales}&\textbf{Diferentes}&\textbf{Duda iguales}&\textbf{Duda diferentes}\\ \hline
                (0-2)&61&40&21&11&9\\ \hline
                [2-3)&76&40&36&11&16\\ \hline
                [3-4)&102&47&55&14&14\\ \hline
                [4-5)&111&30&81&13&29\\ \hline
                [5-6)&95&24&71&9&21\\ \hline
                [6-7)&78&11&67&6&10\\ \hline
                [7-8)&70&9&61&2&4\\ \hline
                [8-9)&52&4&48&3&9\\ \hline
                [9-10)&62&3&59&1&8\\ \hline
                [10-11)&47&1&46&0&3\\ \hline
                [11-12)&33&1&32&0&2\\ \hline
                [12-14)&33&1&32&1&0\\ \hline
                [14-18)&24&0&24&0&1\\ \hline \hline
                \textbf{Total}&844&211&633&71&126\\ \hline
			\end{tabular}
			\caption{Resultados del test final en función de la distancia entre butacas.}
			\label{tablaTestButacas}
			\end{scriptsize}
			\end{center}	
		\end{table}
		
		La otra forma de organización de los datos es ordenandolos en función de la distancia relativa a la fuente sonora (considerando esta como un origen de coordenadas). En este caso, la agrupación sigue la misma nomenclatura que en el caso anterior, sólo que esta vez el intervalo $[x-y)$ se corresponde con las distancias promedio a cada pareja a la fuente, en vez de la distancia entre butacas. Haciendo esta organización, los datos se reparten de la forma que se muestra en la tabla \ref{tablaTestFuente}
		
		\begin{table}[H]
			\begin{center}
			\begin{scriptsize}
			\begin{tabular}{| c | c | c | c | c | c |}
			    \hline
				\textbf{Distancia [m]}&\textbf{Respuestas}&\textbf{Iguales}&\textbf{Diferentes}&\textbf{Duda iguales}&\textbf{Duda diferentes}\\ \hline
                [6-8)&15&5&10&2&2\\ \hline
                [8-10)&35&10&25&2&3\\ \hline
                [10-11)&32&8&24&2&7\\ \hline
                [11-12)&53&12&41&7&7\\ \hline
                [12-13)&55&14&41&3&5\\ \hline
                [13-14)&67&14&53&6&9\\ \hline
                [14-15)&101&22&79&10&10\\ \hline
                [15-16)&99&18&81&3&12\\ \hline
                [16-17)&84&18&66&4&12\\ \hline
                [17-18)&63&10&53&3&11\\ \hline
                [18-19)&95&21&74&3&15\\ \hline
                [19-20)&62&19&43&11&9\\ \hline
                [20-21)&44&19&25&9&10\\ \hline
                [21-24]&39&21&18&6&4\\ \hline \hline
                \textbf{Total}&844&211&633&71&126\\ \hline
			\end{tabular}
			\caption{Resultados del test final en función de la distancia relativa a la fuente.}
			\label{tablaTestFuente}
			\end{scriptsize}
			\end{center}	
		\end{table}
		
		Para el cálculo de ambas distancias, se ha generado un script en Python que extrae la información de la posición de recepción de cada fichero de audio que se encuentra en el nombre de dicho archivo y las distancias medidas presencialmente en el auditorio (toda esta información se comentó en el apartado de ``Toma de datos'').
		
		Tomando esta información se hace el cálculo de las distancias para todos los casos que se han producido durante la realización de los test y que han quedado recogidas en el fichero csv. En la figura \ref{fig:calcDist} se puede observar un ejemplo gráfico sobre cómo se han calculado dichas distancias y en el anexo C se puede consultar el código de Python en su totalidad.
		
		\begin{figure}[H]
                \begin{center}
                    \includegraphics[scale=.4]{../imagenes/calculoDistancia.png}
                    \caption{Ejemplo gráfico del cálculo de la distancia a la fuente.}
                    \label{fig:calcDist}
                \end{center}
            \end{figure}
		
		Para ambas distribuciones se han eliminado los resultados en los que la posición de las butacas es la misma ya que el análisis es diferente para la similitud que para la diferencia y, como se mostrará en el apartado de ``Análisis del test previo'', los participantes detectaron sin problemas cuando ambas señales son exactamente idénticas. Por este motivo, ya no se consideró necesario incluirlas.
		
\chapter{Análisis estadístico}    
    
    En este apartado, se muestran los diferentes análisis que se han realizado para los diferentes test; así como los procedimientos utilizados para su obtención.
        
    \section{Análisis del test previo}
        Para este test sólo se pretende comprobar si las personas eran capaces de distinguir entre audios en posiciones muy diferentes. Por este motivo, se decidió realizar un análisis más sencillo que en el caso del test final. Más concretamente, se optó por utilizar la tabla A1 del anexo A de la norma UNE-EN ISO 10399\cite{ISO10399} (ver Anexo B del proyecto), así como la ecuación \ref{eq:alpha} ya mostrada en el capítulo de ``Estado del arte''.
            
        Aplicando dichos recursos se obtienen las tablas \ref{tablaResultadosDuda} y \ref{tablaResultadosSinDuda}: en la primera, se incluyen todas las respuestas y, en la segundam únicamente las que se marcaron como ``seguras'' por los participantes.
            
        \begin{table}[H]
	    \begin{center}
		\begin{scriptsize}
		\begin{tabular}{| c | c | c | c || c |}
			\hline
			\textbf{Separación butacas}&\textbf{Respuestas}&\textbf{Iguales}&\textbf{Diferentes}&\textbf{$\alpha$-risk}\\ \hline
            Horizontal&27&9&18&0.1\\ \hline
            Vertical&10&1&9&0.01\\ \hline
            Horizontal y vertical&23
            &5&18&0.01\\ \hline
            Misma posición&10&9&1&NO\\ \hline
		\end{tabular}
		\caption{Resultados del test previo.}
		\label{tablaResultadosDuda}
		\end{scriptsize}
		\end{center}	
		\end{table}	
		
		\begin{table}[H]
			\begin{center}
			\begin{scriptsize}
			\begin{tabular}{| c | c | c | c || c |}
			    \hline
				\textbf{Separación butacas}&\textbf{Respuestas}&\textbf{Iguales}&\textbf{Diferentes}&\textbf{$\alpha$-risk}\\ \hline
                Horizontal&23&7&16&0.05\\ \hline
                Vertical&9&1&8&0.05\\ \hline
                Horizontal y vertical&20&4&16&0.01\\ \hline
                Misma posición&9&8&1&NO\\ \hline
			\end{tabular}
			\caption{Resultados del test previo considerando las respuestas marcadas como ``Sin duda''.}
			\label{tablaResultadosSinDuda}
			\end{scriptsize}
			\end{center}	
		\end{table}
		
		Como se puede observar de las tablas, existen diferencias estadísticamente significativas cuando se separan los puntos de escucha. Esto se puede afirmar ya que se obtienen probabilidades bastante bajas de que las diferencias sean debidas al azar (entre 10\% y un 1\%). Dicho efecto se sigue manteniendo cuando se suprimen las respuestas marcadas como ``No seguras'', incluso se ve reforzado (el porcentaje de posibilidad de error se sitúa entre el 5\% y el1\%). Por este motivo, se considera probada nuestra primera hipótesis y se decidió continuar con el estudio. Para la situación de ``Misma posición'', se hano se alcanza el mínimo de respuestas correctas para alcanzar alguno de los valores de $\alpha$. Por este motivo, se representa en la tabla con un valor ``NO''. De esta forma, se marca que los participantes no han podido identificarlos como eventos distintos salvo por azar, lo cual era algo esperable puesto que se tratan de audios idénticos.
		
		Los valores concretos obtenidos para $\alpha$-risk no son realmente importantes, puesto que el número de datos es relativamente pequeño y la repetición de dicho experimentos podría producir que dichos valores fluctuaran. Para este caso, el experimento lo hemos considerado como válido al obtener valores inferiores o iguales a 0.1, que se corresponden a probabilidades de error menores o iguales al 10\%. (excepto para las respuestas en la misma posición).
		
    \section{Análisis del test final}
        Para el análisis del test final, se realizan dos análisis de datos diferentes para cada una de las dos formas de agrupación de los datos (distancia relativa a la fuente y distancia relativa entre las dos  butacas). El primer análisis se hará mediante el seguimiento de la norma UNE-EN ISO 10399 que ya se utilizó en el test previo, mediante la tabla A.1 de la norma y la ecuación \ref{eq:alpha}.
        
        Por otro lado, se realiza un análisis mediante modelos thurstonianos. Para ello, se genera un script en el lenguaje de programación estadística ``R'' para utilizar el paquete ``SensR''\footnote{https://cran.r-project.org/web/packages/sensR/index.html}. Este paquete permite obtener de forma sencilla los valores del coeficiente $d'$, así como su desviación estandar. En el anexo E puede consultarse el script.
        
        
		\subsection{Distancia relativa a la fuente}
		    Aplicando el mismo procedimiento explicado en apartado de ``Conceptos teóricos'', se obtienen los resultados para el análisis mediante la norma UNE-EN ISO 10399 para todas las respuestas y sólo para las marcadas como ``Nivel de seguridad: mucha'', respectivamente. También se utiliza el lenguaje de programación R y el paquete antes mencionado y así obtener el coeficiente $d'$ y su desviación estándar ($\sigma$). Todos los valores de ambos análisis están representados en las tablas \ref{tablaFuenteDuda} y \ref{tablaFuenteSinDuda}. Para la parte del análisis mediante modelos thurstonianos se generan además las figuras \ref{fig:ThurstFuenteDuda} y \ref{fig:ThurstFuenteSinDuda} que muestran de forma gráfica la variación de dicho coeficiente. El procedimiento para obtener el análisis con ``R'' es el siguiente: en primer lugar se introduce el núemero total de respuestas, así como el número de respuestas ``correctas'' (entendiéndose como correctas aquellas respuestas en las que los participantes hayan identificado los estímulos como diferentes). A continuación, hay que indicar a la función qué tipo de test se ha realizado para que el cálculo de los coeficientes sea el correcto. En nuestro caso, el test se corresponde con un ``\textit{2-AFC}'', ya que el participante estaba forzado a seleccionar una de las dos opciones sin posibilidad de escoger una opción intermedia.
		    
		    \begin{table}[H]
			\begin{center}
			\begin{scriptsize}
			\begin{tabular}{| c | c | c | c || c | c | c |}
			    \hline
				\textbf{Distancia [m]}&\textbf{Respuestas}&\textbf{Iguales}&\textbf{Diferentes}&\textbf{$\alpha$-Risk}&\textbf{$d'$}&\textbf{$\sigma (d')$}\\ \hline
                [6-8)&15&5&10&0.2&0.609&0.473\\ \hline
                [8-10)&35&10&25&0.05&0.800&0.318\\ \hline
                [10-11)&32&8&24&0.01&0.954&0.341\\ \hline
                [11-12)&53&12&41&0.001&1.062&0.270\\ \hline
                [12-13)&55&14&41&0.001&0.934&0.259\\ \hline
                [13-14)&67&14&53&0.001&1.146&0.244\\ \hline
                [14-15)&101&22&79&0.001&1.102&0.197\\ \hline
                [15-16)&99&18&81&0.001&1.285&0.208\\ \hline
                [16-17)&84&18&66&0.001&1.120&0.217\\ \hline
                [17-18)&63&10&53&0.001&1.414&0.269\\ \hline
                [18-19)&95&21&74&0.001&1.087&0.203\\ \hline
                [19-20)&62&19&43&0.01&0.715&0.236\\ \hline
                [20-21)&44&19&25&NO&0.243&0.269\\ \hline
                [21-24]&39&21&18&NO&0.000&NA\\ \hline
			\end{tabular}
			\caption{Valores de $\alpha$-risk, $d'$ y su desviación estandar para los resultados totales agrupados en función de la distancia a la fuente según la norma UNE-EN ISO 10399 y aplicando modelos thurstonianos, respectivamente.}
			\label{tablaFuenteDuda}
			\end{scriptsize}
			\end{center}	
		    \end{table}
		    
		    \begin{table}[H]
			\begin{center}
			\begin{scriptsize}
			\begin{tabular}{| c | c | c | c || c | c | c |}
			    \hline
				\textbf{Distancia [m]}&\textbf{Respuestas}&\textbf{Iguales}&\textbf{Diferentes}&\textbf{$\alpha$-risk}&\textbf{$d'$}&\textbf{$\sigma (d')$}\\ \hline
                [6-8)&11&3&8&0.2&0.855&0.571\\ \hline
                [8-10)&30&8&22&0.01&0.881&0.347\\ \hline
                [10-11)&23&6&17&0.05&0.906&0.399\\ \hline
                [11-12)&39&5&34&0.001&1.605&0.361\\ \hline
                [12-13)&47&11&36&0.001&1.026&0.285\\ \hline
                [13-14)&52&8&44&0.001&1.443&0.298\\ \hline
                [14-15)&81&12&69&0.001&1.477&0.241\\ \hline
                [15-16)&84&15&69&0.001&1.302&0.226\\ \hline
                [16-17)&68&12&56&0.001&1.314&0.252\\ \hline
                [17-18)&49&7&42&0.001&1.510&0.313\\ \hline
                [18-19)&77&18&59&0.001&1.027&0.223\\ \hline
                [19-20)&42&8&34&0.001&1.239&0.315\\ \hline
                [20-21)&25&10&15&No&0.358&0.359\\ \hline
                [21-24]&29&15&14&No&0&NA\\ \hline
			\end{tabular}
			\caption{Valores de $\alpha$-risk, $d'$ y su desviación estandar  para los resultados marcados como ``Sin duda'' agrupados en función de la distancia a la fuente según la norma UNE-EN ISO 10399 y aplicando modelos thurstonianos, respectivamente.}
			\label{tablaFuenteSinDuda}
			\end{scriptsize}
			\end{center}	
		    \end{table}
		    
		    
		    
		    \begin{figure}[H]
                \includegraphics[scale=0.7]{../imagenes/analisisThurstFuenteDuda.png}
			    \centering
			    \caption{Coeficientes $d'$ y su desviación estándar en función de la distancia a la fuente.} 
			    \label{fig:ThurstFuenteDuda}
            \end{figure}
            
            \begin{figure}[H]
                \includegraphics[scale=0.7]{../imagenes/analisisThurstFuenteSinDuda.png}
			    \centering
			    \caption{Coeficientes $d'$ y su desviación estándar en función de la distancia a la fuente cuando las respuestas están marcadas como ``Sin duda''.} 
			    \label{fig:ThurstFuenteSinDuda}
            \end{figure}

            Como se puede observar en las tablas y figuras antes mencionadas, el comportamiento para ambos procedimientos es bastante similar. Hay que tener en cuenta que para $\alpha$-risk, cuanto menor es el valor, estadísticamente más representativa es la diferencia, mientras que para los modelos thurstonianos, se produce a la inversa; cuanto mayor es el valor de $d'$, más diferentes son los estímulos.
            
            En ambos casos, se observa que las diferencias aumentan conforme aumenta la distancia respecto de la fuente, pero llegada una determinada distancia, las diferencias comienzan a dejar de ser perceptibles hasta llegar a posiciones donde las posibles diferencias pueden considerarse debidas al azar.
            
            La mayor diferencia entre ambos sistemas se encuentra en lo que podríamos denominar como ``sensibilidad''. La norma UNE-EN ISO 10399 es menos sensible a las variaciones puntuales del número de respuestas consideradas correctas; además de que los cambios que se producen son mucho más abruptos ya que sólo existen seis posibles valores para $\alpha$-risk (0.2, 0.1, 0.05, 0.01, 0.001 y que no pueda obtenerse el valor).
            
            Los modelos thurstonianos, por otra parte, son mucho más sensibles, ya que ligeras variaciones en los resultados, producen variaciones en los valores de $d'$. Al mismo tiempo, se dispone de todo un rango contínuo de valores que dicho coeficiente puede obtener. Esto permite que las representaciones gráficas de las variaciones de $d'$ muestren mejor el comportamiento psicoacústico de los estímulos y analizar si existe una tendencia, como es el caso de nuestro experimento.
            
            Un ejemplo de esto puede observarse en los casos de la tabla \ref{tablaFuenteDuda}. En esta tabla, se ve como para el caso del análisis mediante la norma, el descenso se produce a partir de la distancia de [19-20) metros, mientras que en los modelos thurstonianos, el decaimiento se produce antes, a partir de la distancia [18-19) metros. Lo mismo ocurre con el caso de la tabla \ref{tablaFuenteSinDuda}, donde los decaimientos vuelven a producirse en las mismas distancias. Para tener una idea de este comportamiento sobre el espacio del auditorio, se presenta la figura \ref{fig:dprimeauditorio}, donde se representa el comportamiento de $d'$ (la percepción de diferencia entre parejas de estímulos) en función de dicha distancia en la fuente, pero representado sobre el plano del auditorio directamente mediante un diagrama de colores en el que las zonas verdes representan valores bajos de dichas diferencias perceptuales y las zonas rojas representan valores elevados. Las zonas sin pintar corresponden a zonas donde o bien no se han tomado medidas, o bien el valor de las diferencias perceptuales ($d'$) es nulo. También se ha considerado que el comportamiento de la sala es simétrico respecto al eje longitudinal de la sala y que, para la zona delantera donde no hay butacas, se ha realizado una extrapolación de los datos.
            
            \begin{figure}[H]
                \includegraphics[scale=0.9]{../imagenes/auditoriodprime.png}
			    \centering
			    \caption{Coeficientes $d'$ en función de la distancia a la fuente representado sobre el plano del auditorio. Los valores verdes representan un valor bajo de $d'$; mientras que los rojos representan valores elevados.} 
			    \label{fig:dprimeauditorio}
            \end{figure}
        
        \subsection{Distancia relativa entre butacas}
            Para este apartado se vuelven a aplicar tanto los procedimientos de la norma UNE-EN ISO 10399, como los modelos thusthonianos. De ambos análisis se obtienen las tablas \ref{tablaButacasDuda} y \ref{tablaButacasSinDuda}, así como las figuras \ref{fig:ThurstButacasDuda} y \ref{fig:ThurstButacasSinDuda}. De nuevo, para los modelos thurstonianos, se ha determinado que el protocolo es ``\textit{2-AFC}'' para que la función de R pueda realizar de forma correcta la estimación del valor $d'$.
            
            Como se puede observar en las tablas recien mencionadas, los participantes empiezan a notar diferencias cuando la distancia entre butacas se encuentra a partir de los [3-4) metros, aunque este efecto se retrasa hasta los [4-5) metros si suprimimos las respuestas marcadas como ``Con duda''. Una vez ahí, las diferencias se mantienen estadísticamente representativas con el máximo nivel de seguridad que ofrece la norma, ya que la probabilidad de que los eventos sean iguales cuando, en realidad, se han percibido como diferentes, es de 0.001. En los resultados obtenidos no hay un momento en el que dicha probabilidad decrezca.
            
            \begin{table}[H]
			\begin{center}
			\begin{scriptsize}
			\begin{tabular}{| c | c | c | c || c | c | c |}
			    \hline
				\textbf{Distancia [m]}&\textbf{Respuestas}&\textbf{Total iguales}&\textbf{Total diferentes}&\textbf{$\alpha$-risk}&\textbf{$d'$}&\textbf{$\sigma (d')$}\\ \hline
                (0-2)&61&40&21&NO&0.000&NA\\ \hline
                [2-3)&76&40&36&NO&0.000&NA\\ \hline
                [3-4)&102&47&55&0.05&0.139&0.176\\ \hline
                [4-5)&111&30&81&0.001&0.865&0.18\\ \hline0
                [5-6)&95&24&71&0.001&0.942&0.197\\ \hline
                [6-7)&78&11&67&0.001&1.521&0.249\\ \hline
                [7-8)&70&9&61&0.001&1.603&0.270\\ \hline
                [8-9)&52&4&48&0.001&2.017&0.362\\ \hline
                [9-10)&62&3&59&0.001&2.349&0.384\\ \hline
                [10-11)&47&1&46&0.001&2.868&0.583\\ \hline
                [11-12)&33&1&32&0.001&2.654&0.615\\ \hline
                [12-14)&33&1&32&0.001&2.654&0.615\\ \hline
                [14-18)&24&0&24&0.001&$\infty$&NA\\ \hline
			\end{tabular}
			\caption{Valores de $\alpha$-risk, $d'$ y su desviación estándar para los resultados totales agrupados en función de la distancia entre butacas según la norma UNE-EN ISO 10399 y aplicando modelos thurstonianos, respectivamente.}
			\label{tablaButacasDuda}
			\end{scriptsize}
			\end{center}	
		    \end{table}
		    
		    \begin{table}[H]
			\begin{center}
			\begin{scriptsize}
			\begin{tabular}{| c | c | c | c || c | c | c |}
			    \hline
				\textbf{Distancia [m]}&\textbf{Respuestas}&\textbf{Total iguales}&\textbf{Total diferentes}&\textbf{$\alpha$-risk}&\textbf{$d'$}&\textbf{$\sigma (d')$}\\ \hline
                (0-2)&41&27&14&NO&0.000&NA\\ \hline
                [2-3)&49&29&20&NO&0.000&NA\\ \hline
                [3-4)&74&33&41&NO&0.192&0.207\\ \hline
                [4-5)&79&17&62&0.001&1.115&0.224\\ \hline
                [5-6)&65&15&50&0.001&1.041&0.243\\ \hline
                [6-7)&62&5&57&0.001&1.981&0.327\\ \hline
                [7-8)&64&7&57&0.001&1.739&0.295\\ \hline
                [8-9)&40&1&39&0.001&2.772&0.597\\ \hline
                [9-10)&53&2&51&0.001&2.514&0.450\\ \hline
                [10-11)&44&1&43&0.001&2.829&0.589\\ \hline
                [11-12)&31&1&30&0.001&2.614&0.621\\ \hline
                [12-14)&32&0&32&0.001&$\infty$&NA\\ \hline
                [14-18)&23&0&23&0.001&$\infty$&NA\\ \hline
			\end{tabular}
			\caption{Valores de $\alpha$-risk, $d'$ y se desviación estándar para los resultados agrupados en función de la distancia entre butacas según la norma UNE-EN ISO 10399 y aplicando modelos thurstonianos, respectivamente, cuando los resultados son marcados como ``Sin duda''.}
			\label{tablaButacasSinDuda}
			\end{scriptsize}
			\end{center}	
		    \end{table}
            
            
            Por otro lado, la tabla \ref{tablaButacasDuda}, así como la figura \ref{fig:ThurstButacasDuda}, muestran, para el análisis mediante modelos thurstonianos, un crecimiento más o menos constante del coeficiente $d'$ desde los [3-4) metros hasta el final. En este caso, en el que están incluidos los resultados marcados como ``Con duda'', puede parecer que existe una bajada de la percepción de las diferencias, pero al observar los datos de la tabla, se ve que ese decaimiento se debe a que se tienen menos datos en dichas posiciones y, por tanto, el hecho de que una persona determine que ha percibido los audios como iguales, hace que el valor del coeficiente baje. Esto se comprueba observando la tabla \ref{tablaButacasSinDuda} y figura \ref{fig:ThurstButacasSinDuda} donde se han eliminado las respuestas dudosas y se observa un reforzamiento en la tendencia creciente en las distancias más lejanas.
		    
		    \begin{figure}
                \includegraphics[scale=0.7]{../imagenes/analisisThurstButacasDuda.png}
			    \centering
			    \caption{Coeficientes $d'$ y su desviación estándar en función de la distancia entre butacas.} 
			    \label{fig:ThurstButacasDuda}
            \end{figure}
            
            \begin{figure}
                \includegraphics[scale=0.7]{../imagenes/analisisThurstButacasSinDuda.png}
			    \centering
			    \caption{Coeficientes $d'$ y su desviación estándar en función de la distancia entre butacas cuando las respuestas están marcadas como ``Sin duda''.} 
			    \label{fig:ThurstButacasSinDuda}
            \end{figure}
            
            Al igual que ocurría con los resultados agrupados según la distancia a la fuente, se observa que, a pesar, de que el comportamiento es similar para ambas formas de analizar los datos, la norma UNE-EN ISO 10399\cite{ISO10399} es mucho menos sensible a los pequeños cambios en los resultados; mientras que los modelos thurstonianos ofrecen un rango mayor de datos que fluctúan más facilmente. Esto no es necesariamente peor; depende de lo que se pretenda conseguir con el estudio. Si se pretende determinar a partir de qué distancias las diferencias son significativas, la norma da información más útil en este sentido. Por otro lado, si se pretende estudiar el comportamiento de las diferencias perceptuales en función de la distancia, aquí los modelos thurstonianos tienen una gran ventaja frente a la norma.
            
\chapter{Conclusiones}
    En este proyecto se ha estudiado la percepción de las respuestas al impulso de una sala en diferentes posiciones de escucha. Para ello, se ha realizado un análisis del estado del arte de los test subjetivos relacionados con la percpeción auditva. De esta forma, se han fijado las características y requerimientos necesarios para la correcta toma de datos y la realización de nuestro test perceptual. De forma paralela, se ha diseñado una aplicación escrita en Python para la interacción por parte de los participantes y el análisis de los datos (esta última parte junto al desarrollo de un script en lenguaje de programación R). Por último, se ha realizado el análisis de los resultados categorizándolos en función de dos tipos de distribuciones.
    
    \section{Satisfacción de los objetivos secundarios}
    
        Con todo este trabajo, se ha tratado de cumplir una serie de objetivos secundarios:
        
        \subsection*{Selección de una sala adecuada donde realizar las medidas}    
            El primero de ellos consistía en la selección de una sala adecuada donde realizar las medidas. Esta sala debía tener un tamaño lo suficientemente grande para que pudieran existir diferencias perceptuales importantes. Así mismo, la sala debía disponer de unas propiedades acústicas invariantes y que no dependieran del exterior. Esto se ha conseguido utilizando como espacio para las medidas, el auditorio de la ETSIST de la UPM. Este espacio cuenta con un aforo superior a las 300 butacas y su acústica está pensada para música coral con un aislamiento acústico suficiente para minimizar los efectos del ruido exterior. Con esto, los mayores desafíos se han visto reflejados en el control de las fuentes de ruido interno como zumbidos eléctricos o el efecto de los sistemas de climatización. Todas estas medidas han limitado el posible efecto de variables externas en las escuchas de los participantes.
        
        \subsection*{Estudio del número de respuestas necesarias}
            Otro de los objetivos era el de estudiar el número de respuestas al impulso necesarias para el proyecto tanto para posiciones distitnas como en posiciones repetidas.A la vista de los resultados y el análisis realizado, se puede concluir que la percepción auditiva de las respuestas al impulso de un determinado espacio dependen, en gran medida, de la distancia a la fuente sonora. De esta forma, se puede ver como las diferencias son más perceptibles a medida que las respuestas se alejan de la fuente hasta que, en las posiciones más alejadas, dichas diferencias dejan de percibirse con una pendiente bastante abrupta. También se puede concluir que la distancia absoluta entre las posiciones de escucha tiene, a su vez, un rol muy importante a la hora de encontrar diferencias perceptuales. En este caso, se ha observado que las diferencias se encuentran más fácilmente a medida que aumenta dicha distancia. A priori, parece que esa percepción de diferencias se estabiliza llegado un determinado punto, pero no se tienen datos suficientes para afirmarlo. Debido a esto, se puede afirmar que la mejor forma de realizar los tomas de datos no es uniforme, sino que en las posiciones más cercanas y alejadas, la densidad de los datos puede ser menor, ya que la capacidad de los participantes para distinguir las señales es bastante menor que en el centro. En esta parte central, la habilidad para distinguir los estímulos es mayor, por lo que es necesario tener un mayor volumen de datos. 
    
            Las mayores dificultades que se han encontrado para la consecución de este objetivo es la imposibilidad de establecer unos criterios específicos para el número de respuestas. Aquí entran en juego variables como las dimensiones de la sala, la reverberación o la disposición del patio de butacas. Por este motivo, se antoja necesaria la realización de un análisis específico según el tipo de espacio sobre el que vayan a tomarse las medidas. También hay que tener en cuenta que la toma de datos es una de las partes de los proyectos más propensa a errores (voluntarios o involuntarios), lo que puede llevar a repeticiones y demoras en la realización del proyecto.
        
        \subsection*{Postprocesado de las señales para test perceptuales}
            Para el objetivo del postprocesado de las señales para su utilización en test psicoacústicos, se ha visto que no hay un sistema único para llevarse a cabo. El procedimiento varía en función del tipo de señales que se vayan a utilizar. Por ejemplo, para el caso de respuestas impulsivas, la utilización del software ``Dirac'' se ha mostrado como la opción más cómoda ya que, además de permitir la grabación de las señales, puede confirgurarse para realizar el procesado de forma automática. Si se optara por otro tipo de señales, como las respuestas convolucionadas con audios anecoicos, hay que acudir a otras herramientas que, preferiblemente, permitan la automatización del proceso para grupos de señales.
        
        \subsection*{Diseño de los test más adecuados para lograr el objetivo principal}
            El último subojetivo era el de diseñar los test más adecuados para lograr el objetivo principal. Esto se ha conseguido, en gran medida, gracias a la utilización de un test sencillo de utilizar para los participantes, lo que nos ha permitido contar con personas sin experiencia previa. El \textit{handicap} de la experiencia se ha suplido, en parte, con la realización de un entrenamiento previo por parte de todos los participantes, independientemente de sus conocimientos previos. Gracias a esto, se ha formado a las personas para que supieran responder a las cuestiones que se les preguntaba y obtener así el mayor volumen de datos posible. 
    
            Atendiendo a los procedimientos de análisis, se ha visto la utilidad de los modelos thurstonianos frente a otros tipos de análisis más tradicionales como la norma UNE-EN ISO 10399 \cite{ISO10399}. Estos modelos permiten comparar el nivel de diferencia entre diferentes grupos de datos (incluso siendo completamente independientes). Esto permite realizar análisis estadísticamente más complejos e interesantes que las posibilidades que brindan otros sistemas. 
    
            No obstante, al final, la utilidad del tipo de análisis depende de lo que se pretenda estudiar. De esta forma, si se pretende estudiar sólamente si unas respuestas son estadísticamente representativas, la norma UNE-EN ISO 10399 es un sistema rápido y sencillo para dar respuesta a esta cuestión. Por otro lado, si se pretende estudiar la variación de las diferencias percibidas a lo largo de un espacio, los modelos thurstonianos cuentan con herramientas que pueden ser más útiles. 
    
            Dentro de este objetivo, los mayores desafios se han encontrado a la hora de escoger el tipo de test adecuado. Existen diferentes opciones que permiten obtener un tipo de resultados u otros. Esto puede repercutir en un incremento de la dificultad del test o añadir un nivel adicional de detalle en los resultados. Es importante realizar un análisis previo sobre qué se pretende estudiar para seleccionar el modelo que mejor satisface sus necesidades. Esto incluye también al tipo de análisis de datos que, como se ha expresado antes, tiene una gran repercusión, no sólo en la complejidad y el tiempo de cálculo, sino en la calidad de los resultados obtenidos.
    
    \section{Satisfacción del objetivo principal}
        El objetivo principal del proyecto era el de comprobar las diferencias perceptuales entre distintas respuestas al impulso tomadas en distintas posiciones de una sala. A raiz de lo comentado en el apartado anterior, se puede concluir que las diferencias percibidas dependen en gran medida de la distancia entre ambas posiciones. De esta forma, las diferencias se detectan más fácilmente a medida que la distancia aumenta.
        
        Por otro lado, el efecto de la distancia también es relevante si se analiza la distancia de las posiciones de escucha respecto de la fuente sonora. Aquí se observa que la sensibilidad de los participantes para detectar las diferencias aumenta a medida que se alejan de la fuente hasta que, llegado un punto en la parte más lejana de la sala, la sensibilidad disminuye de forma rápida.
        
        Con todo esto, se puede decir que el objetivo marcado en el proyecto se ha cumplido de forma satisfactoria. Del mismo modo, se ha cumplido la hipótesis inicial de que la percepción de las respuestas al impulso varían en función de la distancia.
    
    \section{Mejoras y nuevas líneas de investigación}
    
        Si se empezara ahora el proyecto, habría algunos elementos que se hubieran realizado de forma distinta.
        
        Por un lado, se hubiera modificado la distribución de las posiciones en la toma de datos. En vez de seleccionar los datos de una de cada tres butacas con un desplazamiento de una butaca entre filas, se hubiera hecho una distribución dependiente de la distancia a la fuente sonora. De esta forma, se hubieran tomado menos datos en las posiciones más cercanas y el grueso de las medidas se encontrarían en las zonas medias y lejanas. También, de haber disponido del tiempo y material necesario, se hubiera repetido los test y análisis utilizando otro tipo de señales para comprobar si los resultados varían en función de los audios que se aplican.
    
        Otras posibles futuras líneas de investigación incluyen el análisis de otros aspectos perceptuales más concretos como la coloración de las señales, la percepción de reverberación o comprobar de qué forma las variaciones del nivel de percepción son modificadas en función de diferentes tipos de salas (de tamaños diferentes, variando el acondicionamiento acústico, etc.)
        
    \section{Posibles aplicaciones del proyecto}
        Los resultados de este proyecto pueden resultar de utilidad dentro del campo de las mediciones acústicas. De esta manera, si se van a realizar mediciones relacionadas con la percepción auditiva, podrían ahorrarse medidas en las posiciones más cercanas a la fuente sonora y concentrarlas en puntos más sensibles como la zona central.
        
        También existen aplicaciones para la recreación virtual de espacios acústicos. Aquí podría simplificarse la carga de trabajo al poder centrar los esfuerzos en las áreas donde la sensibilidad de los oyentes es mayor.
        
        Para ambos casos, no obstante, sería necesario realizar un análisis previo de las características de los espacios para limitar correctamente las diferentes zonas de percepción auditiva. 

\appendix

\chapter{Presupuesto}
En este anexo se realiza el cálculo del presupuesto que ha sido necesario para la realización del proyecto.

    \section*{Equipamiento utilizado:}

    \begin{itemize}
        \item Ordenador portátil ASUS 2 con Sistema operativo ``Windows 10'' proporcionado por la Universidad Politécnica de Madrid.
	    \item Tarjeta de sonido MOTU UltraLite-mk3 Hybrid.
	    \item Amplificador de potencia Crown XLS 2002.
	    \item Fuente dodecaédrica AVM DO-12.
	    \item Micrófono ominidireccional AKG-CK92.
	    \item Micrófono bidireccional AKG-CK94.
	    \item Alargaderas y cableado necesario para el conexionado de los equipos.
	    \item trípodes y sujecciones para los micrófonos.
	    \item Micrófonos biaurales Roland CS-10EM.
	    \item Grabadora portátil Yamaha pr7.
        \item Ordenador portátil con Sistema operativo ``Debian''. 12GB de memoria RAM. Procesador ``Intel i5''.
	    \item Auriculares ``Tascam TH-02''. 
	    \item Software \textit{Dirac}. Licencia proporcionada por la Universidad Politécnica de Madrid.
	    \item Software \textit{Audacity}. Editor de audio de código abierto.
	    \item Software \textit{LibreOffice Calc}. Programa de hojas de cálculo de código abierto.
		\item Software \textit{Setzer}. Editor de lenguaje LaTeX de código abierto.
		\item Servicio de suscripción a bases de datos bibliográficas y de normativas. Suscripción proporcionada por la Universidad Politécnica de Madrid.
    \end{itemize}
    \section*{Horas de trabajo y personal utilizado:}
        El proyecto comenzó a realizarse a mediados de 2021 y ha requerido de una dedicación parcial hasta finales del año 2021. La mayoría del tiempo invertido ha sido de forma individual (con el apoyo del tutor académico). No obstante, han existido momentos en los que se ha precisado de personal adicional. Concretamente, estos momentos han sido las tomas de datos donde, durante un total de tres días, se ha requerido de la presencia de tres personas en el auditorio (es decir, dos personales adicionales). Estas personas se trataban de un profesor de la Universidad Politécnica de Madrid y una estudiante de doctorado.
        
        Para la realización de los test subjetivos de audio se invirtieron un total de 20 días en jornadas de 8 horas. Los participantes del test fueron voluntarios, por lo que no suponen de un coste adicional dentro del proyecto. Durante la realización de los test había una persona encargada de solucionar cualquier problema que pudiera aparecer, resolver dudas a los participantes y encargarse de las tareas de limpieza y desinfección del material.
        
        Por último, el análisis de los resultados de los dos test ha requerido del trabajo de una persona en durante 14 días. También ha sido necesaria alrededor de una 7 días para la revisión de los contenidos y la corrección de errores.
        
        A lo largo de todo el proyecto se han realizado varias reuniones con el tutor académico para hacer revisiones, buscar soluciones para problemas que iban apareciendo, etc.
        
    \section*{Costes del proyecto:}
        En la tabla \ref{table:tablaPresupuesto} se han representado los costes relacionados con la realización del proyecto. Los valores reflejados de muchos de los productos son estimaciones de su valor real. Lo mismo ocurre con la cuantía de las horas dedicadas y el valor por unidad de diferentes equipos. Todos los costes tienen incluidos sus impuestos asociados.
        
        \begin{table}[H]
			\begin{center}
			\begin{scriptsize}
			\begin{tabular}{| c || c | c | c | c | c |}
				\hline
				\textbf{Elemento}&\textbf{Coste/ud [\euro]}&\textbf{Unidades}&\textbf{Coste Total [\euro]}\\ \hline
                Portátil, 12 GB RAM, Intel i5&800.00&2&1600.00\\ \hline
                Licencia Windows 10&259.00&1&259.00\\ \hline
                Licencia Debian&0.00&1&0.00\\ \hline
                Licencicia software código abierto&0.00&3&0.00\\ \hline
                Licencia software Dirac&450.00&1&450.00\\ \hline
                Auriculares TASCAM TH-02&30.00&1&30.00\\ \hline
                Amplificador potencia Crown XLS 2002&585.00&1&585.00\\ \hline
                Micrófono ominidireccional AKG-CK92&180.00&1&180.00\\ \hline
                Micrófono bidireccional AKG-CK94&440.85&1&440.85\\ \hline
                Tarjeta de sonido MOTU UltraLite-mk3 Hybrid&410.00&1&410.00\\ \hline
                Fuente dodecaédrica AVM DO-12&&1&0.00\\ \hline
                Alargaderas y cableado para conexionado de equipos&51.43&2&102.86\\ \hline
                Trípodes y sujecciones para los micrófonos&35.00&1&35.00\\ \hline
                Micrófonos biaurales Roland CS-10EM&99.99&1&99.99\\ \hline
                Grabadora portátil Yamaha pr7&663.35&1&663.35\\ \hline
                Suscripción a artículos o norma&27.00&50&1350.00\\ \hline
                Hora de trabajo Ing. Junior&9.75&268&2613.00\\ \hline
                Hora de trabajo Ing. Senior x2&12.50&90&1125.00\\ \hline
                Alquiler auditorio ETSIST UPM&1863.79&3&5591.37\\ \hline
                Alquiler laboratorio CITSEM&&1&0.00\\ \hline
                -&-&\textbf{Total [\euro]}&\textbf{15535.42}\\ \hline
        
            \end{tabular}
			\caption{Tabla de costes del proyecto.}
			\label{table:tablaPresupuesto}
			\end{scriptsize}
			\end{center}	
		\end{table}

\chapter{Tabla A.1 de la norma UNE-EN ISO 10399}

        \begin{table}[H]
			\begin{center}
			\begin{scriptsize}
			\begin{tabular}{| c || c | c | c | c | c |}
				\hline
				&\multicolumn{5}{c |}{\textbf{Alpha-Risk ($\alpha$)}} \\\hline
				\textit{n}&0.2&0.1&0.05&0.01&0.001\\ \hline
				6&5&6&6&-&-\\ \hline
				7&6&6&7&7&-\\ \hline
				8&6&7&7&8&-\\ \hline
				9&7&7&8&9&-\\ \hline
				10&7&8&9&10&10\\ \hline
				11&8&9&9&10&11\\ \hline
				12&8&9&10&11&12\\ \hline
				13&9&10&10&12&13\\ \hline
				14&10&10&11&12&13\\ \hline
				15&10&11&12&13&14\\ \hline
				16&11&12&12&14&15\\ \hline
				17&11&12&13&14&16\\ \hline
				18&12&13&13&15&16\\ \hline
				19&12&13&13&15&17\\ \hline
				20&13&14&15&16&18\\ \hline
				21&13&14&15&17&18\\ \hline
				22&13&14&15&17&19\\ \hline
				23&15&16&16&18&20\\ \hline
				24&15&16&17&19&20\\ \hline
				25&16&17&18&19&21\\ \hline
				26&16&17&18&20&22\\ \hline
                27&17&18&19&20&22\\ \hline
                28&17&18&19&21&23\\ \hline
                29&18&19&20&22&24\\ \hline
                30&18&20&20&22&24\\ \hline
                32&19&21&22&24&26\\ \hline
                36&22&23&24&26&28\\ \hline
                40&24&25&26&28&31\\ \hline
                44&26&27&28&31&33\\ \hline
                48&28&29&31&33&36\\ \hline
                52&30&32&33&35&38\\ \hline
                56&32&34&35&38&40\\ \hline
                60&34&36&37&40&43\\ \hline
                64&36&38&40&42&45\\ \hline
                68&38&40&42&45&48\\ \hline
                72&41&42&44&47&50\\ \hline
                76&43&45&46&49&52\\ \hline
                80&45&47&48&51&55\\ \hline
                84&47&49&51&54&57\\ \hline
                88&49&51&53&56&59\\ \hline
			\end{tabular}
			\caption{Valores de \textit{Alpha-Risk} en función del número de personas y el número de respuestas positivas según la norma UNE-EN ISO 10399.}
			\label{tablaAlpha}
			\end{scriptsize}
			\end{center}	
		\end{table}
		
\chapter{Código Python}
\section{Código Python del test perceptual}
\begin{verbatim}

import numpy as np
import pandas as pd
import os
import random
from playsound import playsound
import gi
gi.require_version('Gtk', '3.0')
from gi.repository import Gtk as gtk
from gi.repository import Gdk

## DEFINICIÓN DE FUNCIONES

# Función que devuelve una lista con los archivos dentro de un directorio
def listaArchivos(ruta):
    archivos = os.listdir(ruta)
    return archivos

# Función que reproduce un archivo aleatorio dentro de los existentes dentro 
# de un directorio que se le pasa como parámetro
def reproduceAudio(ruta, archivos, indice):
    print('Estamos reproduciendo: '+archivos[indice])
    playsound(ruta+'/'+archivos[indice])


## Variables Globales
rutaTutorial='/home/vic/Documents/universidad/PFM/Audios/tutorial'
miruta='/home/vic/Documents/universidad/PFM/Audios/IRsTest'
misArchivosTutorial=listaArchivos(rutaTutorial)
misArchivos=listaArchivos(miruta)
respuestas=[None]
opcion=0
i=0
maximo=24
rutacsv='/home/vic/Documents/universidad/PFM/Resultados/test2.csv'
df=pd.DataFrame(
    {
        'archivo1': np.array(respuestas),
        'archivo2': np.array(respuestas),
        'respuesta': np.array(respuestas),
        'seguridad': np.array(respuestas),
    }
    )

miIndice1=random.randint(0,len(misArchivos)-1)
miIndice2=random.randint(0,len(misArchivos)-1)
df.iloc[0,0]=misArchivos[miIndice1]
df.iloc[0,1]=misArchivos[miIndice2]

iguales=False
seguridad=False

class Main:
    def __init__(self):
        gladeFile="appTFMGUITuto.glade"
        self.builder=gtk.Builder()
        self.builder.add_from_file(gladeFile)


        #Instanciar Ventana y botones tutorial
        wTutoWindow=self.builder.get_object("tutorialWindow")
        bSame1=self.builder.get_object("buttonSame1")
        bSame2=self.builder.get_object("buttonSame2")
        bDiferente1=self.builder.get_object("buttonDiferente1")
        bDiferente2=self.builder.get_object("buttonDiferente2")
        bFila1=self.builder.get_object("buttonFila1")
        bFila2=self.builder.get_object("buttonFila2")
        bStartTest=self.builder.get_object("buttonEmpezar")
        
        
        #Instancias Ventana y objetos ventana principal
        #wMainWindow=self.builder.get_object("mainWindow")
        bPlayAudio1=self.builder.get_object("PlayAudio1")
        bPlayAudio2=self.builder.get_object("PlayAudio2")
        sRespuesta=self.builder.get_object("SwitchRespuesta")
        sSeguridad=self.builder.get_object("SwitchSeguridad")
        bValidar=self.builder.get_object("ButtonValidar")


        #llamadas objetos
        
        ##Llamadas Ventana Tutorial
        wTutoWindow.connect("delete-event",gtk.main_quit)
        wTutoWindow.show()
        
        ##Llamadas botones tutorial
        bSame1.connect("clicked",self.botonSame1)
        bSame2.connect("clicked",self.botonSame1)
        bDiferente1.connect("clicked",self.botonSame1)
        bDiferente2.connect("clicked",self.botonDiferente2)
        bFila1.connect("clicked",self.botonSame1)
        bFila2.connect("clicked",self.botonFila2)
        bStartTest.connect("clicked",self.botonStartTest)

        #Llamadas Botones
        bPlayAudio1.connect("clicked", self.botonAudio1)
        bPlayAudio2.connect("clicked", self.botonAudio2)
        bValidar.connect("clicked",self.enviarRespuesta)

        #Llamadas Swithces
        sRespuesta.connect("notify::active", self.siPulsado)
        sSeguridad.connect("notify::active", self.siPulsadoSeguridad)
        
    # Funciones Interfaz Tutorial
    
    def botonSame1(self,widget):
        reproduceAudio(rutaTutorial, misArchivosTutorial, 1)
        
    def botonDiferente2(self,widget):
        reproduceAudio(rutaTutorial, misArchivosTutorial, 6)
        
    def botonFila2(self,widget):
        reproduceAudio(rutaTutorial, misArchivosTutorial, 5)
        
    def botonStartTest(self,widget):
        wMainWindow=self.builder.get_object("mainWindow")
        wMainWindow.connect("delete-event",gtk.main_quit)
        wMainWindow.connect("key-press-event", self.key_press)
        wMainWindow.show()
        
         
    # Funciones interfaz Test

    def key_press (self,window,event):
        keyname=Gdk.keyval_name(event.keyval)
        if keyname=="1":
            reproduceAudio(miruta, misArchivos, miIndice1)
            print("1 pulsado")
        if keyname=="2":
            reproduceAudio(miruta, misArchivos,miIndice2)
            print("2 pulsado")
            
    def botonAudio1(self, widget):
        print("Boton 1 pulsado")
        reproduceAudio(miruta, misArchivos, miIndice1)


    def botonAudio2(self,widget):
        print("Boton 2 pulsado")
        reproduceAudio(miruta, misArchivos,miIndice2)

    def siPulsado(self,widget,estado):
        global iguales
        if widget.get_active():
            iguales=True
        else:
            iguales=False
        print(iguales)

    def siPulsadoSeguridad(self,widget,estado):
        global seguridad
        if widget.get_active():
            seguridad=True
        else:
            seguridad=False
        print(seguridad)

    def enviarRespuesta(self,widget):
        global i
        global df
        global miIndice1
        global miIndice2

        txtTextPregunta=self.builder.get_object("TextPregunta")
        sRespuesta=self.builder.get_object("SwitchRespuesta")
        sSeguridad=self.builder.get_object("SwitchSeguridad")
        
        if iguales==False:
            df.iloc[0,2]=0
            print("Son diferentes")
        else:
            df.iloc[0,2]=1
            print("Son iguales")
        if seguridad==True:
            df.iloc[0,3]=1
            print("Respuesta segura")
        else:
            df.iloc[0,3]=0
            print("Respuesta insegura")
        if os.path.exists(rutacsv):
            print("El archivo ya existe, 
            los resultados se añadirán al final del fichero.")
            df.to_csv(rutacsv, index=False, header=False, mode='a')
        else:
            print("El archivo no existe, se creará uno nuevo.")
            df.to_csv(rutacsv, index=False)
        if i<maximo:

            gtk.Label.set_text (txtTextPregunta, "Pregunta "+str(i+2)+" de 25: 
            ¿Los audios son iguales?");
            i=i+1
            miIndice1=random.randint(0,len(misArchivos)-1)
            miIndice2=random.randint(0,len(misArchivos)-1)
            df.iloc[0,0]=misArchivos[miIndice1]
            df.iloc[0,1]=misArchivos[miIndice2]
            
            #Resetear switches
            sRespuesta.set_state(False)
            sSeguridad.set_state(False)
            
        else:
            gtk.Label.set_text (txtTextPregunta, "Test acabado. ¡Gracias!");

if __name__ =='__main__':
    main=Main()
    gtk.main()
\end{verbatim}

\section{Código Python del cálculo de distancias}

\begin{verbatim}
import numpy as np
import pandas as pd
import matplotlib.pyplot as plt
import os
import math

# Función que devuelve una lista con los archivos dentro de un directorio
def listaArchivos(ruta):
    archivos = os.listdir(ruta)
    return archivos

rutaCSV="/home/vic/Documents/universidad/PFM/analisis_datos/
Test_Final/resultadosTest.csv"
rutaResultados="/home/vic/Documents/universidad/PFM/analisis_datos/
Test_Final/analisisTestMetros.csv"

dx=0.53
dy=0.9
dy0=3.80+1.75+0.29
dpasillo=1.80
df=pd.read_csv(rutaCSV)

resultados=pd.DataFrame(
    {
        'archivo1': np.array([None]*len(df)),
        'archivo2': np.array([None]*len(df)),
        'respuesta': np.array([None]*len(df)),
        'seguridad': np.array([None]*len(df)),
        'distancia Fila': np.array([None]*len(df)),
        'distancia Butaca':np.array([None]*len(df)),
        'distancia': np.array([None]*len(df)),
        'distancia fuente X': np.array([None]*len(df)),
        'distancia fuente Y': np.array([None]*len(df)),
        'distancia Fuente': np.array([None]*len(df)),
    }

)

#Recorrer todo el documento
for i in range(0,len(df)):
    string1=df.iloc[i,0]
    string2=df.iloc[i,1]


    fila1=string1[0]
    fila2=string2[0]
    numerofila1=0
    numerofila2=0
    numerobutaca1=0
    numerobutaca2=0
    distanciaFila=0
    distanciaButaca=0
    distancia=0
    distanciaFilaX=0
    distanciaFilaY=0
    distanciaFila=0


    #Extraer fila y butaca del primer archivo. 
    #En primer lugar, analiza si la fila es A o F
    #Después, analiza si la fila es mayor que 10 viendo 
    #si la empieza el código de butaca B
    #Tras guardar la fila, analiza la longitud de la butaca para saber cuántos str 
    #tiene que tomar
    #Finalmente aplica corrección según sea fila A o F (desplazar 3)

    if fila1=="F":
        if string1[2]!="B":
            numerofila1=int(string1[1]+string1[2])
            if string1[5]=="_" or string1[5]=="." or string1[4]=="B":
                numerobutaca1=int(string1[4])
            else:
                numerobutaca1=int(string1[4]+string1[5])


        else:
            numerofila1=int(string1[1])
            if string1[4]=="_" or string1[4]=="." or string1[4]=="B":
                numerobutaca1=int(string1[3])
            else:
                numerobutaca1=int(string1[3]+string1[4])

        numerofila1=numerofila1+3


    else:
        numerofila1=int(string1[1])
        if string1[4]=="_" or string1[4]=="." or string1[4]=="B":
                numerobutaca1=int(string1[3])
        else:
                numerobutaca1=int(string1[3]+string1[4])


    #Extraer fila y butaca del segundo archivo. 
    #En primer lugar, analiza si la fila es A o F
    #Después, analiza si la fila es mayor que 10 viendo 
    #si la empieza el código de butaca B
    #Tras guardar la fila, analiza la longitud de la butaca para saber cuántos str 
    #tiene que tomar
    #Finalmente aplica corrección según sea fila A o F (desplazar 3)
    if fila2=="F":
        if string2[2]!="B":
            numerofila2=int(string2[1]+string2[2])
            if string2[5]=="_" or string2[5]=="." or string2[4]=="B":
                numerobutaca2=int(string2[4])
            else:
                numerobutaca2=int(string2[4]+string2[5])
        else:
            numerofila2=int(string2[1])
            if string2[4]=="_" or string2[4]=="." or string2[4]=="B":
                numerobutaca2=int(string2[3])
            else:
                numerobutaca2=int(string2[3]+string2[4])

        numerofila2=numerofila2+3

    else:
        numerofila2=int(string2[1])
        if string2[4]=="_" or string2[4]=="." or string2[4]=="B":
                numerobutaca2=int(string2[3])
        else:
                numerobutaca2=int(string2[3]+string2[4])

    #Hacer cálculos de distancia
    if numerobutaca1%2!=0:
        numerobutaca1=numerobutaca1+1
    if numerobutaca2%2!=0:
        numerobutaca2=numerobutaca2+1

    distanciaFila=abs(numerofila1-numerofila2)
    distanciaButaca=(abs(numerobutaca1-numerobutaca2))/2
    distancia=math.sqrt(distanciaFila*dy*distanciaFila*dy +
    distanciaButaca*dx*distanciaButaca*dx)

    #Calculo de distancia respecto de la fuente
    if numerobutaca1>14:
        x1=dx/2+dx*(numerobutaca1/2-1)+dpasillo
    else:
        x1=dx/2+dx*(numerobutaca1/2-1)
    if numerobutaca2>14:
        x2=dx/2+dx*(numerobutaca2/2-1)+dpasillo
    else:
        x2=dx/2+dx*(numerobutaca2/2-1)
    distanciaFuenteY=min((dy0+dy*(numerofila1-1)),(dy0+dy*(numerofila2-1)))
    +distanciaFila/2
    if (numerobutaca1>14 and numerobutaca2>14):
        distanciaFuenteX=min((dx/2+dx*(numerobutaca1/2-1)),
        (dx/2+dx*(numerobutaca2/2-1)))+dx*abs(x1-x2)+dpasillo
    else:
        distanciaFuenteX=min((dx/2+dx*(numerobutaca1/2-1)),
        (dx/2+dx*(numerobutaca2/2-1)))+dx*abs(x1-x2)/2

    #distanciaFuenteX=min((dx/2+dx*(numerobutaca1/2-1)),
    (dx/2+dx*(numerobutaca2/2-1)))+dx*distanciaButaca/2
    #distanciaFuenteY=min((dy0+dy*(numerofila1-1)),
    (dy0+dy*(numerofila2-1)))+distanciaFila/2
    distanciaFuente=math.sqrt(distanciaFuenteX*distanciaFuenteX+
    distanciaFuenteY*distanciaFuenteY)

    #Añadir al plot
    plt.plot(distanciaFuenteX,distanciaFuenteY,'ok')
    #Almacenar datos en resultados
    resultados.iloc[i,0]=df.iloc[i,0]
    resultados.iloc[i,1]=df.iloc[i,1]
    resultados.iloc[i,2]=df.iloc[i,2]
    resultados.iloc[i,3]=df.iloc[i,3]
    resultados.iloc[i,4]=distanciaFila
    resultados.iloc[i,5]=distanciaButaca
    resultados.iloc[i,6]=distancia
    resultados.iloc[i,7]=distanciaFuenteX
    resultados.iloc[i,8]=distanciaFuenteY
    resultados.iloc[i,9]=distanciaFuente

#Imprimir y guardar en csv
print(resultados)
resultados.to_csv(rutaResultados, index=False)

#Representar
plt.axis('equal')
plt.xlim([0,10]) #<-- set the x axis limits
plt.ylim([0,24]) #<-- set the y axis limits
plt.grid(b=True, which='major') #<-- plot grid lines
plt.show()
\end{verbatim}

\chapter{Consentimiento informado}
    \includepdf[pages=-]{consentimientoInformado.pdf}
    
\chapter{Código del script en R}
\begin{verbatim}
library(sensR)
library(numDeriv)
library(multcomp)

# Calculo para distancia respecto a distancia entre butacas en unidades unitarias
aciertos<- c(6,17,23,31,55,40,61,59,73,52,47,54,47,15,21,14,18)
total<-c(15,45,52,60,81,64,81,69,87,59,55,57,50,15,22,14,18)
Protocolos<- c("twoAFC","twoAFC","twoAFC","twoAFC","twoAFC","twoAFC",
"twoAFC","twoAFC","twoAFC","twoAFC","twoAFC","twoAFC","twoAFC",
"twoAFC","twoAFC","twoAFC","twoAFC")
a<-dprime_table(aciertos,total,Protocolos)

# Calculo para distancia respecto de la fuente en metros
aciertos2<- c(10,25,24,41,41,53,79,81,66,53,74,43,25,18)
total2<-c(15,35,32,53,55,67,101,99,84,63,95,62,44,39)
Protocolos2<- c("twoAFC","twoAFC","twoAFC","twoAFC","twoAFC","twoAFC",
"twoAFC","twoAFC","twoAFC","twoAFC","twoAFC","twoAFC","twoAFC","twoAFC")
a2<-dprime_table(aciertos2,total2,Protocolos2)

# Calculo para distancia respecto de distancia entre butacas en metros
aciertos3<- c(21,36,55,81,71,67,61,48,59,46,32,32,24)
total3<- c(61,76,102,111,95,78,70,52,62,47,33,33,24)
Protocolos3<- c("twoAFC","twoAFC","twoAFC","twoAFC","twoAFC","twoAFC",
"twoAFC","twoAFC","twoAFC","twoAFC","twoAFC","twoAFC","twoAFC")
a3<- dprime_table(aciertos3,total3,Protocolos3)

# Calculo para distancia respecto de la fuente en metros solo SIN DUDA
aciertos4<- c(8,22,17,34,36,44,69,69,56,42,59,34,15,14)
total4<-c(11,30,23,39,47,52,81,84,68,49,77,42,25,29)
Protocolos4<- c("twoAFC","twoAFC","twoAFC","twoAFC","twoAFC","twoAFC",
"twoAFC","twoAFC","twoAFC","twoAFC","twoAFC","twoAFC","twoAFC","twoAFC")
a4<-dprime_table(aciertos4,total4,Protocolos4)

# Calculo para distancia respecto de distancia entre butacas en metros solo SIN DUDA
aciertos5<- c(14,20,41,62,50,57,57,39,51,43,30,32,23)
total5<- c(41,49,74,79,65,62,64,40,53,44,31,32,23)
Protocolos5<- c("twoAFC","twoAFC","twoAFC","twoAFC","twoAFC","twoAFC",
"twoAFC","twoAFC","twoAFC","twoAFC","twoAFC","twoAFC","twoAFC")
a5<- dprime_table(aciertos5,total5,Protocolos5)

\end{verbatim}

\chapter{Resultados de los test}

\section{Resultados del test previo}
\scriptsize
\begin{longtable}[c]{|c|c|c|c|}
		\hline
		\textbf{archivo1}&\textbf{archivo2}&\textbf{respuesta}&\textbf{seguridad}\\ \hline
		\endfirsthead
		\hline
		\textbf{archivo1}&\textbf{archivo2}&\textbf{respuesta}&\textbf{seguridad}\\ \hline
		\endhead
		
		F15B25.wav&F15B6REPE.wav&0&1\\ \hline
F15B25.wav&F15B6REPE.wav&1&1\\ \hline
F15B25.wav&F15B6REPE.wav&0&1\\ \hline
F15B25.wav&F15B6REPE.wav&0&1\\ \hline
F15B25.wav&F15B6REPE.wav&0&0\\ \hline
F15B25.wav&F1B27.wav&0&1\\ \hline
F15B25.wav&F1B27.wav&0&1\\ \hline
F15B25.wav&F1B28.wav&0&1\\ \hline
F15B25.wav&F1B28.wav&1&1\\ \hline
F15B25.wav&F1B8.wav&0&1\\ \hline
F15B26REPE.wav&F15B26REPE.wav&1&1\\ \hline
F15B26REPE.wav&F15B5.wav&1&1\\ \hline
F15B26REPE.wav&F15B6REPE.wav&1&0\\ \hline
F15B26REPE.wav&F1B2.wav&0&0\\ \hline
F15B26REPE.wav&F1B27.wav&0&0\\ \hline
F15B26REPE.wav&F1B27.wav&0&1\\ \hline
F15B26REPE.wav&F1B28.wav&1&1\\ \hline
F15B26REPE.wav&F1B8.wav&0&1\\ \hline
F15B26REPE.wav&F1B8.wav&0&1\\ \hline
F15B5.wav&F15B25.wav&1&1\\ \hline
F15B5.wav&F15B25.wav&0&1\\ \hline
F15B5.wav&F15B26REPE.wav&0&1\\ \hline
F15B5.wav&F1B2.wav&0&1\\ \hline
F15B5.wav&F1B27.wav&1&1\\ \hline
F15B5.wav&F1B28.wav&0&1\\ \hline
F15B5.wav&F1B8.wav&0&0\\ \hline
F15B6REPE.wav&F15B25.wav&0&0\\ \hline
F15B6REPE.wav&F15B26REPE.wav&1&0\\ \hline
F15B6REPE.wav&F15B26REPE.wav&0&1\\ \hline
F15B6REPE.wav&F15B6REPE.wav&1&1\\ \hline
F15B6REPE.wav&F1B28.wav&0&1\\ \hline
F15B6REPE.wav&F1B28.wav&0&1\\ \hline
F1B2.wav&F15B25.wav&0&1\\ \hline
F1B2.wav&F15B25.wav&0&1\\ \hline
F1B2.wav&F15B26REPE.wav&0&1\\ \hline
F1B2.wav&F15B6REPE.wav&0&1\\ \hline
F1B2.wav&F1B2.wav&1&1\\ \hline
F1B2.wav&F1B2.wav&1&1\\ \hline
F1B2.wav&F1B27.wav&0&1\\ \hline
F1B2.wav&F1B27.wav&0&1\\ \hline
F1B27.wav&F15B25.wav&0&1\\ \hline
F1B27.wav&F15B26REPE.wav&0&1\\ \hline
F1B27.wav&F15B26REPE.wav&1&0\\ \hline
F1B27.wav&F15B5.wav&1&1\\ \hline
F1B27.wav&F15B6REPE.wav&0&1\\ \hline
F1B27.wav&F1B27.wav&1&1\\ \hline
F1B27.wav&F1B27.wav&1&1\\ \hline
F1B27.wav&F1B28.wav&1&1\\ \hline
F1B27.wav&F1B28.wav&1&1\\ \hline
F1B27.wav&F1B8.wav&0&1\\ \hline
F1B27.wav&F1B8.wav&1&1\\ \hline
F1B28.wav&F15B26REPE.wav&0&1\\ \hline
F1B28.wav&F15B26REPE.wav&0&1\\ \hline
F1B28.wav&F15B5.wav&0&1\\ \hline
F1B28.wav&F15B5.wav&0&1\\ \hline
F1B28.wav&F1B2.wav&0&1\\ \hline
F1B28.wav&F1B27.wav&0&1\\ \hline
F1B28.wav&F1B27.wav&1&1\\ \hline
F1B28.wav&F1B8.wav&0&1\\ \hline
F1B8.wav&F15B25.wav&0&1\\ \hline
F1B8.wav&F15B6REPE.wav&0&1\\ \hline
F1B8.wav&F1B27.wav&0&1\\ \hline
F1B8.wav&F1B27.wav&0&1\\ \hline
F1B8.wav&F1B27.wav&0&1\\ \hline
F1B8.wav&F1B28.wav&0&1\\ \hline
F1B8.wav&F1B8.wav&1&1\\ \hline
F1B8.wav&F1B8.wav&1&0\\ \hline
F1B8.wav&F1B8.wav&0&1\\ \hline
F1B8.wav&F1B8.wav&1&1\\ \hline
F1B8.wav&F1B8.wav&1&1\\ \hline
\caption{Resultados obtenidos del test previo.} 
\end{longtable}
\normalsize		
Las dos primeras columnas muestran el nombre de los dos archivos de audio de donde se puede extraer la posición. Los números detrás de las letras ``F'' y ``B'' marcan el número de fila y de butaca, respectivamente. La columna ``Respuesta'' marca ``0'' si se seleccionó que ambos audios eran diferentes y ``1'' si se seleccionó que eran iguales. La columna ``Seguridad'' determina el nivel de seguridad marcado en la respuesta: ``1'' si se está seguro, ``0'' si no.

\section{Resultados del test final}

\scriptsize
\begin{longtable}[c]{|c|c|c|c|c|c|c|c|c|c|}

		\hline
		\textbf{Archivo1}&\textbf{Archivo2}&\textbf{Respuesta}&\textbf{Seguridad}&\textbf{D.Fila}&\textbf{D.Butaca}&\textbf{Distancia}&\textbf{D.fte X}&\textbf{D.fte Y}&\textbf{D.Fuente}\\ \hline
		\endfirsthead
		\hline
		\textbf{Archivo1}&\textbf{Archivo2}&\textbf{Respuesta}&\textbf{Seguridad}&\textbf{D.Fila}&\textbf{D.Butaca}&\textbf{Distancia}&\textbf{D.fte X}&\textbf{D.fte Y}&\textbf{D.Fuente}\\ \hline
		\endhead
		
		F13B28.wav&F13B28.wav&1&1&0&0&0.000&8.955&19.340&21.313\\ \hline
F2B26.wav&F2B26.wav&1&0&0&0&0.000&8.425&9.440&12.653\\ \hline
F6B24.wav&F6B24.wav&1&0&0&0&0.000&7.895&13.040&15.244\\ \hline
F7B14.wav&F7B14.wav&1&1&0&0&0.000&3.445&13.940&14.359\\ \hline
F14B26.wav&F14B26.wav&1&1&0&0&0.000&8.425&20.240&21.923\\ \hline
F4B8.wav&F4B8.wav&1&1&0&0&0.000&1.855&11.240&11.392\\ \hline
F10B14.wav&F10B16.wav&0&1&0&1&0.530&4.062&16.640&17.129\\ \hline
F7B22.wav&F8B20.wav&1&1&1&1&1.044&7.116&14.440&16.098\\ \hline
F14B20.wav&F15B18.wav&1&1&1&1&1.044&6.586&20.740&21.761\\ \hline
F10B28.wav&F11B26.wav&0&1&1&1&1.044&8.706&17.140&19.224\\ \hline
F9B10.wav&F10B8.wav&1&0&1&1&1.044&1.995&16.240&16.362\\ \hline
F8B6.wav&F9B4.wav&0&1&1&1&1.044&0.935&15.340&15.368\\ \hline
F11B6.wav&F12B4.wav&0&1&1&1&1.044&0.935&18.040&18.064\\ \hline
F16B8.wav&F15B10.wav&1&0&1&1&1.044&1.995&21.640&21.732\\ \hline
F5B20.wav&F6B18.wav&1&1&1&1&1.044&6.586&12.640&14.253\\ \hline
F11B20.wav&F12B18.wav&1&0&1&1&1.044&6.586&18.040&19.205\\ \hline
F14B12.wav&F15B10.wav&1&1&1&1&1.044&2.525&20.740&20.893\\ \hline
F4B14.wav&F5B12.wav&0&1&1&1&1.044&3.055&11.740&12.131\\ \hline
F3B24.wav&F2B26.wav&1&0&1&1&1.044&8.176&9.940&12.870\\ \hline
F9B24.wav&F8B26.wav&1&1&1&1&1.044&8.176&15.340&17.383\\ \hline
F10B28.wav&F11B26.wav&0&1&1&1&1.044&8.706&17.140&19.224\\ \hline
F2B20.wav&F3B24.wav&1&1&1&2&1.391&7.397&9.940&12.390\\ \hline
F15B24.wav&F14B20.wav&1&1&1&2&1.391&7.397&20.740&22.020\\ \hline
F14B6.wav&F13B2.wav&0&1&1&2&1.391&0.546&19.840&19.848\\ \hline
F13B22.wav&F12B18.wav&0&0&1&2&1.391&6.867&18.940&20.146\\ \hline
F10B14.wav&F9B18.wav&1&1&1&2&1.391&4.203&16.240&16.775\\ \hline
F4B28.wav&F3B24.wav&1&0&1&2&1.391&8.457&10.840&13.749\\ \hline
F13B28.wav&F12B24.wav&0&0&1&2&1.391&8.457&18.940&20.742\\ \hline
F10B2.wav&F11B6.wav&0&0&1&2&1.391&0.546&17.140&17.149\\ \hline
F11B6.wav&F12B10.wav&1&1&1&2&1.391&1.606&18.040&18.111\\ \hline
F16B8.wav&F15B4.wav&1&0&1&2&1.391&1.076&21.640&21.667\\ \hline
F9B18.wav&F10B22.wav&0&1&1&2&1.391&6.867&16.240&17.632\\ \hline
F4B16.wav&F4B22.wav&0&1&0&3&1.590&6.618&11.240&13.043\\ \hline
F15B4.wav&F15B10.wav&1&0&0&3&1.590&1.216&21.140&21.175\\ \hline
A2B10.wav&A2B4.wav&1&1&0&3&1.590&1.216&6.740&6.849\\ \hline
F14B20.wav&F14B26.wav&0&0&0&3&1.590&7.678&20.240&21.647\\ \hline
F3B4.wav&F3B10.wav&1&1&0&3&1.590&1.216&10.340&10.411\\ \hline
F2B26.wav&F2B20.wav&0&0&0&3&1.590&7.678&9.440&12.168\\ \hline
F1B8.wav&F1B14.wav&1&1&0&3&1.590&2.276&8.540&8.838\\ \hline
F1B14.wav&F1B8.wav&1&1&0&3&1.590&2.276&8.540&8.838\\ \hline
F12B10.wav&F12B4.wav&1&1&0&3&1.590&1.216&18.440&18.480\\ \hline
F16B28.wav&F16B22.wav&0&1&0&3&1.590&8.208&22.040&23.519\\ \hline
F10B2.wav&F10B8.wav&1&1&0&3&1.590&0.686&16.640&16.654\\ \hline
F7B28.wav&F7B22.wav&2&0&0&3&1.590&8.208&13.940&16.177\\ \hline
F12B18.wav&F12B24.wav&1&1&0&3&1.590&7.148&18.440&19.777\\ \hline
F2B12.wav&F3B18.wav&1&0&1&3&1.827&3.813&9.940&10.646\\ \hline
F16B16.wav&F15B10.wav&0&1&1&3&1.827&3.283&21.640&21.888\\ \hline
F1B26.wav&A3B20.wav&1&0&1&3&1.827&7.678&8.140&11.190\\ \hline
F10B14.wav&F11B20.wav&0&1&1&3&1.827&4.343&17.140&17.682\\ \hline
F15B18.wav&F14B12.wav&1&1&1&3&1.827&3.813&20.740&21.088\\ \hline
F12B10.wav&F14B12.wav&1&1&2&1&1.876&2.525&19.440&19.603\\ \hline
F10B22.wav&F12B24.wav&1&1&2&1&1.876&7.646&17.640&19.226\\ \hline
F11B26.wav&F9B24.wav&1&1&2&1&1.876&8.176&16.740&18.630\\ \hline
F8B20.wav&F6B18.wav&1&1&2&1&1.876&6.586&14.040&15.508\\ \hline
F8B6.wav&F6B4.wav&1&1&2&1&1.876&0.935&14.040&14.071\\ \hline
F5B20.wav&F7B22.wav&1&1&2&1&1.876&7.116&13.140&14.943\\ \hline
F7B16.wav&F9B18.wav&0&0&2&1&1.876&6.056&14.940&16.121\\ \hline
F3B18.wav&F1B20.wav&1&0&2&1&1.876&6.586&9.540&11.592\\ \hline
F1B8.wav&A2B10.wav&0&0&2&1&1.876&1.995&7.740&7.993\\ \hline
F6B18.wav&F8B20.wav&1&1&2&1&1.876&6.586&14.040&15.508\\ \hline
F7B8.wav&F9B10.wav&0&0&2&1&1.876&1.995&14.940&15.073\\ \hline
F13B22.wav&F15B24.wav&1&1&2&1&1.876&7.646&20.340&21.730\\ \hline
F7B22.wav&F5B20.wav&1&0&2&1&1.876&7.116&13.140&14.943\\ \hline
F3B4.wav&F5B6.wav&1&1&2&1&1.876&0.935&11.340&11.379\\ \hline
F1B26.wav&F3B24.wav&0&1&2&1&1.876&8.176&9.540&12.564\\ \hline
F15B18.wav&F13B16.wav&0&1&2&1&1.876&6.056&20.340&21.222\\ \hline
F9B10.wav&F7B8.wav&1&1&2&1&1.876&1.995&14.940&15.073\\ \hline
F11B26.wav&F13B22.wav&1&1&2&2&2.089&7.927&18.540&20.163\\ \hline
F5B6.wav&F3B10.wav&0&1&2&2&2.089&1.606&11.340&11.453\\ \hline
F2B12.wav&F4B8.wav&1&1&2&2&2.089&2.136&10.440&10.656\\ \hline
F13B22.wav&F15B18.wav&1&1&2&2&2.089&6.867&20.340&21.468\\ \hline
F7B14.wav&F9B18.wav&0&0&2&2&2.089&4.203&14.940&15.520\\ \hline
F11B12.wav&F13B16.wav&1&0&2&2&2.089&3.673&18.540&18.900\\ \hline
F9B18.wav&F7B22.wav&1&1&2&2&2.089&6.867&14.940&16.443\\ \hline
F13B22.wav&F15B18.wav&1&1&2&2&2.089&6.867&20.340&21.468\\ \hline
F10B8.wav&F8B12.wav&1&1&2&2&2.089&2.136&15.840&15.983\\ \hline
F10B14.wav&F12B18.wav&0&1&2&2&2.089&4.203&17.640&18.134\\ \hline
A2B10.wav&F1B14.wav&0&0&2&2&2.089&2.666&7.740&8.186\\ \hline
F9B24.wav&F11B20.wav&0&1&2&2&2.089&7.397&16.740&18.301\\ \hline
F6B24.wav&F4B28.wav&1&1&2&2&2.089&8.457&12.240&14.877\\ \hline
F12B18.wav&F10B14.wav&1&1&2&2&2.089&4.203&17.640&18.134\\ \hline
F9B4.wav&F7B8.wav&1&1&2&2&2.089&1.076&14.940&14.979\\ \hline
F13B2.wav&F11B6.wav&0&1&2&2&2.089&0.546&18.540&18.548\\ \hline
F11B12.wav&F13B8.wav&0&1&2&2&2.089&2.136&18.540&18.663\\ \hline
F10B16.wav&F10B8.wav&1&1&0&4&2.120&2.894&16.640&16.890\\ \hline
F13B8.wav&F13B16.wav&0&0&0&4&2.120&2.894&19.340&19.555\\ \hline
F12B10.wav&F13B2.wav&1&1&1&4&2.303&0.827&18.940&18.958\\ \hline
F10B2.wav&F9B10.wav&0&0&1&4&2.303&0.827&16.240&16.261\\ \hline
F13B14.wav&F14B6.wav&0&1&1&4&2.303&1.887&19.840&19.930\\ \hline
F10B16.wav&F9B24.wav&1&1&1&4&2.303&6.899&16.240&17.644\\ \hline
F12B4.wav&F11B12.wav&0&1&1&4&2.303&1.357&18.040&18.091\\ \hline
F2B26.wav&A3B20.wav&0&0&2&3&2.402&7.678&8.640&11.558\\ \hline
F5B12.wav&F3B18.wav&0&1&2&3&2.402&3.813&11.340&11.964\\ \hline
F4B14.wav&F2B20.wav&0&0&2&3&2.402&4.343&10.440&11.307\\ \hline
F16B16.wav&F13B16.wav&0&1&3&0&2.700&5.775&20.840&21.625\\ \hline
F6B24.wav&F9B24.wav&0&0&3&0&2.700&7.895&14.540&16.545\\ \hline
F6B24.wav&F3B24.wav&0&1&3&0&2.700&7.895&11.840&14.231\\ \hline
A1B2.wav&F1B2.wav&1&0&3&0&2.700&0.265&7.340&7.345\\ \hline
F9B4.wav&F6B4.wav&0&0&3&0&2.700&0.795&14.540&14.562\\ \hline
F15B24.wav&F12B24.wav&1&1&3&0&2.700&7.895&19.940&21.446\\ \hline
F1B8.wav&A1B8.wav&1&1&3&0&2.700&1.855&7.340&7.571\\ \hline
F1B8.wav&A1B8.wav&1&1&3&0&2.700&1.855&7.340&7.571\\ \hline
F3B10.wav&F6B10.wav&0&0&3&0&2.700&2.385&11.840&12.078\\ \hline
F12B24.wav&F15B24.wav&1&1&3&0&2.700&7.895&19.940&21.446\\ \hline
F10B8.wav&F7B8.wav&1&1&3&0&2.700&1.855&15.440&15.551\\ \hline
F9B4.wav&F12B4.wav&0&1&3&0&2.700&0.795&17.240&17.258\\ \hline
F16B16.wav&F13B16.wav&0&1&3&0&2.700&5.775&20.840&21.625\\ \hline
F7B16.wav&F10B16.wav&1&1&3&0&2.700&5.775&15.440&16.485\\ \hline
F13B2.wav&F10B2.wav&0&1&3&0&2.700&0.265&18.140&18.142\\ \hline
F13B8.wav&F16B8.wav&0&0&3&0&2.700&1.855&20.840&20.922\\ \hline
F1B8.wav&F4B8.wav&1&0&3&0&2.700&1.855&10.040&10.210\\ \hline
F15B18.wav&F12B18.wav&1&0&3&0&2.700&6.305&19.940&20.913\\ \hline
F7B28.wav&F4B28.wav&1&1&3&0&2.700&8.955&12.740&15.572\\ \hline
F14B6.wav&F11B6.wav&0&1&3&0&2.700&1.325&19.040&19.086\\ \hline
F8B6.wav&F5B6.wav&1&0&3&0&2.700&1.325&13.640&13.704\\ \hline
F13B16.wav&F16B14.wav&0&0&3&1&2.752&4.062&20.840&21.232\\ \hline
F1B20.wav&F4B22.wav&1&1&3&1&2.752&7.116&10.040&12.306\\ \hline
F7B16.wav&F10B14.wav&0&0&3&1&2.752&4.062&15.440&15.965\\ \hline
F7B14.wav&F4B16.wav&1&1&3&1&2.752&4.062&12.740&13.372\\ \hline
F13B16.wav&F16B14.wav&0&1&3&1&2.752&4.062&20.840&21.232\\ \hline
A3B20.wav&F3B18.wav&1&1&3&1&2.752&6.586&9.140&11.266\\ \hline
F1B14.wav&A1B16.wav&0&1&3&1&2.752&4.062&7.340&8.389\\ \hline
F13B14.wav&F16B16.wav&1&1&3&1&2.752&4.062&20.840&21.232\\ \hline
F5B6.wav&F7B14.wav&0&1&2&4&2.781&1.887&13.140&13.275\\ \hline
F3B10.wav&F1B2.wav&1&1&2&4&2.781&0.827&9.540&9.576\\ \hline
F13B14.wav&F11B6.wav&0&0&2&4&2.781&1.887&18.540&18.636\\ \hline
F14B20.wav&F16B28.wav&1&1&2&4&2.781&7.959&21.240&22.682\\ \hline
F6B10.wav&F4B2.wav&0&1&2&4&2.781&0.827&12.240&12.268\\ \hline
F15B24.wav&F13B16.wav&1&0&2&4&2.781&6.899&20.340&21.478\\ \hline
F9B4.wav&F11B12.wav&0&0&2&4&2.781&1.357&16.740&16.795\\ \hline
F9B24.wav&F10B14.wav&1&0&1&5&2.799&4.624&16.240&16.886\\ \hline
F13B16.wav&F14B6.wav&0&1&1&5&2.799&2.504&19.840&19.997\\ \hline
F15B4.wav&F16B14.wav&1&1&1&5&2.799&1.497&21.640&21.692\\ \hline
F10B22.wav&F11B12.wav&1&0&1&5&2.799&4.094&17.140&17.622\\ \hline
F4B8.wav&F3B18.wav&1&0&1&5&2.799&3.034&10.840&11.257\\ \hline
F7B28.wav&F6B18.wav&1&1&1&5&2.799&7.710&13.540&15.581\\ \hline
F16B14.wav&F15B24.wav&0&0&1&5&2.799&4.624&21.640&22.129\\ \hline
F13B22.wav&F14B12.wav&1&0&1&5&2.799&4.094&19.840&20.258\\ \hline
F13B22.wav&F14B12.wav&1&1&1&5&2.799&4.094&19.840&20.258\\ \hline
F10B14.wav&F9B4.wav&0&1&1&5&2.799&1.497&16.240&16.309\\ \hline
F8B20.wav&F9B10.wav&1&1&1&5&2.799&3.564&15.340&15.749\\ \hline
F4B16.wav&F1B20.wav&1&0&3&2&2.901&6.337&10.040&11.873\\ \hline
F1B20.wav&A1B16.wav&0&0&3&2&2.901&6.337&7.340&9.697\\ \hline
F14B26.wav&F11B20.wav&1&1&3&3&3.133&7.678&19.040&20.530\\ \hline
F4B2.wav&F1B8.wav&0&0&3&3&3.133&0.686&10.040&10.063\\ \hline
F1B8.wav&F4B14.wav&1&1&3&3&3.133&2.276&10.040&10.295\\ \hline
F5B20.wav&F2B26.wav&0&0&3&3&3.133&7.678&10.940&13.365\\ \hline
F8B12.wav&F11B6.wav&0&1&3&3&3.133&1.746&16.340&16.433\\ \hline
F15B18.wav&F12B24.wav&1&1&3&3&3.133&7.148&19.940&21.182\\ \hline
F16B22.wav&F13B28.wav&0&1&3&3&3.133&8.208&20.840&22.398\\ \hline
F11B26.wav&F14B20.wav&1&1&3&3&3.133&7.678&19.040&20.530\\ \hline
F10B28.wav&F7B22.wav&0&1&3&3&3.133&8.208&15.440&17.486\\ \hline
F11B12.wav&F8B6.wav&0&1&3&3&3.133&1.746&16.340&16.433\\ \hline
F7B16.wav&F4B22.wav&1&1&3&3&3.133&6.618&12.740&14.356\\ \hline
F10B14.wav&F13B8.wav&0&1&3&3&3.133&2.276&18.140&18.282\\ \hline
F8B26.wav&F5B20.wav&0&1&3&3&3.133&7.678&13.640&15.652\\ \hline
F10B8.wav&F7B2.wav&0&0&3&3&3.133&0.686&15.440&15.455\\ \hline
F10B8.wav&F7B14.wav&0&1&3&3&3.133&2.276&15.440&15.607\\ \hline
F7B22.wav&F10B16.wav&1&1&3&3&3.133&6.618&15.440&16.798\\ \hline
F8B12.wav&F11B6.wav&0&1&3&3&3.133&1.746&16.340&16.433\\ \hline
F4B28.wav&F7B22.wav&0&0&3&3&3.133&8.208&12.740&15.155\\ \hline
F12B4.wav&F15B10.wav&1&1&3&3&3.133&1.216&19.940&19.977\\ \hline
F1B14.wav&A1B8.wav&0&1&3&3&3.133&2.276&7.340&7.685\\ \hline
F2B6.wav&F5B12.wav&1&1&3&3&3.133&1.746&10.940&11.079\\ \hline
F4B16.wav&F4B28.wav&1&1&0&6&3.180&7.460&11.240&13.491\\ \hline
F13B16.wav&F13B28.wav&0&1&0&6&3.180&7.460&19.340&20.729\\ \hline
F1B14.wav&A2B4.wav&0&1&2&5&3.204&1.497&7.740&7.883\\ \hline
F8B26.wav&F10B16.wav&1&0&2&5&3.204&7.180&15.840&17.391\\ \hline
F2B12.wav&F4B22.wav&1&1&2&5&3.204&4.094&10.440&11.214\\ \hline
F3B4.wav&F1B14.wav&0&1&2&5&3.204&1.497&9.540&9.657\\ \hline
F1B8.wav&A2B18.wav&1&1&2&5&3.204&3.034&7.740&8.313\\ \hline
F14B12.wav&F16B22.wav&0&1&2&5&3.204&4.094&21.240&21.631\\ \hline
F15B18.wav&F13B8.wav&1&0&2&5&3.204&3.034&20.340&20.565\\ \hline
F5B12.wav&F7B2.wav&0&0&2&5&3.204&0.967&13.140&13.176\\ \hline
F1B14.wav&F3B24.wav&1&1&2&5&3.204&4.624&9.540&10.602\\ \hline
F13B14.wav&F15B24.wav&0&0&2&5&3.204&4.624&20.340&20.859\\ \hline
A2B4.wav&F1B14.wav&0&1&2&5&3.204&1.497&7.740&7.883\\ \hline
F12B18.wav&F10B8.wav&0&0&2&5&3.204&3.034&17.640&17.899\\ \hline
A2B10.wav&F1B20.wav&0&1&2&5&3.204&3.564&7.740&8.521\\ \hline
F13B16.wav&F11B6.wav&0&1&2&5&3.204&2.504&18.540&18.708\\ \hline
F12B24.wav&F10B14.wav&0&1&2&5&3.204&4.624&17.640&18.236\\ \hline
F13B16.wav&F12B4.wav&1&1&1&6&3.305&2.115&18.940&19.058\\ \hline
F15B4.wav&F16B16.wav&1&1&1&6&3.305&2.115&21.640&21.743\\ \hline
F12B18.wav&F11B6.wav&1&1&1&6&3.305&2.645&18.040&18.233\\ \hline
F3B10.wav&F4B22.wav&0&0&1&6&3.305&3.705&10.840&11.456\\ \hline
F9B24.wav&F8B12.wav&1&0&1&6&3.305&4.235&15.340&15.914\\ \hline
F12B24.wav&F11B12.wav&0&1&1&6&3.305&4.235&18.040&18.530\\ \hline
F15B10.wav&F12B18.wav&0&1&3&4&3.433&3.424&19.940&20.232\\ \hline
F6B10.wav&F9B18.wav&1&0&3&4&3.433&3.424&14.540&14.938\\ \hline
F9B10.wav&F6B18.wav&0&1&3&4&3.433&3.424&14.540&14.938\\ \hline
F10B14.wav&F7B22.wav&1&1&3&4&3.433&4.484&15.440&16.078\\ \hline
F10B8.wav&F13B16.wav&1&1&3&4&3.433&2.894&18.140&18.369\\ \hline
F5B20.wav&F2B12.wav&0&1&3&4&3.433&3.954&10.940&11.633\\ \hline
F8B20.wav&F11B12.wav&0&0&3&4&3.433&3.954&16.340&16.812\\ \hline
F16B22.wav&F13B14.wav&0&1&3&4&3.433&4.484&20.840&21.317\\ \hline
F10B8.wav&F7B16.wav&0&1&3&4&3.433&2.894&15.440&15.709\\ \hline
F13B14.wav&F10B22.wav&1&1&3&4&3.433&4.484&18.140&18.686\\ \hline
F3B4.wav&A2B4.wav&0&1&4&0&3.600&0.795&8.740&8.776\\ \hline
F9B4.wav&F5B6.wav&1&1&4&1&3.639&0.935&14.140&14.171\\ \hline
F7B2.wav&F3B4.wav&1&1&4&1&3.639&0.405&12.340&12.347\\ \hline
F5B20.wav&F9B18.wav&1&1&4&1&3.639&6.586&14.140&15.599\\ \hline
F2B6.wav&A1B8.wav&1&0&4&1&3.639&1.465&7.840&7.976\\ \hline
F9B18.wav&F13B16.wav&1&0&4&1&3.639&6.056&17.740&18.745\\ \hline
F14B20.wav&F10B22.wav&0&0&4&1&3.639&7.116&18.640&19.952\\ \hline
F11B26.wav&F7B28.wav&0&1&4&1&3.639&8.706&15.940&18.162\\ \hline
F9B18.wav&F13B16.wav&0&1&4&1&3.639&6.056&17.740&18.745\\ \hline
F9B18.wav&F11B6.wav&0&1&2&6&3.654&2.645&16.740&16.948\\ \hline
F11B26.wav&F13B14.wav&1&1&2&6&3.654&4.765&18.540&19.142\\ \hline
F11B20.wav&F13B8.wav&1&1&2&6&3.654&3.175&18.540&18.810\\ \hline
F11B26.wav&F13B14.wav&1&0&2&6&3.654&4.765&18.540&19.142\\ \hline
F8B26.wav&F10B14.wav&1&1&2&6&3.654&4.765&15.840&16.541\\ \hline
F12B18.wav&F12B4.wav&0&0&0&7&3.710&2.255&18.440&18.577\\ \hline
F11B26.wav&F11B12.wav&1&0&0&7&3.710&4.375&17.540&18.077\\ \hline
F3B4.wav&F3B18.wav&0&0&0&7&3.710&2.255&10.340&10.583\\ \hline
F13B2.wav&F13B16.wav&1&0&0&7&3.710&1.725&19.340&19.417\\ \hline
F13B2.wav&F13B16.wav&0&1&0&7&3.710&1.725&19.340&19.417\\ \hline
F14B12.wav&F14B26.wav&0&1&0&7&3.710&4.375&20.240&20.707\\ \hline
F12B4.wav&F12B18.wav&1&1&0&7&3.710&2.255&18.440&18.577\\ \hline
F3B24.wav&F3B10.wav&0&1&0&7&3.710&3.845&10.340&11.032\\ \hline
F16B14.wav&F16B28.wav&0&1&0&7&3.710&4.905&22.040&22.579\\ \hline
F16B28.wav&F16B14.wav&0&1&0&7&3.710&4.905&22.040&22.579\\ \hline
F3B10.wav&F7B14.wav&1&0&4&2&3.753&2.666&12.340&12.625\\ \hline
F12B4.wav&F16B8.wav&1&1&4&2&3.753&1.076&20.440&20.468\\ \hline
F10B8.wav&F6B4.wav&1&1&4&2&3.753&1.076&15.040&15.078\\ \hline
F16B22.wav&F12B18.wav&0&1&4&2&3.753&6.867&20.440&21.563\\ \hline
F9B4.wav&F13B8.wav&1&1&4&2&3.753&1.076&17.740&17.773\\ \hline
F9B10.wav&F13B14.wav&0&1&4&2&3.753&2.666&17.740&17.939\\ \hline
F7B22.wav&F3B18.wav&0&1&4&2&3.753&6.867&12.340&14.122\\ \hline
F13B14.wav&F9B18.wav&1&1&4&2&3.753&4.203&17.740&18.231\\ \hline
F7B14.wav&F3B18.wav&1&0&4&2&3.753&4.203&12.340&13.036\\ \hline
F7B14.wav&F3B18.wav&1&1&4&2&3.753&4.203&12.340&13.036\\ \hline
F5B12.wav&F1B8.wav&1&1&4&2&3.753&2.136&10.540&10.754\\ \hline
F4B22.wav&F8B26.wav&0&1&4&2&3.753&7.927&13.240&15.432\\ \hline
F7B8.wav&F11B12.wav&0&1&4&2&3.753&2.136&15.940&16.082\\ \hline
F5B6.wav&F9B10.wav&0&0&4&2&3.753&1.606&14.140&14.231\\ \hline
F13B28.wav&F9B24.wav&1&0&4&2&3.753&8.457&17.740&19.653\\ \hline
F9B24.wav&F5B20.wav&1&0&4&2&3.753&7.397&14.140&15.958\\ \hline
A2B10.wav&F2B20.wav&0&1&3&5&3.783&3.564&8.240&8.978\\ \hline
A1B16.wav&F1B26.wav&0&1&3&5&3.783&7.180&7.340&10.267\\ \hline
F1B26.wav&F4B16.wav&1&1&3&5&3.783&7.180&10.040&12.343\\ \hline
F2B6.wav&F1B20.wav&1&1&1&7&3.818&2.785&9.040&9.459\\ \hline
F4B14.wav&A3B20.wav&0&0&4&3&3.935&4.343&9.640&10.573\\ \hline
A2B10.wav&F3B4.wav&0&1&4&3&3.935&1.216&8.740&8.824\\ \hline
F15B18.wav&F11B12.wav&1&0&4&3&3.935&3.813&19.540&19.909\\ \hline
F6B18.wav&F2B12.wav&0&1&4&3&3.935&3.813&11.440&12.059\\ \hline
A3B20.wav&F2B6.wav&1&0&2&7&4.124&2.785&8.640&9.078\\ \hline
A3B20.wav&F2B6.wav&1&1&2&7&4.124&2.785&8.640&9.078\\ \hline
F10B2.wav&F7B14.wav&0&1&3&6&4.172&1.108&15.440&15.480\\ \hline
F1B26.wav&F4B14.wav&0&1&3&6&4.172&4.765&10.040&11.113\\ \hline
F7B14.wav&F10B2.wav&1&0&3&6&4.172&1.108&15.440&15.480\\ \hline
F10B14.wav&F13B2.wav&0&0&3&6&4.172&1.108&18.140&18.174\\ \hline
F4B28.wav&F7B16.wav&1&1&3&6&4.172&7.460&12.740&14.764\\ \hline
F16B14.wav&F13B2.wav&1&1&3&6&4.172&1.108&20.840&20.869\\ \hline
F3B18.wav&A2B10.wav&0&1&4&4&4.178&3.424&8.740&9.387\\ \hline
F9B18.wav&F5B26.wav&0&1&4&4&4.178&7.429&14.140&15.973\\ \hline
F5B12.wav&F9B4.wav&1&0&4&4&4.178&1.357&14.140&14.205\\ \hline
F14B6.wav&F10B14.wav&0&1&4&4&4.178&1.887&18.640&18.735\\ \hline
F3B10.wav&A2B18.wav&1&1&4&4&4.178&3.424&8.740&9.387\\ \hline
F10B14.wav&F14B6.wav&0&1&4&4&4.178&1.887&18.640&18.735\\ \hline
F12B24.wav&F16B16.wav&1&1&4&4&4.178&6.899&20.440&21.573\\ \hline
F6B18.wav&F2B26.wav&0&0&4&4&4.178&7.429&11.440&13.640\\ \hline
F2B20.wav&F3B4.wav&0&1&1&8&4.334&2.396&9.940&10.225\\ \hline
F9B18.wav&F10B2.wav&0&1&1&8&4.334&1.866&16.240&16.347\\ \hline
F5B26.wav&F6B10.wav&0&1&1&8&4.334&3.986&12.640&13.253\\ \hline
F5B6.wav&F4B22.wav&0&1&1&8&4.334&2.926&11.740&12.099\\ \hline
A2B18.wav&A1B2.wav&0&1&1&8&4.334&1.866&6.340&6.609\\ \hline
F11B12.wav&F10B28.wav&1&1&1&8&4.334&4.516&17.140&17.725\\ \hline
F7B22.wav&F8B6.wav&0&1&1&8&4.334&2.926&14.440&14.733\\ \hline
F8B20.wav&F9B4.wav&0&1&1&8&4.334&2.396&15.340&15.526\\ \hline
F9B18.wav&F13B28.wav&0&1&4&5&4.470&7.710&17.740&19.343\\ \hline
F14B12.wav&F10B2.wav&1&1&4&5&4.470&0.967&18.640&18.665\\ \hline
F12B18.wav&F16B8.wav&1&0&4&5&4.470&3.034&20.440&20.664\\ \hline
F4B22.wav&F8B12.wav&0&1&4&5&4.470&4.094&13.240&13.859\\ \hline
F7B22.wav&F11B12.wav&0&0&4&5&4.470&4.094&15.940&16.457\\ \hline
F4B2.wav&F8B12.wav&0&1&4&5&4.470&0.967&13.240&13.275\\ \hline
A1B2.wav&F2B12.wav&0&1&4&5&4.470&0.967&7.840&7.899\\ \hline
A1B16.wav&F2B26.wav&1&1&4&5&4.470&7.180&7.840&10.631\\ \hline
F13B8.wav&F8B6.wav&0&0&5&1&4.531&1.465&17.340&17.402\\ \hline
F3B18.wav&A1B16.wav&0&1&5&1&4.531&6.056&8.340&10.307\\ \hline
F13B28.wav&F8B26.wav&0&0&5&1&4.531&8.706&17.340&19.403\\ \hline
F11B6.wav&F6B4.wav&0&1&5&1&4.531&0.935&15.540&15.568\\ \hline
F3B10.wav&A1B8.wav&0&1&5&1&4.531&1.995&8.340&8.575\\ \hline
F6B18.wav&F1B20.wav&0&1&5&1&4.531&6.586&11.040&12.855\\ \hline
F15B10.wav&F10B8.wav&0&1&5&1&4.531&1.995&19.140&19.244\\ \hline
F16B28.wav&F11B26.wav&0&0&5&1&4.531&8.706&20.040&21.849\\ \hline
F9B10.wav&F4B8.wav&0&0&5&1&4.531&1.995&13.740&13.884\\ \hline
F11B12.wav&F6B10.wav&0&1&5&1&4.531&2.525&15.540&15.744\\ \hline
F16B14.wav&F11B12.wav&1&0&5&1&4.531&3.055&20.040&20.272\\ \hline
F5B6.wav&F10B8.wav&0&1&5&1&4.531&1.465&14.640&14.713\\ \hline
F10B16.wav&F15B18.wav&0&1&5&1&4.531&6.056&19.140&20.075\\ \hline
F2B26.wav&F7B28.wav&0&1&5&1&4.531&8.706&11.940&14.777\\ \hline
F13B2.wav&F10B16.wav&0&1&3&7&4.588&1.725&18.140&18.222\\ \hline
F4B28.wav&F1B14.wav&0&1&3&7&4.588&4.905&10.040&11.174\\ \hline
F10B16.wav&F13B2.wav&0&1&3&7&4.588&1.725&18.140&18.222\\ \hline
F6B4.wav&F9B18.wav&0&1&3&7&4.588&2.255&14.540&14.714\\ \hline
F3B18.wav&F6B4.wav&0&1&3&7&4.588&2.255&11.840&12.053\\ \hline
F3B24.wav&F6B10.wav&1&1&3&7&4.588&3.845&11.840&12.449\\ \hline
F13B8.wav&F10B22.wav&1&1&3&7&4.588&3.315&18.140&18.440\\ \hline
F11B12.wav&F8B26.wav&1&0&3&7&4.588&4.375&16.340&16.916\\ \hline
F3B18.wav&F6B4.wav&0&1&3&7&4.588&2.255&11.840&12.053\\ \hline
F12B18.wav&F15B4.wav&1&0&3&7&4.588&2.255&19.940&20.067\\ \hline
F9B10.wav&F6B24.wav&1&1&3&7&4.588&3.845&14.540&15.040\\ \hline
F7B22.wav&F4B8.wav&0&1&3&7&4.588&3.315&12.740&13.164\\ \hline
F13B8.wav&F10B22.wav&0&0&3&7&4.588&3.315&18.140&18.440\\ \hline
F5B12.wav&F8B26.wav&1&1&3&7&4.588&4.375&13.640&14.325\\ \hline
F4B22.wav&F1B8.wav&0&0&3&7&4.588&3.315&10.040&10.573\\ \hline
F4B16.wav&F7B2.wav&0&1&3&7&4.588&1.725&12.740&12.856\\ \hline
F1B8.wav&A1B22.wav&0&1&3&7&4.588&3.315&7.340&8.054\\ \hline
F13B22.wav&F11B6.wav&0&1&2&8&4.606&2.926&18.540&18.769\\ \hline
F1B8.wav&F3B24.wav&0&1&2&8&4.606&3.456&9.540&10.147\\ \hline
F11B26.wav&F9B10.wav&1&0&2&8&4.606&3.986&16.740&17.208\\ \hline
F8B20.wav&F6B4.wav&0&0&2&8&4.606&2.396&14.040&14.243\\ \hline
F11B20.wav&F9B4.wav&0&1&2&8&4.606&2.396&16.740&16.911\\ \hline
F6B10.wav&F8B26.wav&0&1&2&8&4.606&3.986&14.040&14.595\\ \hline
F3B24.wav&F1B8.wav&0&1&2&8&4.606&3.456&9.540&10.147\\ \hline
F1B2.wav&F3B18.wav&0&1&2&8&4.606&1.866&9.540&9.721\\ \hline
F9B24.wav&F7B8.wav&0&1&2&8&4.606&3.456&14.940&15.334\\ \hline
F14B26.wav&F12B10.wav&1&0&2&8&4.606&3.986&19.440&19.844\\ \hline
F1B2.wav&A2B18.wav&0&1&2&8&4.606&1.866&7.740&7.962\\ \hline
F13B2.wav&F8B6.wav&0&1&5&2&4.623&0.546&17.340&17.349\\ \hline
F10B28.wav&F15B24.wav&0&0&5&2&4.623&8.457&19.140&20.925\\ \hline
F8B20.wav&F3B24.wav&0&0&5&2&4.623&7.397&12.840&14.818\\ \hline
F9B24.wav&F14B20.wav&0&1&5&2&4.623&7.397&18.240&19.683\\ \hline
F11B6.wav&F16B2.wav&0&0&5&2&4.623&0.546&20.040&20.047\\ \hline
F3B18.wav&A1B22.wav&0&1&5&2&4.623&6.867&8.340&10.803\\ \hline
F1B14.wav&F6B18.wav&0&1&5&2&4.623&4.203&11.040&11.813\\ \hline
F10B8.wav&F15B4.wav&1&1&5&2&4.623&1.076&19.140&19.170\\ \hline
F7B14.wav&F12B10.wav&0&0&5&2&4.623&2.666&16.440&16.655\\ \hline
F12B18.wav&F7B22.wav&1&1&5&2&4.623&6.867&16.440&17.816\\ \hline
F10B28.wav&F15B24.wav&0&1&5&2&4.623&8.457&19.140&20.925\\ \hline
F13B22.wav&F8B26.wav&0&1&5&2&4.623&7.927&17.340&19.066\\ \hline
F7B22.wav&F2B26.wav&0&0&5&2&4.623&7.927&11.940&14.332\\ \hline
F15B10.wav&F10B16.wav&0&1&5&3&4.773&3.283&19.140&19.420\\ \hline
F3B18.wav&F8B12.wav&0&1&5&3&4.773&3.813&12.840&13.394\\ \hline
F10B14.wav&F5B20.wav&1&1&5&3&4.773&4.343&14.640&15.271\\ \hline
F5B26.wav&F1B14.wav&0&1&4&6&4.803&4.765&10.540&11.567\\ \hline
F12B24.wav&F8B12.wav&0&1&4&6&4.803&4.235&16.840&17.364\\ \hline
F13B16.wav&F9B4.wav&0&0&4&6&4.803&2.115&17.740&17.866\\ \hline
F1B14.wav&F5B26.wav&0&1&4&6&4.803&4.765&10.540&11.567\\ \hline
F11B26.wav&F10B8.wav&0&0&1&9&4.854&3.596&17.140&17.513\\ \hline
F10B22.wav&F9B4.wav&1&0&1&9&4.854&2.536&16.240&16.437\\ \hline
F2B20.wav&F1B2.wav&0&1&1&9&4.854&2.006&9.040&9.260\\ \hline
F15B10.wav&F16B28.wav&1&0&1&9&4.854&4.126&21.640&22.030\\ \hline
F10B8.wav&F11B26.wav&0&0&1&9&4.854&3.596&17.140&17.513\\ \hline
F10B8.wav&F11B26.wav&0&1&1&9&4.854&3.596&17.140&17.513\\ \hline
F4B8.wav&F5B26.wav&1&1&1&9&4.854&3.596&11.740&12.278\\ \hline
F10B28.wav&F9B10.wav&1&1&1&9&4.854&4.126&16.240&16.756\\ \hline
F2B26.wav&F1B8.wav&0&1&1&9&4.854&3.596&9.040&9.729\\ \hline
F9B4.wav&F10B22.wav&0&1&1&9&4.854&2.536&16.240&16.437\\ \hline
F14B20.wav&F13B2.wav&1&0&1&9&4.854&2.006&19.840&19.941\\ \hline
F9B18.wav&F14B26.wav&0&1&5&4&4.974&7.429&18.240&19.695\\ \hline
F3B18.wav&F8B26.wav&0&0&5&4&4.974&7.429&12.840&14.834\\ \hline
A2B10.wav&F4B2.wav&1&0&5&4&4.974&0.827&9.240&9.277\\ \hline
F2B20.wav&F7B28.wav&0&1&5&4&4.974&7.959&11.940&14.349\\ \hline
F7B28.wav&F2B20.wav&0&1&5&4&4.974&7.959&11.940&14.349\\ \hline
A1B16.wav&F3B24.wav&0&1&5&4&4.974&6.899&8.340&10.823\\ \hline
F3B4.wav&A3B20.wav&0&1&3&8&5.027&2.396&9.140&9.449\\ \hline
F8B26.wav&F10B8.wav&1&1&2&9&5.098&3.596&15.840&16.243\\ \hline
F12B4.wav&F10B22.wav&0&0&2&9&5.098&2.536&17.640&17.821\\ \hline
F3B4.wav&A2B18.wav&0&1&4&7&5.170&2.255&8.740&9.026\\ \hline
F6B24.wav&F1B14.wav&0&1&5&5&5.222&4.624&11.040&11.969\\ \hline
F4B28.wav&A2B18.wav&1&0&5&5&5.222&7.710&9.240&12.034\\ \hline
F11B12.wav&F16B22.wav&0&1&5&5&5.222&4.094&20.040&20.454\\ \hline
F2B12.wav&F7B22.wav&0&0&5&5&5.222&4.094&11.940&12.622\\ \hline
F6B4.wav&F1B14.wav&0&0&5&5&5.222&1.497&11.040&11.141\\ \hline
F6B4.wav&F1B14.wav&0&1&5&5&5.222&1.497&11.040&11.141\\ \hline
F12B4.wav&F7B14.wav&0&1&5&5&5.222&1.497&16.440&16.508\\ \hline
F8B12.wav&F13B2.wav&0&1&5&5&5.222&0.967&17.340&17.367\\ \hline
F7B2.wav&F2B12.wav&0&1&5&5&5.222&0.967&11.940&11.979\\ \hline
F9B18.wav&F4B28.wav&0&1&5&5&5.222&7.710&13.740&15.755\\ \hline
F5B12.wav&F10B2.wav&0&1&5&5&5.222&0.967&14.640&14.672\\ \hline
F10B16.wav&F5B6.wav&0&0&5&5&5.222&2.504&14.640&14.853\\ \hline
F10B16.wav&F5B26.wav&0&1&5&5&5.222&7.180&14.640&16.306\\ \hline
F10B28.wav&F10B8.wav&1&1&0&10&5.300&3.737&16.640&17.054\\ \hline
A1B2.wav&A1B22.wav&0&0&0&10&5.300&2.147&5.840&6.222\\ \hline
F13B28.wav&F13B8.wav&0&1&0&10&5.300&3.737&19.340&19.698\\ \hline
A1B22.wav&A1B2.wav&0&1&0&10&5.300&2.147&5.840&6.222\\ \hline
F9B10.wav&F3B10.wav&0&1&6&0&5.400&2.385&13.340&13.552\\ \hline
F2B26.wav&F8B26.wav&0&1&6&0&5.400&8.425&12.440&15.024\\ \hline
F15B24.wav&F9B24.wav&1&0&6&0&5.400&7.895&18.740&20.335\\ \hline
F11B12.wav&F5B12.wav&0&1&6&0&5.400&2.915&15.140&15.418\\ \hline
F13B2.wav&F7B2.wav&0&0&6&0&5.400&0.265&16.940&16.942\\ \hline
F15B18.wav&F9B18.wav&0&1&6&0&5.400&6.305&18.740&19.772\\ \hline
F7B16.wav&F13B16.wav&0&1&6&0&5.400&5.775&16.940&17.897\\ \hline
F4B16.wav&A1B16.wav&0&0&6&0&5.400&5.775&8.840&10.559\\ \hline
F16B16.wav&F10B16.wav&1&1&6&0&5.400&5.775&19.640&20.471\\ \hline
F13B14.wav&F7B16.wav&0&1&6&1&5.426&4.062&16.940&17.420\\ \hline
A2B4.wav&F5B6.wav&1&1&6&1&5.426&0.935&9.740&9.785\\ \hline
F1B26.wav&F7B22.wav&0&1&6&2&5.503&7.927&11.540&14.000\\ \hline
A2B10.wav&F5B6.wav&0&0&6&2&5.503&1.606&9.740&9.872\\ \hline
F6B24.wav&A3B20.wav&1&1&6&2&5.503&7.397&10.640&12.958\\ \hline
F10B16.wav&F15B4.wav&0&0&5&6&5.510&2.115&19.140&19.256\\ \hline
F10B22.wav&F15B10.wav&0&0&5&6&5.510&3.705&19.140&19.495\\ \hline
F11B20.wav&F16B8.wav&0&1&5&6&5.510&3.175&20.040&20.290\\ \hline
F10B22.wav&F15B10.wav&1&1&5&6&5.510&3.705&19.140&19.495\\ \hline
F16B14.wav&F11B26.wav&0&1&5&6&5.510&4.765&20.040&20.599\\ \hline
F11B6.wav&F7B22.wav&0&1&4&8&5.562&2.926&15.940&16.206\\ \hline
F8B20.wav&F12B4.wav&0&1&4&8&5.562&2.396&16.840&17.010\\ \hline
F12B24.wav&F16B8.wav&0&0&4&8&5.562&3.456&20.440&20.730\\ \hline
F8B12.wav&F4B28.wav&1&0&4&8&5.562&4.516&13.240&13.989\\ \hline
F2B20.wav&F6B4.wav&0&1&4&8&5.562&2.396&11.440&11.688\\ \hline
F3B18.wav&F7B2.wav&0&1&4&8&5.562&1.866&12.340&12.480\\ \hline
F11B20.wav&F15B4.wav&0&1&4&8&5.562&2.396&19.540&19.686\\ \hline
F9B24.wav&F13B8.wav&0&1&4&8&5.562&3.456&17.740&18.073\\ \hline
F11B6.wav&F5B12.wav&0&1&6&3&5.629&1.746&15.140&15.240\\ \hline
F10B8.wav&F16B2.wav&0&1&6&3&5.629&0.686&19.640&19.652\\ \hline
F1B8.wav&F7B14.wav&1&0&6&3&5.629&2.276&11.540&11.762\\ \hline
F10B8.wav&F16B14.wav&0&1&6&3&5.629&2.276&19.640&19.771\\ \hline
F4B28.wav&F10B22.wav&0&0&6&3&5.629&8.208&14.240&16.436\\ \hline
F2B6.wav&F8B12.wav&1&1&6&3&5.629&1.746&12.440&12.562\\ \hline
F10B22.wav&F4B16.wav&0&1&6&3&5.629&6.618&14.240&15.703\\ \hline
F1B8.wav&F7B2.wav&1&0&6&3&5.629&0.686&11.540&11.560\\ \hline
F15B24.wav&F9B18.wav&0&0&6&3&5.629&7.148&18.740&20.057\\ \hline
F10B22.wav&F4B16.wav&0&1&6&3&5.629&6.618&14.240&15.703\\ \hline
F1B14.wav&F7B8.wav&0&1&6&3&5.629&2.276&11.540&11.762\\ \hline
F4B22.wav&A1B16.wav&0&1&6&3&5.629&6.618&8.840&11.043\\ \hline
F10B8.wav&F16B14.wav&0&0&6&3&5.629&2.276&19.640&19.771\\ \hline
F13B8.wav&F7B2.wav&0&1&6&3&5.629&0.686&16.940&16.954\\ \hline
F8B26.wav&F2B20.wav&0&1&6&3&5.629&7.678&12.440&14.619\\ \hline
A1B22.wav&F4B28.wav&1&1&6&3&5.629&8.208&8.840&12.063\\ \hline
A1B22.wav&F4B28.wav&0&1&6&3&5.629&8.208&8.840&12.063\\ \hline
F10B22.wav&F16B14.wav&0&0&6&4&5.801&4.484&19.640&20.145\\ \hline
F4B16.wav&F10B8.wav&0&1&6&4&5.801&2.894&14.240&14.531\\ \hline
F10B22.wav&F16B14.wav&0&1&6&4&5.801&4.484&19.640&20.145\\ \hline
F2B12.wav&F8B20.wav&1&1&6&4&5.801&3.954&12.440&13.053\\ \hline
F16B8.wav&F10B16.wav&1&0&6&4&5.801&2.894&19.640&19.852\\ \hline
F11B12.wav&F5B20.wav&0&1&6&4&5.801&3.954&15.140&15.648\\ \hline
F8B12.wav&F14B20.wav&0&0&6&4&5.801&3.954&17.840&18.273\\ \hline
F8B20.wav&F2B12.wav&1&0&6&4&5.801&3.954&12.440&13.053\\ \hline
F4B8.wav&F10B16.wav&1&0&6&4&5.801&2.894&14.240&14.531\\ \hline
F8B6.wav&F7B28.wav&0&1&1&11&5.899&3.347&14.440&14.823\\ \hline
F3B24.wav&F4B2.wav&1&1&1&11&5.899&2.287&10.840&11.079\\ \hline
F12B24.wav&F13B2.wav&0&0&1&11&5.899&2.287&18.940&19.078\\ \hline
F5B6.wav&F4B28.wav&0&1&1&11&5.899&3.347&11.740&12.208\\ \hline
F11B26.wav&F12B4.wav&0&0&1&11&5.899&2.817&18.040&18.259\\ \hline
F9B24.wav&F10B2.wav&0&0&1&11&5.899&2.287&16.240&16.400\\ \hline
F12B24.wav&F13B2.wav&0&1&1&11&5.899&2.287&18.940&19.078\\ \hline
F15B4.wav&F14B26.wav&1&0&1&11&5.899&2.817&20.740&20.930\\ \hline
F15B24.wav&F16B2.wav&1&1&1&11&5.899&2.287&21.640&21.761\\ \hline
F13B8.wav&F10B28.wav&0&0&3&10&5.948&3.737&18.140&18.521\\ \hline
F14B6.wav&F11B26.wav&0&1&3&10&5.948&3.207&19.040&19.308\\ \hline
F9B24.wav&F12B4.wav&1&1&3&10&5.948&2.677&17.240&17.447\\ \hline
F7B22.wav&F10B2.wav&0&1&3&10&5.948&2.147&15.440&15.588\\ \hline
F15B24.wav&F12B4.wav&0&1&3&10&5.948&2.677&19.940&20.119\\ \hline
F9B4.wav&F6B24.wav&1&1&3&10&5.948&2.677&14.540&14.784\\ \hline
F9B24.wav&F5B6.wav&0&1&4&9&5.976&3.066&14.140&14.469\\ \hline
F9B24.wav&F5B6.wav&0&1&4&9&5.976&3.066&14.140&14.469\\ \hline
F14B20.wav&F10B2.wav&1&1&4&9&5.976&2.006&18.640&18.748\\ \hline
F8B6.wav&F12B24.wav&0&1&4&9&5.976&3.066&16.840&17.117\\ \hline
F10B22.wav&F6B4.wav&1&1&4&9&5.976&2.536&15.040&15.252\\ \hline
F3B10.wav&F7B28.wav&0&0&4&9&5.976&4.126&12.340&13.012\\ \hline
F11B6.wav&F13B28.wav&0&0&2&11&6.102&3.347&18.540&18.840\\ \hline
F8B26.wav&F6B4.wav&1&1&2&11&6.102&2.817&14.040&14.320\\ \hline
F6B4.wav&F8B26.wav&1&0&2&11&6.102&2.817&14.040&14.320\\ \hline
F16B28.wav&F11B12.wav&1&0&5&8&6.183&4.516&20.040&20.542\\ \hline
F13B22.wav&F8B6.wav&0&1&5&8&6.183&2.926&17.340&17.585\\ \hline
F12B24.wav&F7B8.wav&0&1&5&8&6.183&3.456&16.440&16.799\\ \hline
F5B6.wav&F10B22.wav&0&1&5&8&6.183&2.926&14.640&14.929\\ \hline
F6B4.wav&F11B20.wav&0&1&5&8&6.183&2.396&15.540&15.724\\ \hline
F13B28.wav&F8B12.wav&0&1&5&8&6.183&4.516&17.340&17.918\\ \hline
F1B2.wav&F7B14.wav&0&1&6&6&6.267&1.108&11.540&11.593\\ \hline
F7B14.wav&F1B2.wav&0&1&6&6&6.267&1.108&11.540&11.593\\ \hline
F10B28.wav&F4B16.wav&0&1&6&6&6.267&7.460&14.240&16.076\\ \hline
F16B28.wav&F10B16.wav&1&0&6&6&6.267&7.460&19.640&21.009\\ \hline
A1B16.wav&F4B28.wav&0&1&6&6&6.267&7.460&8.840&11.567\\ \hline
F1B20.wav&F7B8.wav&1&0&6&6&6.267&3.175&11.540&11.969\\ \hline
A2B18.wav&F6B18.wav&0&1&7&0&6.300&6.305&10.240&12.025\\ \hline
F8B12.wav&F15B10.wav&1&1&7&1&6.322&2.525&18.340&18.513\\ \hline
F14B20.wav&F7B22.wav&0&1&7&1&6.322&7.116&17.440&18.836\\ \hline
F1B14.wav&F8B12.wav&0&1&7&1&6.322&3.055&12.040&12.422\\ \hline
F8B26.wav&F15B24.wav&0&0&7&1&6.322&8.176&18.340&20.080\\ \hline
F6B24.wav&F13B22.wav&0&1&7&1&6.322&7.646&16.540&18.222\\ \hline
F4B14.wav&F11B12.wav&0&1&7&1&6.322&3.055&14.740&15.053\\ \hline
F2B6.wav&F9B10.wav&0&1&7&2&6.389&1.606&12.940&13.039\\ \hline
F4B16.wav&F11B12.wav&0&1&7&2&6.389&3.673&14.740&15.191\\ \hline
F8B12.wav&F1B8.wav&0&0&7&2&6.389&2.136&12.040&12.228\\ \hline
F16B28.wav&F9B24.wav&0&1&7&2&6.389&8.457&19.240&21.017\\ \hline
F2B20.wav&F9B24.wav&1&1&7&2&6.389&7.397&12.940&14.905\\ \hline
F5B20.wav&F12B24.wav&0&1&7&2&6.389&7.397&15.640&17.301\\ \hline
F11B6.wav&F4B2.wav&0&1&7&2&6.389&0.546&14.740&14.750\\ \hline
F14B6.wav&F7B2.wav&0&1&7&2&6.389&0.546&17.440&17.449\\ \hline
F15B4.wav&F16B28.wav&0&1&1&12&6.423&2.957&21.640&21.841\\ \hline
F13B2.wav&F14B26.wav&0&0&1&12&6.423&2.427&19.840&19.988\\ \hline
F6B4.wav&F7B28.wav&0&1&1&12&6.423&2.957&13.540&13.859\\ \hline
F7B28.wav&F6B4.wav&0&0&1&12&6.423&2.957&13.540&13.859\\ \hline
A2B18.wav&F6B24.wav&0&1&7&3&6.498&7.148&10.240&12.488\\ \hline
F8B26.wav&F1B20.wav&0&1&7&3&6.498&7.678&12.040&14.280\\ \hline
F8B20.wav&F1B14.wav&0&1&7&3&6.498&4.343&12.040&12.799\\ \hline
F11B20.wav&F4B14.wav&0&1&7&3&6.498&4.343&14.740&15.367\\ \hline
F7B14.wav&F14B20.wav&0&1&7&3&6.498&4.343&17.440&17.973\\ \hline
F10B28.wav&F4B14.wav&0&1&6&7&6.552&4.905&14.240&15.061\\ \hline
F10B16.wav&F16B2.wav&0&1&6&7&6.552&1.725&19.640&19.716\\ \hline
F8B26.wav&F2B12.wav&0&0&6&7&6.552&4.375&12.440&13.187\\ \hline
F8B20.wav&F2B6.wav&1&1&6&7&6.552&2.785&12.440&12.748\\ \hline
F4B28.wav&F10B14.wav&0&1&6&7&6.552&4.905&14.240&15.061\\ \hline
F9B24.wav&F15B10.wav&1&0&6&7&6.552&3.845&18.740&19.130\\ \hline
F4B2.wav&A1B16.wav&0&1&6&7&6.552&1.725&8.840&9.007\\ \hline
F10B28.wav&F4B14.wav&0&1&6&7&6.552&4.905&14.240&15.061\\ \hline
F11B20.wav&F5B6.wav&0&0&6&7&6.552&2.785&15.140&15.394\\ \hline
F9B18.wav&F15B4.wav&0&0&6&7&6.552&2.255&18.740&18.875\\ \hline
F14B6.wav&F8B20.wav&0&1&6&7&6.552&2.785&17.840&18.056\\ \hline
F16B28.wav&F10B14.wav&0&1&6&7&6.552&4.905&19.640&20.243\\ \hline
F9B24.wav&F15B10.wav&0&1&6&7&6.552&3.845&18.740&19.130\\ \hline
F9B18.wav&F15B4.wav&0&1&6&7&6.552&2.255&18.740&18.875\\ \hline
F5B6.wav&F11B20.wav&0&1&6&7&6.552&2.785&15.140&15.394\\ \hline
F16B22.wav&F10B8.wav&0&0&6&7&6.552&3.315&19.640&19.918\\ \hline
F7B8.wav&F2B26.wav&0&1&5&9&6.558&3.596&11.940&12.470\\ \hline
F10B8.wav&F5B26.wav&0&1&5&9&6.558&3.596&14.640&15.075\\ \hline
F10B22.wav&F15B4.wav&0&1&5&9&6.558&2.536&19.140&19.307\\ \hline
F4B28.wav&A2B10.wav&0&0&5&9&6.558&4.126&9.240&10.119\\ \hline
F9B4.wav&F7B28.wav&1&1&2&12&6.610&2.957&14.940&15.230\\ \hline
F10B2.wav&F3B10.wav&0&1&7&4&6.647&0.827&13.840&13.865\\ \hline
F5B12.wav&F12B4.wav&0&1&7&4&6.647&1.357&15.640&15.699\\ \hline
F6B10.wav&F13B2.wav&0&1&7&4&6.647&0.827&16.540&16.561\\ \hline
A2B10.wav&F6B18.wav&0&1&7&4&6.647&3.424&10.240&10.797\\ \hline
F5B26.wav&A1B16.wav&0&1&7&5&6.835&7.180&9.340&11.781\\ \hline
F9B4.wav&F16B14.wav&0&1&7&5&6.835&1.497&19.240&19.298\\ \hline
F11B12.wav&F4B2.wav&0&1&7&5&6.835&0.967&14.740&14.772\\ \hline
F4B16.wav&F11B26.wav&0&1&7&5&6.835&7.180&14.740&16.396\\ \hline
F5B26.wav&A1B16.wav&0&1&7&5&6.835&7.180&9.340&11.781\\ \hline
F13B8.wav&F6B18.wav&0&1&7&5&6.835&3.034&16.540&16.816\\ \hline
F9B4.wav&F16B14.wav&1&0&7&5&6.835&1.497&19.240&19.298\\ \hline
F3B18.wav&F10B8.wav&0&1&7&5&6.835&3.034&13.840&14.169\\ \hline
F16B28.wav&F9B18.wav&0&1&7&5&6.835&7.710&19.240&20.727\\ \hline
F5B20.wav&F12B10.wav&0&1&7&5&6.835&3.564&15.640&16.041\\ \hline
F7B2.wav&F3B24.wav&0&1&4&11&6.852&2.287&12.340&12.550\\ \hline
F7B2.wav&F7B28.wav&0&1&0&13&6.890&2.568&13.940&14.175\\ \hline
F13B28.wav&F13B2.wav&0&1&0&13&6.890&2.568&19.340&19.510\\ \hline
F1B26.wav&A1B2.wav&0&1&3&12&6.909&2.427&7.340&7.731\\ \hline
F7B14.wav&F14B26.wav&0&1&7&6&7.057&4.765&17.440&18.079\\ \hline
F5B20.wav&A1B8.wav&0&1&7&6&7.057&3.175&9.340&9.865\\ \hline
F1B8.wav&F8B20.wav&0&1&7&6&7.057&3.175&12.040&12.452\\ \hline
F10B22.wav&F3B10.wav&0&1&7&6&7.057&3.705&13.840&14.327\\ \hline
F2B6.wav&F9B18.wav&0&1&7&6&7.057&2.645&12.940&13.207\\ \hline
F6B4.wav&F13B16.wav&0&1&7&6&7.057&2.115&16.540&16.675\\ \hline
F12B24.wav&F5B12.wav&1&1&7&6&7.057&4.235&15.640&16.203\\ \hline
F8B26.wav&F1B14.wav&0&1&7&6&7.057&4.765&12.040&12.949\\ \hline
F13B8.wav&F5B6.wav&0&1&8&1&7.219&1.465&16.140&16.206\\ \hline
F12B10.wav&F4B8.wav&0&0&8&1&7.219&1.995&15.240&15.370\\ \hline
F10B28.wav&F2B26.wav&0&1&8&1&7.219&8.706&13.440&16.013\\ \hline
F5B26.wav&F13B28.wav&0&1&8&1&7.219&8.706&16.140&18.338\\ \hline
A2B4.wav&F7B2.wav&0&1&8&1&7.219&0.405&10.740&10.748\\ \hline
F3B10.wav&F11B12.wav&0&1&8&1&7.219&2.525&14.340&14.561\\ \hline
F8B26.wav&F16B22.wav&0&1&8&2&7.278&7.927&18.840&20.440\\ \hline
F1B14.wav&F9B18.wav&1&1&8&2&7.278&4.203&12.540&13.226\\ \hline
F2B26.wav&F10B22.wav&0&1&8&2&7.278&7.927&13.440&15.603\\ \hline
F2B12.wav&F10B8.wav&0&1&8&2&7.278&2.136&13.440&13.609\\ \hline
F7B14.wav&A2B10.wav&0&0&8&2&7.278&2.666&10.740&11.066\\ \hline
F14B20.wav&F6B24.wav&0&1&8&2&7.278&7.397&17.040&18.576\\ \hline
F3B10.wav&F11B6.wav&0&1&8&2&7.278&1.606&14.340&14.430\\ \hline
A1B8.wav&F6B4.wav&0&1&8&2&7.278&1.076&9.840&9.899\\ \hline
F15B18.wav&F7B14.wav&1&1&8&2&7.278&4.203&17.940&18.426\\ \hline
F10B28.wav&F6B4.wav&0&0&4&12&7.308&2.957&15.040&15.328\\ \hline
F6B4.wav&F10B28.wav&0&1&4&12&7.308&2.957&15.040&15.328\\ \hline
F4B2.wav&F9B24.wav&1&0&5&11&7.365&2.287&13.740&13.929\\ \hline
F11B6.wav&F16B28.wav&0&1&5&11&7.365&3.347&20.040&20.318\\ \hline
F8B6.wav&F13B28.wav&0&1&5&11&7.365&3.347&17.340&17.660\\ \hline
F1B2.wav&F6B24.wav&0&0&5&11&7.365&2.287&11.040&11.274\\ \hline
F6B24.wav&F1B2.wav&0&1&5&11&7.365&2.287&11.040&11.274\\ \hline
F7B16.wav&A2B10.wav&0&1&8&3&7.373&3.283&10.740&11.231\\ \hline
F7B2.wav&A2B10.wav&0&1&8&4&7.506&0.827&10.740&10.772\\ \hline
F1B26.wav&F9B18.wav&0&1&8&4&7.506&7.429&12.540&14.575\\ \hline
F16B28.wav&F8B20.wav&1&1&8&4&7.506&7.959&18.840&20.452\\ \hline
F11B26.wav&F3B18.wav&0&1&8&4&7.506&7.429&14.340&16.150\\ \hline
F4B2.wav&A1B22.wav&0&1&6&10&7.566&2.147&8.840&9.097\\ \hline
F4B28.wav&A1B8.wav&0&1&6&10&7.566&3.737&8.840&9.597\\ \hline
F6B4.wav&F12B24.wav&0&1&6&10&7.566&2.677&16.040&16.262\\ \hline
F9B24.wav&F3B4.wav&1&1&6&10&7.566&2.677&13.340&13.606\\ \hline
A1B22.wav&F4B2.wav&0&1&6&10&7.566&2.147&8.840&9.097\\ \hline
F4B28.wav&F11B12.wav&0&1&7&8&7.594&4.516&14.740&15.416\\ \hline
F5B6.wav&A1B22.wav&0&1&7&8&7.594&2.926&9.340&9.787\\ \hline
F9B24.wav&F16B8.wav&0&1&7&8&7.594&3.456&19.240&19.548\\ \hline
F2B20.wav&F9B4.wav&0&1&7&8&7.594&2.396&12.940&13.160\\ \hline
F4B28.wav&F11B12.wav&1&1&7&8&7.594&4.516&14.740&15.416\\ \hline
F4B28.wav&F11B12.wav&0&1&7&8&7.594&4.516&14.740&15.416\\ \hline
F3B24.wav&F10B8.wav&0&1&7&8&7.594&3.456&13.840&14.265\\ \hline
F7B14.wav&A2B4.wav&0&1&8&5&7.672&1.497&10.740&10.844\\ \hline
F7B28.wav&A2B18.wav&0&1&8&5&7.672&7.710&10.740&13.221\\ \hline
F7B14.wav&A2B4.wav&0&1&8&5&7.672&1.497&10.740&10.844\\ \hline
F7B28.wav&F15B18.wav&0&1&8&5&7.672&7.710&17.940&19.526\\ \hline
F2B6.wav&F10B16.wav&0&1&8&5&7.672&2.504&13.440&13.671\\ \hline
F7B14.wav&F15B4.wav&0&1&8&5&7.672&1.497&17.940&18.002\\ \hline
F10B16.wav&F2B26.wav&0&1&8&5&7.672&7.180&13.440&15.237\\ \hline
F16B22.wav&F8B12.wav&0&1&8&5&7.672&4.094&18.840&19.280\\ \hline
F5B26.wav&F10B2.wav&0&1&5&12&7.791&2.427&14.640&14.840\\ \hline
F16B2.wav&F11B26.wav&1&1&5&12&7.791&2.427&20.040&20.186\\ \hline
F3B24.wav&F11B12.wav&0&1&8&6&7.871&4.235&14.340&14.952\\ \hline
F10B8.wav&F2B20.wav&0&1&8&6&7.871&3.175&13.440&13.810\\ \hline
F13B14.wav&F5B26.wav&0&1&8&6&7.871&4.765&16.140&16.829\\ \hline
F4B16.wav&F12B4.wav&0&1&8&6&7.871&2.115&15.240&15.386\\ \hline
F6B18.wav&F14B6.wav&0&1&8&6&7.871&2.645&17.040&17.244\\ \hline
F2B20.wav&F10B8.wav&0&1&8&6&7.871&3.175&13.440&13.810\\ \hline
F11B20.wav&F4B2.wav&1&0&7&9&7.902&2.006&14.740&14.876\\ \hline
F14B26.wav&F7B8.wav&0&1&7&9&7.902&3.596&17.440&17.807\\ \hline
F7B2.wav&F14B20.wav&0&1&7&9&7.902&2.006&17.440&17.555\\ \hline
F11B20.wav&F4B2.wav&0&1&7&9&7.902&2.006&14.740&14.876\\ \hline
F4B8.wav&F11B26.wav&0&1&7&9&7.902&3.596&14.740&15.172\\ \hline
F5B26.wav&A1B8.wav&0&1&7&9&7.902&3.596&9.340&10.008\\ \hline
F12B24.wav&F5B6.wav&0&1&7&9&7.902&3.066&15.640&15.938\\ \hline
F8B6.wav&A3B20.wav&0&1&8&7&8.100&2.785&11.640&11.969\\ \hline
F6B18.wav&F15B18.wav&0&1&9&0&8.100&6.305&17.540&18.639\\ \hline
F5B26.wav&F14B26.wav&0&0&9&0&8.100&8.425&16.640&18.651\\ \hline
F1B14.wav&F10B14.wav&1&1&9&0&8.100&3.445&13.040&13.487\\ \hline
F7B8.wav&F16B8.wav&0&0&9&0&8.100&1.855&18.440&18.533\\ \hline
F15B24.wav&F6B24.wav&0&1&9&0&8.100&7.895&17.540&19.235\\ \hline
F16B22.wav&F7B22.wav&1&0&9&0&8.100&7.365&18.440&19.856\\ \hline
F4B22.wav&F13B22.wav&0&0&9&0&8.100&7.365&15.740&17.378\\ \hline
F11B26.wav&F2B26.wav&0&1&9&0&8.100&8.425&13.940&16.288\\ \hline
F14B26.wav&F5B26.wav&0&1&9&0&8.100&8.425&16.640&18.651\\ \hline
F4B16.wav&F13B16.wav&0&1&9&0&8.100&5.775&15.740&16.766\\ \hline
F4B28.wav&F13B28.wav&0&1&9&0&8.100&8.955&15.740&18.109\\ \hline
F8B12.wav&A2B10.wav&0&1&9&1&8.117&2.525&11.240&11.520\\ \hline
F1B14.wav&F10B16.wav&0&1&9&1&8.117&4.062&13.040&13.658\\ \hline
F1B2.wav&F10B8.wav&0&1&9&3&8.255&0.686&13.040&13.058\\ \hline
F14B26.wav&F5B20.wav&0&1&9&3&8.255&7.678&16.640&18.326\\ \hline
F7B28.wav&F16B22.wav&0&1&9&3&8.255&8.208&18.440&20.184\\ \hline
F10B14.wav&F1B8.wav&0&1&9&3&8.255&2.276&13.040&13.237\\ \hline
F15B18.wav&F6B24.wav&0&1&9&3&8.255&7.148&17.540&18.940\\ \hline
F12B10.wav&F3B4.wav&0&1&9&3&8.255&1.216&14.840&14.890\\ \hline
F14B26.wav&F5B20.wav&0&1&9&3&8.255&7.678&16.640&18.326\\ \hline
F1B20.wav&F10B14.wav&0&1&9&3&8.255&4.343&13.040&13.744\\ \hline
F7B2.wav&A1B8.wav&0&1&9&3&8.255&0.686&10.340&10.363\\ \hline
F5B20.wav&F14B26.wav&0&1&9&3&8.255&7.678&16.640&18.326\\ \hline
F16B8.wav&F7B14.wav&0&0&9&3&8.255&2.276&18.440&18.580\\ \hline
F15B24.wav&F6B18.wav&0&1&9&3&8.255&7.148&17.540&18.940\\ \hline
F2B26.wav&F11B20.wav&0&0&9&3&8.255&7.678&13.940&15.914\\ \hline
F14B12.wav&F5B6.wav&0&1&9&3&8.255&1.746&16.640&16.731\\ \hline
F7B8.wav&F15B24.wav&0&1&8&8&8.356&3.456&17.940&18.270\\ \hline
F2B12.wav&F10B28.wav&1&0&8&8&8.356&4.516&13.440&14.178\\ \hline
F15B24.wav&F7B8.wav&0&1&8&8&8.356&3.456&17.940&18.270\\ \hline
F16B22.wav&F7B14.wav&0&1&9&4&8.373&4.484&18.440&18.977\\ \hline
F7B14.wav&F16B22.wav&0&1&9&4&8.373&4.484&18.440&18.977\\ \hline
F1B14.wav&F10B22.wav&0&1&9&4&8.373&4.484&13.040&13.789\\ \hline
F5B20.wav&F13B2.wav&0&1&8&9&8.637&2.006&16.140&16.264\\ \hline
F12B10.wav&F4B28.wav&0&0&8&9&8.637&4.126&15.240&15.789\\ \hline
F5B20.wav&F13B2.wav&0&1&8&9&8.637&2.006&16.140&16.264\\ \hline
F1B2.wav&F10B14.wav&0&0&9&6&8.702&1.108&13.040&13.087\\ \hline
F10B14.wav&F1B2.wav&0&1&9&6&8.702&1.108&13.040&13.087\\ \hline
F7B14.wav&F16B2.wav&0&1&9&6&8.702&1.108&18.440&18.473\\ \hline
F10B14.wav&F1B26.wav&0&1&9&6&8.702&4.765&13.040&13.883\\ \hline
F7B28.wav&F13B2.wav&0&1&6&13&8.754&2.568&16.940&17.134\\ \hline
F4B28.wav&A1B2.wav&0&1&6&13&8.754&2.568&8.840&9.205\\ \hline
F13B2.wav&F4B16.wav&0&1&9&7&8.909&1.725&15.740&15.834\\ \hline
F2B20.wav&F11B6.wav&0&1&9&7&8.909&2.785&13.940&14.216\\ \hline
F13B22.wav&F4B8.wav&0&0&9&7&8.909&3.315&15.740&16.085\\ \hline
F7B14.wav&F16B28.wav&0&1&9&7&8.909&4.905&18.440&19.081\\ \hline
F12B10.wav&F3B24.wav&0&1&9&7&8.909&3.845&14.840&15.330\\ \hline
F7B8.wav&A1B22.wav&0&1&9&7&8.909&3.315&10.340&10.858\\ \hline
F7B16.wav&A1B2.wav&0&1&9&7&8.909&1.725&10.340&10.483\\ \hline
F11B26.wav&F4B2.wav&0&0&7&12&8.952&2.427&14.740&14.939\\ \hline
F10B28.wav&F3B4.wav&1&0&7&12&8.952&2.957&13.840&14.152\\ \hline
F11B26.wav&F1B26.wav&0&1&10&0&9.000&8.425&13.540&15.947\\ \hline
F8B20.wav&A1B22.wav&1&1&10&1&9.016&7.116&10.840&12.967\\ \hline
F4B28.wav&F14B26.wav&0&1&10&1&9.016&8.706&16.240&18.426\\ \hline
F13B16.wav&F3B18.wav&0&1&10&1&9.016&6.056&15.340&16.492\\ \hline
F4B22.wav&F14B20.wav&0&1&10&1&9.016&7.116&16.240&17.731\\ \hline
F4B8.wav&F14B6.wav&0&1&10&1&9.016&1.465&16.240&16.306\\ \hline
F6B10.wav&F16B14.wav&1&1&10&2&9.062&2.666&18.040&18.236\\ \hline
F16B14.wav&F6B10.wav&0&1&10&2&9.062&2.666&18.040&18.236\\ \hline
F2B20.wav&F12B24.wav&0&1&10&2&9.062&7.397&14.440&16.224\\ \hline
F15B10.wav&F5B6.wav&0&0&10&2&9.062&1.606&17.140&17.215\\ \hline
F6B10.wav&F16B14.wav&0&0&10&2&9.062&2.666&18.040&18.236\\ \hline
F6B10.wav&F16B14.wav&0&1&10&2&9.062&2.666&18.040&18.236\\ \hline
F9B4.wav&A2B10.wav&0&1&10&3&9.139&1.216&11.740&11.803\\ \hline
F5B12.wav&F15B18.wav&0&1&10&3&9.139&3.813&17.140&17.559\\ \hline
F1B14.wav&F11B20.wav&0&1&10&3&9.139&4.343&13.540&14.220\\ \hline
F11B26.wav&F1B20.wav&0&0&10&3&9.139&7.678&13.540&15.565\\ \hline
F1B20.wav&F11B26.wav&0&1&10&3&9.139&7.678&13.540&15.565\\ \hline
F12B18.wav&F2B12.wav&1&0&10&3&9.139&3.813&14.440&14.935\\ \hline
A3B20.wav&F10B28.wav&0&1&10&4&9.246&7.959&12.640&14.937\\ \hline
F16B2.wav&F6B10.wav&0&0&10&4&9.246&0.827&18.040&18.059\\ \hline
F16B2.wav&F6B10.wav&0&1&10&4&9.246&0.827&18.040&18.059\\ \hline
A2B18.wav&F9B10.wav&0&1&10&4&9.246&3.424&11.740&12.229\\ \hline
F6B24.wav&F16B16.wav&0&1&10&4&9.246&6.899&18.040&19.314\\ \hline
A3B20.wav&F10B28.wav&0&1&10&4&9.246&7.959&12.640&14.937\\ \hline
F10B28.wav&A3B20.wav&0&1&10&4&9.246&7.959&12.640&14.937\\ \hline
F2B12.wav&F12B4.wav&0&0&10&4&9.246&1.357&14.440&14.504\\ \hline
F5B26.wav&F15B18.wav&0&1&10&4&9.246&7.429&17.140&18.681\\ \hline
F6B24.wav&A1B2.wav&0&0&8&11&9.264&2.287&9.840&10.102\\ \hline
F5B6.wav&F13B28.wav&0&1&8&11&9.264&3.347&16.140&16.483\\ \hline
F3B18.wav&F13B8.wav&0&0&10&5&9.382&3.034&15.340&15.637\\ \hline
F6B18.wav&F16B28.wav&0&1&10&5&9.382&7.710&18.040&19.618\\ \hline
A1B16.wav&F8B6.wav&0&1&10&5&9.382&2.504&10.840&11.126\\ \hline
F3B18.wav&F13B28.wav&0&1&10&5&9.382&7.710&15.340&17.168\\ \hline
F1B26.wav&F10B8.wav&0&1&9&9&9.400&3.596&13.040&13.527\\ \hline
F14B26.wav&F4B14.wav&0&1&10&6&9.545&4.765&16.240&16.925\\ \hline
F2B26.wav&F11B6.wav&0&1&9&10&9.680&3.207&13.940&14.304\\ \hline
F4B22.wav&F13B2.wav&0&1&9&10&9.680&2.147&15.740&15.886\\ \hline
F10B28.wav&F1B8.wav&0&1&9&10&9.680&3.737&13.040&13.565\\ \hline
F7B2.wav&A1B22.wav&0&1&9&10&9.680&2.147&10.340&10.560\\ \hline
F12B24.wav&F3B4.wav&0&1&9&10&9.680&2.677&14.840&15.079\\ \hline
F16B22.wav&F7B2.wav&0&0&9&10&9.680&2.147&18.440&18.565\\ \hline
F15B4.wav&F6B24.wav&0&1&9&10&9.680&2.677&17.540&17.743\\ \hline
F11B6.wav&F2B26.wav&0&1&9&10&9.680&3.207&13.940&14.304\\ \hline
F15B4.wav&F6B24.wav&0&1&9&10&9.680&2.677&17.540&17.743\\ \hline
F10B22.wav&F1B2.wav&0&1&9&10&9.680&2.147&13.040&13.215\\ \hline
F5B26.wav&F14B6.wav&0&1&9&10&9.680&3.207&16.640&16.946\\ \hline
F1B8.wav&F10B28.wav&0&1&9&10&9.680&3.737&13.040&13.565\\ \hline
A2B10.wav&F9B24.wav&0&1&10&7&9.735&3.845&11.740&12.354\\ \hline
F9B4.wav&A2B18.wav&0&1&10&7&9.735&2.255&11.740&11.955\\ \hline
F4B8.wav&F15B10.wav&0&1&11&1&9.914&1.995&16.740&16.859\\ \hline
F9B4.wav&A1B2.wav&0&1&11&1&9.914&0.405&11.340&11.347\\ \hline
A1B22.wav&F9B24.wav&0&1&11&1&9.914&7.646&11.340&13.677\\ \hline
F10B2.wav&A2B4.wav&0&1&11&1&9.914&0.405&12.240&12.247\\ \hline
A2B18.wav&F10B16.wav&0&1&11&1&9.914&6.056&12.240&13.656\\ \hline
F13B2.wav&F3B18.wav&0&1&10&8&9.949&1.866&15.340&15.453\\ \hline
F2B12.wav&F13B16.wav&0&1&11&2&9.957&3.673&14.940&15.385\\ \hline
F1B20.wav&F12B24.wav&0&1&11&2&9.957&7.397&14.040&15.869\\ \hline
F12B4.wav&F1B8.wav&0&1&11&2&9.957&1.076&14.040&14.081\\ \hline
F2B20.wav&F13B16.wav&0&1&11&2&9.957&6.337&14.940&16.228\\ \hline
F9B18.wav&A1B22.wav&0&1&11&2&9.957&6.867&11.340&13.257\\ \hline
F16B16.wav&F5B12.wav&0&1&11&2&9.957&3.673&17.640&18.018\\ \hline
F16B2.wav&F5B6.wav&0&1&11&2&9.957&0.546&17.640&17.648\\ \hline
F5B20.wav&F16B14.wav&0&1&11&3&10.027&4.343&17.640&18.167\\ \hline
F4B2.wav&F15B10.wav&0&1&11&4&10.124&0.827&16.740&16.760\\ \hline
A1B2.wav&F9B10.wav&0&1&11&4&10.124&0.827&11.340&11.370\\ \hline
F4B8.wav&F14B26.wav&0&1&10&9&10.186&3.596&16.240&16.633\\ \hline
F6B4.wav&F16B22.wav&0&1&10&9&10.186&2.536&18.040&18.217\\ \hline
F13B22.wav&F3B4.wav&0&1&10&9&10.186&2.536&15.340&15.548\\ \hline
F10B2.wav&A3B20.wav&0&1&10&9&10.186&2.006&12.640&12.798\\ \hline
F5B12.wav&F16B22.wav&0&1&11&5&10.249&4.094&17.640&18.109\\ \hline
F16B22.wav&F5B12.wav&0&1&11&5&10.249&4.094&17.640&18.109\\ \hline
F2B26.wav&F13B16.wav&0&1&11&5&10.249&7.180&14.940&16.576\\ \hline
F5B12.wav&F16B2.wav&0&0&11&5&10.249&0.967&17.640&17.666\\ \hline
F1B14.wav&F12B4.wav&0&1&11&5&10.249&1.497&14.040&14.120\\ \hline
A1B22.wav&F9B10.wav&0&1&11&6&10.398&3.705&11.340&11.930\\ \hline
A2B4.wav&F10B16.wav&0&1&11&6&10.398&2.115&12.240&12.421\\ \hline
F13B8.wav&F2B20.wav&0&1&11&6&10.398&3.175&14.940&15.274\\ \hline
A2B10.wav&F10B22.wav&0&1&11&6&10.398&3.705&12.240&12.788\\ \hline
F5B12.wav&F16B28.wav&0&1&11&8&10.770&4.516&17.640&18.209\\ \hline
F4B8.wav&F15B24.wav&0&1&11&8&10.770&3.456&16.740&17.093\\ \hline
F5B12.wav&F16B28.wav&0&1&11&8&10.770&4.516&17.640&18.209\\ \hline
F1B20.wav&F12B4.wav&0&1&11&8&10.770&2.396&14.040&14.243\\ \hline
F1B8.wav&F13B8.wav&0&1&12&0&10.800&1.855&14.540&14.658\\ \hline
F15B24.wav&F3B24.wav&0&1&12&0&10.800&7.895&16.340&18.147\\ \hline
F4B8.wav&F16B8.wav&0&1&12&0&10.800&1.855&17.240&17.340\\ \hline
F4B16.wav&F16B16.wav&0&1&12&0&10.800&5.775&17.240&18.182\\ \hline
F4B16.wav&F16B16.wav&1&1&12&0&10.800&5.775&17.240&18.182\\ \hline
A1B8.wav&F10B8.wav&0&0&12&0&10.800&1.855&11.840&11.984\\ \hline
F13B16.wav&F1B14.wav&0&1&12&1&10.813&4.062&14.540&15.097\\ \hline
F16B16.wav&F4B14.wav&0&1&12&1&10.813&4.062&17.240&17.712\\ \hline
A2B18.wav&F11B20.wav&0&1&12&1&10.813&6.586&12.740&14.342\\ \hline
F4B16.wav&F16B14.wav&0&1&12&1&10.813&4.062&17.240&17.712\\ \hline
A2B10.wav&F11B6.wav&0&1&12&2&10.852&1.606&12.740&12.841\\ \hline
F12B24.wav&A3B20.wav&0&1&12&2&10.852&7.397&13.640&15.517\\ \hline
A2B10.wav&F11B6.wav&0&1&12&2&10.852&1.606&12.740&12.841\\ \hline
F15B18.wav&F3B24.wav&0&1&12&3&10.916&7.148&16.340&17.835\\ \hline
F16B28.wav&F4B22.wav&0&1&12&3&10.916&8.208&17.240&19.094\\ \hline
A1B16.wav&F10B22.wav&0&0&12&3&10.916&6.618&11.840&13.564\\ \hline
F14B26.wav&F2B20.wav&0&1&12&3&10.916&7.678&15.440&17.244\\ \hline
F4B22.wav&F16B28.wav&0&1&12&3&10.916&8.208&17.240&19.094\\ \hline
F4B14.wav&F16B8.wav&0&1&12&3&10.916&2.276&17.240&17.390\\ \hline
F10B16.wav&A1B22.wav&0&1&12&3&10.916&6.618&11.840&13.564\\ \hline
F3B18.wav&F15B24.wav&0&1&12&3&10.916&7.148&16.340&17.835\\ \hline
F4B22.wav&F16B16.wav&0&1&12&3&10.916&6.618&17.240&18.466\\ \hline
F16B2.wav&F4B8.wav&0&1&12&3&10.916&0.686&17.240&17.254\\ \hline
F9B4.wav&A1B22.wav&0&1&11&9&10.989&2.536&11.340&11.620\\ \hline
F10B22.wav&A2B4.wav&0&1&11&9&10.989&2.536&12.240&12.500\\ \hline
F13B2.wav&F2B20.wav&0&1&11&9&10.989&2.006&14.940&15.074\\ \hline
F5B26.wav&F16B8.wav&0&1&11&9&10.989&3.596&17.640&18.003\\ \hline
F13B28.wav&F1B20.wav&0&1&12&4&11.006&7.959&14.540&16.576\\ \hline
F15B18.wav&F3B10.wav&0&1&12&4&11.006&3.424&16.340&16.695\\ \hline
F10B14.wav&A1B22.wav&0&1&12&4&11.006&4.484&11.840&12.661\\ \hline
F11B26.wav&A2B18.wav&0&1&12&4&11.006&7.429&12.740&14.748\\ \hline
F13B16.wav&F1B8.wav&0&1&12&4&11.006&2.894&14.540&14.825\\ \hline
F16B28.wav&F6B4.wav&0&1&10&12&11.020&2.957&18.040&18.281\\ \hline
F4B2.wav&F14B26.wav&0&1&10&12&11.020&2.427&16.240&16.420\\ \hline
F6B4.wav&F16B28.wav&0&1&10&12&11.020&2.957&18.040&18.281\\ \hline
F1B2.wav&F11B26.wav&0&1&10&12&11.020&2.427&13.540&13.756\\ \hline
F13B28.wav&F3B4.wav&0&1&10&12&11.020&2.957&15.340&15.622\\ \hline
F12B10.wav&A3B20.wav&0&1&12&5&11.120&3.564&13.640&14.098\\ \hline
A1B16.wav&F10B2.wav&0&1&12&7&11.419&1.725&11.840&11.965\\ \hline
A1B16.wav&F10B2.wav&0&1&12&7&11.419&1.725&11.840&11.965\\ \hline
F10B22.wav&A1B8.wav&0&1&12&7&11.419&3.315&11.840&12.295\\ \hline
F13B22.wav&F1B8.wav&0&1&12&7&11.419&3.315&14.540&14.913\\ \hline
A1B16.wav&F10B2.wav&0&1&12&7&11.419&1.725&11.840&11.965\\ \hline
F4B2.wav&F16B16.wav&0&1&12&7&11.419&1.725&17.240&17.326\\ \hline
F14B20.wav&F1B20.wav&0&1&13&0&11.700&6.835&15.040&16.520\\ \hline
F12B4.wav&A2B4.wav&0&1&13&0&11.700&0.795&13.240&13.264\\ \hline
F12B4.wav&A2B4.wav&0&1&13&0&11.700&0.795&13.240&13.264\\ \hline
F1B8.wav&F14B6.wav&0&1&13&1&11.712&1.465&15.040&15.111\\ \hline
A1B22.wav&F11B26.wav&0&1&13&2&11.748&7.927&12.340&14.667\\ \hline
F11B26.wav&A1B22.wav&0&1&13&2&11.748&7.927&12.340&14.667\\ \hline
F3B18.wav&F16B14.wav&0&1&13&2&11.748&4.203&16.840&17.357\\ \hline
F1B2.wav&F14B6.wav&0&1&13&2&11.748&0.546&15.040&15.050\\ \hline
F11B12.wav&A1B8.wav&0&0&13&2&11.748&2.136&12.340&12.523\\ \hline
F1B20.wav&F13B2.wav&0&1&12&9&11.806&2.006&14.540&14.678\\ \hline
F12B4.wav&A2B10.wav&0&1&13&3&11.808&1.216&13.240&13.296\\ \hline
F16B2.wav&F3B10.wav&0&1&13&4&11.891&0.827&16.840&16.860\\ \hline
F2B12.wav&F15B4.wav&0&1&13&4&11.891&1.357&15.940&15.998\\ \hline
F1B2.wav&F14B12.wav&0&1&13&5&11.996&0.967&15.040&15.071\\ \hline
F16B8.wav&F3B18.wav&0&0&13&5&11.996&3.034&16.840&17.111\\ \hline
A1B22.wav&F11B12.wav&1&1&13&5&11.996&4.094&12.340&13.001\\ \hline
F10B28.wav&A1B8.wav&0&1&12&10&12.030&3.737&11.840&12.416\\ \hline
A1B8.wav&F11B20.wav&0&1&13&6&12.124&3.175&12.340&12.742\\ \hline
F16B16.wav&F3B4.wav&0&1&13&6&12.124&2.115&16.840&16.972\\ \hline
F11B26.wav&A2B4.wav&0&1&12&11&12.273&2.817&12.740&13.048\\ \hline
F14B12.wav&F1B26.wav&0&1&13&7&12.274&4.375&15.040&15.663\\ \hline
F1B20.wav&F14B6.wav&0&1&13&7&12.274&2.785&15.040&15.296\\ \hline
F14B20.wav&F1B2.wav&0&1&13&9&12.635&2.006&15.040&15.173\\ \hline
A1B2.wav&F11B20.wav&0&1&13&9&12.635&2.006&12.340&12.502\\ \hline
F12B4.wav&A1B8.wav&0&1&14&2&12.645&1.076&12.840&12.885\\ \hline
F12B10.wav&A1B2.wav&0&1&14&4&12.777&0.827&12.840&12.867\\ \hline
F1B2.wav&F13B28.wav&0&1&12&13&12.811&2.568&14.540&14.765\\ \hline
F1B20.wav&F15B10.wav&0&1&14&5&12.876&3.564&15.540&15.944\\ \hline
A2B18.wav&F13B8.wav&0&1&14&5&12.876&3.034&13.740&14.071\\ \hline
F13B28.wav&A2B18.wav&0&1&14&5&12.876&7.710&13.740&15.755\\ \hline
F12B4.wav&A1B16.wav&0&1&14&6&12.995&2.115&12.840&13.013\\ \hline
A1B22.wav&F12B10.wav&0&1&14&6&12.995&3.705&12.840&13.364\\ \hline
F16B14.wav&F2B26.wav&0&1&14&6&12.995&4.765&16.440&17.117\\ \hline
F1B8.wav&F15B24.wav&0&1&14&8&13.294&3.456&15.540&15.920\\ \hline
F15B4.wav&F1B20.wav&0&1&14&8&13.294&2.396&15.540&15.724\\ \hline
F13B28.wav&A2B10.wav&0&1&14&9&13.473&4.126&13.740&14.346\\ \hline
A2B4.wav&F13B22.wav&0&1&14&9&13.473&2.536&13.740&13.972\\ \hline
F16B2.wav&F1B2.wav&0&1&15&0&13.500&0.265&16.040&16.042\\ \hline
F1B8.wav&F16B8.wav&0&1&15&0&13.500&1.855&16.040&16.147\\ \hline
F16B14.wav&F1B14.wav&0&1&15&0&13.500&3.445&16.040&16.406\\ \hline
A2B18.wav&F14B20.wav&0&1&15&1&13.510&6.586&14.240&15.689\\ \hline
A2B4.wav&F14B6.wav&0&1&15&1&13.510&0.935&14.240&14.271\\ \hline
F1B14.wav&F16B16.wav&0&1&15&1&13.510&4.062&16.040&16.546\\ \hline
F16B28.wav&F1B26.wav&0&1&15&1&13.510&8.706&16.040&18.250\\ \hline
A1B2.wav&F13B8.wav&0&1&15&3&13.593&0.686&13.340&13.358\\ \hline
F1B8.wav&F16B2.wav&0&1&15&3&13.593&0.686&16.040&16.055\\ \hline
A2B18.wav&F14B12.wav&1&0&15&3&13.593&3.813&14.240&14.742\\ \hline
F16B14.wav&F1B2.wav&0&1&15&6&13.869&1.108&16.040&16.078\\ \hline
F15B24.wav&F1B2.wav&0&1&14&11&13.883&2.287&15.540&15.707\\ \hline
F13B16.wav&A1B2.wav&0&1&15&7&14.001&1.725&13.340&13.451\\ \hline
F15B10.wav&A2B10.wav&0&1&16&0&14.400&2.385&14.740&14.932\\ \hline
A1B8.wav&F14B12.wav&0&1&16&2&14.439&2.136&13.840&14.004\\ \hline
F16B16.wav&A3B20.wav&0&1&16&2&14.439&6.337&15.640&16.875\\ \hline
F14B12.wav&A1B8.wav&0&1&16&2&14.439&2.136&13.840&14.004\\ \hline
F1B2.wav&F16B22.wav&0&1&15&10&14.503&2.147&16.040&16.183\\ \hline
F14B6.wav&A1B16.wav&0&1&16&5&14.642&2.504&13.840&14.065\\ \hline
A1B16.wav&F14B6.wav&0&1&16&5&14.642&2.504&13.840&14.065\\ \hline
F14B12.wav&A1B22.wav&0&1&16&5&14.642&4.094&13.840&14.433\\ \hline
A1B2.wav&F13B28.wav&0&1&15&13&15.157&2.568&13.340&13.585\\ \hline
F16B16.wav&A2B18.wav&0&1&17&1&15.309&6.056&15.240&16.399\\ \hline
F15B10.wav&A1B8.wav&0&1&17&1&15.309&1.995&14.340&14.478\\ \hline
A2B4.wav&F16B8.wav&0&1&17&2&15.337&1.076&15.240&15.278\\ \hline
F16B14.wav&A2B18.wav&0&1&17&2&15.337&4.203&15.240&15.809\\ \hline
F15B10.wav&A1B16.wav&0&0&17&3&15.382&3.283&14.340&14.711\\ \hline
A1B2.wav&F15B10.wav&0&1&17&4&15.446&0.827&14.340&14.364\\ \hline
A1B2.wav&F14B26.wav&0&1&16&12&15.742&2.427&13.840&14.051\\ \hline
F14B26.wav&A1B2.wav&0&1&16&12&15.742&2.427&13.840&14.051\\ \hline
A1B8.wav&F15B24.wav&0&1&17&8&15.877&3.456&14.340&14.750\\ \hline
F16B8.wav&A1B8.wav&0&1&18&0&16.200&1.855&14.840&14.955\\ \hline
A1B8.wav&F16B16.wav&0&1&18&4&16.338&2.894&14.840&15.120\\ \hline
A1B2.wav&F16B14.wav&0&1&18&6&16.509&1.108&14.840&14.881\\ \hline
F16B8.wav&A1B22.wav&0&1&18&7&16.619&3.315&14.840&15.206\\ \hline
F16B2.wav&A1B22.wav&0&1&18&10&17.045&2.147&14.840&14.994\\ \hline
\caption{Resultados obtenidos del test final.}
\end{longtable}
\normalsize
Al igual que en con los resultados del test previo, las dos primeras columnas muestran el nombre de los dos archivos de audio de donde se puede extraer su posición. Los números detrás de las letras ``F'' y ``B'' marcan el número de fila y de butaca, respectivamente. La columna ``Respuesta'' marca ``0'' si se seleccionó que ambos audios eran diferentes y ``1'' si se seleccionó que eran iguales. La columna ``Seguridad'' determina el nivel de seguridad marcado en la respuesta: ``1'' si se está seguro, ``0'' si no. ``D.Fila'' muestra la distancia, en metros, entre las filas de los dos audios. ``D.Butacas'' arroja el mismo valor, pero entre la distancia del número de butaca. ``Distancia'' es la distancia entre ambas posiciones. ``D.fte X'' y ``D.fte Y'' son las distancias, en coordenada ``X'' e ``Y'', respectivamente,  de la posición intermedia entre ambas localizaciones. Por último, ``D.Fuente'' es la distancia de la posición intermedia entre ambas posiciones de butacas respecto de la fuete.
    
    \bibliography{biblio}
    \bibliographystyle{babunsrt}
\end{document}